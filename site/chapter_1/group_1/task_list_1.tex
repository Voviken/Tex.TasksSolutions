
<strong>1.</strong> Обчислити:
$$\frac{3+\sqrt{5}}{3-\sqrt{5}}-\frac{3\sqrt{5}}{2};$$

<strong>2.</strong> Обчислити:
$$\frac{\sqrt{10}+\sqrt{6}}{\sqrt{10}-\sqrt{6}} - \sqrt{15};$$

<strong>3.</strong> Обчислити:
$$\left(\sqrt{\left(0,5 - \sqrt{2}\right)^2} - \sqrt[3]{\left(1 + \sqrt{2}\right)^3}\;\right)^2;$$

<strong>4.</strong> Обчислити:
$$8 \cdot \left(\sqrt{\left(\sqrt{5} - 2,5\right)^2} - \sqrt[3]{\left(1 - \sqrt{5}\right)^3}\;\right)^2;$$

<strong>5.</strong> Обчислити:
$$\left(\sqrt{\left(0,5 - \sqrt{2}\right)} - \sqrt[3]{\left(-1 + \sqrt{2}\right)}\;\right)^3;$$

<strong>6.</strong> Обчислити:
$$1 + \frac{1 + 2^\frac{1}{2}}{3 + 2^\frac{1}{2}} : \frac{1}{2^\frac{3}{2} - 1};$$

<strong>7.</strong> обчислити:
$$1 + \frac{1 + 3^{1/2}}{4 + 3^{1/2}} : \frac{1}{3^{3/2} - 1};$$

<strong>8.</strong> Обчислити:
$$1 + \frac{1 + 30^{1/2}}{31 + 30^{1/2}} : \frac{1}{30^{3/2} - 1};$$

<strong>9.</strong> При якому значенні параметра $a$ вираз $25x^2 + 30x + a$ можна записати у вигляді повного квадрата суми двох одночленів?
$$\left(mx + n\right)^2 = m^2x^2 + 2mnx + n^2;$$

<strong>10.</strong> При якому значенні параметра $a$ вираз $36x^2 - 12x + a$ можна записати у вигляді
повного квадрата різниці двох одночленів?
$$\left(mx - n\right)^2 = m^2x^2 - 2mnx + n^2;$$

<strong>11.</strong> Дано вираз:
$$\left(\frac{1}{\sqrt{a} + \sqrt{a + 1}} + \frac{1}{\sqrt{a} - \sqrt{a - 1}}\right) \cdot \frac{\sqrt{a - 1}}{\sqrt{a + 1} + \sqrt{a - 1}}.$$
При якому значенні $a$ цей вираз набуває найменшого значення?

<strong>12.</strong> Дано вираз:
$$\frac{\sqrt[4]{\left(x + 1,1\right)^3} - 1,2^3}{\sqrt[4]{x + 1,1} - 1,2} - 1,2\sqrt[4]{x + 1,1}.$$
Яке його найменше значення?

<strong>13.</strong> Дано вираз:
$$\frac{\sqrt[4]{\left(x - 2\right)^3} - 4^3}{16 - 4\sqrt[4]{x - 2}} + \frac{\sqrt{x - 2}}{4}.$$
Яке його найбільше значення?

<strong>14.</strong> Обчислити:
$$\frac{x^5 + 64x^{-1}}{x^3 - 4x + 16x^{-1}} : \frac{x^2 + 4}{2}.$$
