\section*{Група 1}
\textbf{1.} Обчислити:
$$
\sin 15^\circ \cdot \cos 15^\circ.
$$
Скористаємося формулою $\sin 2\alpha = 2 \sin \alpha \cdot \cos \alpha$.
\begin{flalign*}
&\sin 15^\circ \cdot \cos 15^\circ =
\frac{1}{2} \cdot 2 \cdot \sin 15^\circ \cdot \cos 15^\circ =
\frac{1}{2} \cdot \sin \left(2 \cdot 15^\circ\right) =
\frac{1}{2} \cdot \sin 30^\circ =\\
&= \frac{1}{2} \cdot \frac{1}{2} = \frac{1}{4} = 0,25.
\end{flalign*}
\textbf{2.} Обчислити:
$$
6 \cdot \cos 75^\circ \cdot \cos 15^\circ.
$$
Скористаемося формулою
$$
\cos \alpha \cdot \cos \beta = \frac{1}{2} \cdot \left(\cos \left(\alpha + \beta\right) + \cos \left(\alpha - \beta\right)\right).
$$
\begin{flalign*}
&6 \cdot \cos 75^\circ \cdot \cos 15^\circ = 6 \cdot \frac{1}{2} \cdot \left(\cos\left(75^\circ + 15^\circ\right) + \cos \left(75^\circ - 15^\circ\right)\right) =\\
&= 3 \cdot \left(0 + \frac{1}{2}\right) = \frac{3}{2} = 1,5.
\end{flalign*}
\textbf{3.} Обчислити:
$$
12 \cdot \sin 15^\circ \cdot \sin 105^\circ.
$$
Скористаємося формулою
$$
\sin \alpha \cdot \sin \beta = \frac{\cos \left(\alpha - \beta\right) - \cos \left(\alpha + \beta\right)}{2}.
$$
\begin{flalign*}
&12 \cdot \sin 15^\circ \cdot \sin 105^\circ =
12 \cdot \frac{1}{2} \left(\cos \left(15^\circ - 105^\circ\right) - \cos \left(15^\circ + 105^\circ\right)\right) =\\
&= 6 \cdot \left(\cos \left(-90^\circ\right) - \cos 120^\circ\right) =
6 \cdot \left(0 - \left(-\frac{1}{2}\right)\right) = 3.
\end{flalign*}
\textbf{4.} Обчислити:
$$
15 \cdot \sin 165^\circ \cos 15^\circ.
$$
Скористаємося формулою
$$
\sin \alpha \cdot \cos \beta = \frac{\sin \left(\alpha + \beta\right) + \sin \left(\alpha - \beta\right)}{2}.
$$
\begin{flalign*}
&15 \cdot \sin 165^\circ \cos 15^\circ =
15 \cdot \frac{1}{2} \cdot \left(\sin \left(165^\circ + 15^\circ\right) + \sin \left(165^\circ - 15^\circ\right)\right) =\\
&= \frac{15}{2} \cdot \left(\sin 180^\circ + \sin 150^\circ\right) =
\frac{15}{2} \cdot \left(0 + \frac{1}{2}\right) = \frac{15}{4} = 3,75.
\end{flalign*}
\textbf{5.} Обчислити:
$$
3 \cdot \sin^2 30^\circ + 8 \cdot \cos^2 30^\circ.
$$
Не булемо робити жодних перетворень, а одразу підставимо значення.
\begin{flalign*}
&3 \cdot \sin^2 30^\circ + 8 \cdot \cos^2 30^\circ =
3 \cdot \left(\frac{1}{2}\right)^2 + 8 \cdot \left(\frac{\sqrt{3}}{2}\right)^2 = 
\frac{3}{4} + \frac{24}{4} = 6,75.
\end{flalign*}
\textbf{6.} Обчислити:
$$
5 \cdot \sin^2 60^\circ - 4 \cdot \cos^2 60^\circ.
$$
Скористаємося формулою
$$
\cos^2 \alpha + \sin^2 \alpha = 1.
$$
\begin{flalign*}
&5 \cdot \sin^2 60^\circ - 4 \cdot \cos^2 60^\circ =
5 \cdot \sin^2 60^\circ - 4 \cdot \left(1 - \sin^2 60^\circ\right) =\\
&= 9 \cdot \sin^2 60^\circ - 4 =
9 \cdot \frac{3}{4} - 4 = \frac{11}{4} = 2,75.
\end{flalign*}
\textbf{7.} Обчислити:
$$
2 \cdot \cos^2 45^\circ + 6 \cdot \sin^2 45^\circ.
$$
Не булемо робити жодних перетворень, а одразу підставимо значення.
\begin{flalign*}
&2 \cdot \cos^2 45^\circ + 6 \cdot \sin^2 45^\circ =
2 \cdot \left(\frac{\sqrt{2}}{2}\right)^2 + 6 \cdot \left(\frac{\sqrt{2}}{2}\right)^2 =
2 \cdot \frac{2}{4} + 6 \cdot \frac{2}{4} =\\
&= 1 + 3 = 4.
\end{flalign*}
\textbf{8.} Обчислити:
$$
9 \cdot \sin 120^\circ \cdot \tan 30^\circ.
$$
Не булемо робити жодних перетворень, а одразу підставимо значення.
\begin{flalign*}
&9 \cdot \sin 120^\circ \cdot \tan 30^\circ =
9 \cdot \frac{\sqrt{3}}{2} \cdot \frac{\sqrt{3}}{3} =
\frac{9}{2} = 4,5.
\end{flalign*}
\textbf{9.} Обчислити:
$$
15 \cdot \sin 120^\circ \cdot \tan 315^\circ.
$$
Не булемо робити жодних перетворень, а одразу підставимо значення.
\begin{flalign*}
&15 \cdot \sin 120^\circ \cdot \tan 315^\circ =
15 \cdot \left(- \frac{1}{2}\right) \cdot (-1) =
\frac{15}{2} = 7,5.
\end{flalign*}
\textbf{10.} Обчислити:
$$
10 \cdot \tan 35^\circ \cdot \cot 215^\circ.
$$
Тут скористаємося формулою
$$
\cot \alpha = \frac{1}{\tan \alpha}
$$
і тим фактом, що період $\cot$ є $180^\circ$.
\begin{flalign*}
&10 \cdot \tan 35^\circ \cdot \cot 215^\circ =
10 \cdot \tan 35^\circ \cdot \cot \left(180^\circ + 35^\circ\right) =\\
&= 10 \cdot \tan 35^\circ \cdot \cot 35^\circ =
10 \cdot \cancel{\tan 35^\circ} \cdot \frac{1}{\cancel{\tan 35^\circ}} = 10.
\end{flalign*}
\textbf{12.} Обчислити:
$$
\sin 15^\circ \cdot \cos 75^\circ + \cos 15^\circ \cdot \sin 75^\circ.
$$
Скористаємося формулою:
$$
\cos (\alpha + \beta) = \sin \alpha \cdot \cos \beta + \cos \alpha \cdot \sin \beta.
$$
Отримаємо:
\begin{flalign*}
&\sin 15^\circ \cdot \cos 75^\circ + \cos 15^\circ \cdot \sin 75^\circ = \cos (15^\circ + 75^\circ) = \cos 90^\circ = 0.
\end{flalign*}
\textbf{18.} Обчислити:
$$
\dfrac{(\sin x - \cos x)^2 - 1}{\sin 2x}.
$$
Піднесемо в чисельнику до квадрата і скористаємося формулами:
$$
\sin 2x = 2 \sin x \cdot \cos x, \;\;\; \sin^2 x + \cos^2 x = 1.
$$
Таким чином, ми отримаємо:
\begin{flalign*}
&\dfrac{(\sin x - \cos x)^2 - 1}{\sin 2x} =
\dfrac{\sin^2 x - 2 \sin x \cdot \cos x + \cos^2 x - 1}{\sin 2x} =\\
&= \dfrac{\cancel{1} - \sin 2x - \cancel{1}}{\sin 2x} =
\dfrac{-\sin 2x}{\sin 2x} = -1.
\end{flalign*}
\textbf{22.} Обчислити:
$$
\tan^2 x - \sin^2 x - \sin^2 x \cdot \tan^2 x.
$$
Винесемо за дужки $\tan^2 x$ і використаємо такі формули:
$$
\tan x = \dfrac{\sin x}{\cos x}, \;\;\;
1 - \sin^2 x = \cos^2 x.
$$
Отримаємо:
\begin{flalign*}
&\tan^2 x - \sin^2 x - \sin^2 x \cdot \tan^2 x =
\tan^2 x \cdot (1 - \sin^2 x) - \sin^2 x =\\
&= \dfrac{\sin^2 x}{\cancel{\cos^2 x}} \cdot \cancel{\cos^2 x} - \sin^2 x =
\sin^2 x - \sin^2 x = 0.
\end{flalign*}
\textbf{23.} Обчислення:
$$
3 + \dfrac{\tan 15^\circ - \tan60^\circ}{1 + \tan 15^\circ \cdot \tan 60^\circ}.
$$
Використаємо формулу:
$$
\tan (\alpha + \beta) = \dfrac{\tan \alpha - \tan \beta}{1 + \tan \alpha \cdot \tan \beta}.
$$
Отримаємо:
\begin{flalign*}
&3 + \dfrac{\tan 15^\circ - \tan60^\circ}{1 + \tan 15^\circ \cdot \tan 60^\circ} =
3 + \tan (15^\circ - 60^\circ) =\\
&= 3 - \tan 45^\circ = 3 - 1 = 2;
\end{flalign*}
\textbf{27.} Обчислити:
$$
\dfrac{2\sin x - 4\cos x}{5\sin x - 2\cos x}, \;\;\; \mbox{} \;\;\; \tan x = 2.
$$
Розділимо чисельник і знаменник на $\cos x$:
\begin{flalign*}
&\dfrac{2\sin x - 4\cos x}{5\sin x - 2\cos x} = 
\dfrac{\dfrac{1}{\cos x}}{\dfrac{1}{\cos x}} \cdot \dfrac{2\sin x - 4\cos x}{5\sin x - 2\cos x} =\\
&= \dfrac{\dfrac{2\sin x}{\cos x} - 4}{5\dfrac{\sin x}{\cos x} - 2} =
\dfrac{2 \tan x - 4}{5 \tan x - 2}.
\end{flalign*}
Після заміни отримаємо:
$$
\dfrac{2 \cdot 2 - 4}{5 \cdot 2 -2} = \dfrac{0}{3} = 0.
$$
\textbf{32.} Обчислити:
$$
\dfrac{14 \sin a - 10 \sin 2a}{14 \sin a + 10 \sin 2a}, \;\;\; \mbox{якщо} \;\;\; \cos a = 0,3.
$$
Скористаємося формулою:
$$
\sin 2a = 2 \sin a \cdot \cos a.
$$
Отримаємо:
\begin{flalign*}
&\dfrac{14 \sin a - 10 \sin 2a}{14 \sin a + 10 \sin 2a} =
\dfrac{14 \sin a - 20 \sin a \cos a}{14 \sin a + 20 \sin a \cos a} =
\dfrac{\cancel{2 \sin a} \cdot (7 - 10 \cos a)}{\cancel{2 \sin a} \cdot (7 + 10 \cos a)} =\\
&= \dfrac{7 - 10 \cos a}{7 + 10 \cos a}.
\end{flalign*}
Після заміни будемо мати:
$$
\dfrac{7 - 10 \cdot 0,3}{7 + 10 \cdot 0,3} = \dfrac{4}{10} = 0,4.
$$