\section*{Група 1}
\textbf{1.} Обчислити:
$$
\sin 15^\circ \cdot \cos 15^\circ.
$$
Скористаємося формулою $\sin 2\alpha = 2 \sin \alpha \cdot \cos \alpha$.
\begin{flalign*}
&\sin 15^\circ \cdot \cos 15^\circ =
\frac{1}{2} \cdot 2 \cdot \sin 15^\circ \cdot \cos 15^\circ =
\frac{1}{2} \cdot \sin \left(2 \cdot 15^\circ\right) =
\frac{1}{2} \cdot \sin 30^\circ =\\
&= \frac{1}{2} \cdot \frac{1}{2} = \frac{1}{4} = 0,25.
\end{flalign*}
\textbf{2.} Обчислити:
$$
6 \cdot \cos 75^\circ \cdot \cos 15^\circ.
$$
Скористаемося формулою
$$
\cos \alpha \cdot \cos \beta = \frac{1}{2} \cdot \left(\cos \left(\alpha + \beta\right) + \cos \left(\alpha - \beta\right)\right).
$$
\begin{flalign*}
&6 \cdot \cos 75^\circ \cdot \cos 15^\circ = 6 \cdot \frac{1}{2} \cdot \left(\cos\left(75^\circ + 15^\circ\right) + \cos \left(75^\circ - 15^\circ\right)\right) =\\
&= 3 \cdot \left(0 + \frac{1}{2}\right) = \frac{3}{2} = 1,5.
\end{flalign*}
\textbf{3.} Обчислити:
$$
12 \cdot \sin 15^\circ \cdot \sin 105^\circ.
$$
Скористаємося формулою
$$
\sin \alpha \cdot \sin \beta = \frac{\cos \left(\alpha - \beta\right) - \cos \left(\alpha + \beta\right)}{2}.
$$
\begin{flalign*}
&12 \cdot \sin 15^\circ \cdot \sin 105^\circ =
12 \cdot \frac{1}{2} \left(\cos \left(15^\circ - 105^\circ\right) - \cos \left(15^\circ + 105^\circ\right)\right) =\\
&= 6 \cdot \left(\cos \left(-90^\circ\right) - \cos 120^\circ\right) =
6 \cdot \left(0 - \left(-\frac{1}{2}\right)\right) = 3.
\end{flalign*}
\textbf{4.} Обчислити:
$$
15 \cdot \sin 165^\circ \cos 15^\circ.
$$
Скористаємося формулою
$$
\sin \alpha \cdot \cos \beta = \frac{\sin \left(\alpha + \beta\right) + \sin \left(\alpha - \beta\right)}{2}.
$$
\begin{flalign*}
&15 \cdot \sin 165^\circ \cos 15^\circ =
15 \cdot \frac{1}{2} \cdot \left(\sin \left(165^\circ + 15^\circ\right) + \sin \left(165^\circ - 15^\circ\right)\right) =\\
&= \frac{15}{2} \cdot \left(\sin 180^\circ + \sin 150^\circ\right) =
\frac{15}{2} \cdot \left(0 + \frac{1}{2}\right) = \frac{15}{4} = 3,75.
\end{flalign*}
\textbf{5.} Обчислити:
$$
3 \cdot \sin^2 30^\circ + 8 \cdot \cos^2 30^\circ.
$$
Не булемо робити жодних перетворень, а одразу підставимо значення.
\begin{flalign*}
&3 \cdot \sin^2 30^\circ + 8 \cdot \cos^2 30^\circ =
3 \cdot \left(\frac{1}{2}\right)^2 + 8 \cdot \left(\frac{\sqrt{3}}{2}\right)^2 = 
\frac{3}{4} + \frac{24}{4} = 6,75.
\end{flalign*}
\textbf{6.} Обчислити:
$$
5 \cdot \sin^2 60^\circ - 4 \cdot \cos^2 60^\circ.
$$
Скористаємося формулою
$$
\cos^2 \alpha + \sin^2 \alpha = 1.
$$
\begin{flalign*}
&5 \cdot \sin^2 60^\circ - 4 \cdot \cos^2 60^\circ =
5 \cdot \sin^2 60^\circ - 4 \cdot \left(1 - \sin^2 60^\circ\right) =\\
&= 9 \cdot \sin^2 60^\circ - 4 =
9 \cdot \frac{3}{4} - 4 = \frac{11}{4} = 2,75.
\end{flalign*}
\textbf{7.} Обчислити:
$$
2 \cdot \cos^2 45^\circ + 6 \cdot \sin^2 45^\circ.
$$
Не булемо робити жодних перетворень, а одразу підставимо значення.
\begin{flalign*}
&2 \cdot \cos^2 45^\circ + 6 \cdot \sin^2 45^\circ =
2 \cdot \left(\frac{\sqrt{2}}{2}\right)^2 + 6 \cdot \left(\frac{\sqrt{2}}{2}\right)^2 =
2 \cdot \frac{2}{4} + 6 \cdot \frac{2}{4} =\\
&= 1 + 3 = 4.
\end{flalign*}
\textbf{8.} Обчислити:
$$
9 \cdot \sin 120^\circ \cdot \tan 30^\circ.
$$
Не булемо робити жодних перетворень, а одразу підставимо значення.
\begin{flalign*}
&9 \cdot \sin 120^\circ \cdot \tan 30^\circ =
9 \cdot \frac{\sqrt{3}}{2} \cdot \frac{\sqrt{3}}{3} =
\frac{9}{2} = 4,5.
\end{flalign*}
\textbf{9.} Обчислити:
$$
15 \cdot \sin 120^\circ \cdot \tan 315^\circ.
$$
Не булемо робити жодних перетворень, а одразу підставимо значення.
\begin{flalign*}
&15 \cdot \sin 120^\circ \cdot \tan 315^\circ =
15 \cdot \left(- \frac{1}{2}\right) \cdot (-1) =
\frac{15}{2} = 7,5.
\end{flalign*}
\textbf{10.} Обчислити:
$$
10 \cdot \tan 35^\circ \cdot \cot 215^\circ.
$$
Тут скористаємося формулою
$$
\cot \alpha = \frac{1}{\tan \alpha}
$$
і тим фактом, що період $\cot$ є $180^\circ$.
\begin{flalign*}
&10 \cdot \tan 35^\circ \cdot \cot 215^\circ =
10 \cdot \tan 35^\circ \cdot \cot \left(180^\circ + 35^\circ\right) =\\
&= 10 \cdot \tan 35^\circ \cdot \cot 35^\circ =
10 \cdot \cancel{\tan 35^\circ} \cdot \frac{1}{\cancel{\tan 35^\circ}} = 10.
\end{flalign*}