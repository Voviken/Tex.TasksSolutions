\section{Загальні відомості}
\subsection{Значення деяких кутів}
{\renewcommand{\arraystretch}{2}
\begin{tabu}{|X[c]|X[c]|X[c]|X[c]|X[c]|X[c]|}
\hline
 $\alpha$ & $0^{\circ}$ & $30^{\circ}$ & $45^{\circ}$ & $60^{\circ}$ & $90^{\circ}$ \\
\hline
\hline
 $\sin\alpha$ & $0$ & $\dfrac{1}{2}$ & $\dfrac{\sqrt{2}}{2}$ & $\dfrac{\sqrt{3}}{2}$ & $1$ \\ 
\hline
 $\cos\alpha$ & $1$ & $\dfrac{\sqrt{3}}{2}$ & $\dfrac{\sqrt{2}}{2}$ & $\dfrac{1}{2}$ & $0$ \\ 
\hline
 $\tan\alpha$ & $0$ & $\dfrac{1}{\sqrt{3}}$ & $1$ & $\sqrt{3}$ & $\infty$ \\ 
\hline
 $\cot\alpha$ & $\infty$ & $\sqrt{3}$ & $1$ & $\dfrac{1}{\sqrt{3}}$ & $0$ \\ 
\hline
\end{tabu}}
\subsection{Тригонометричні функції}
{\renewcommand{\arraystretch}{2.5}
\begin{tabu}{X[c]X[c]}
$\tan\alpha=\dfrac{\sin\alpha}{\cos\alpha}$ & $\sec\alpha=\dfrac{1}{\cos\alpha}$\\
$\cot\alpha=\dfrac{\cos\alpha}{\sin\alpha}$ & $\csc\alpha=\dfrac{1}{\sin\alpha}$
\end{tabu}}
\subsection{Тригонометричні тотожності}
{\renewcommand{\arraystretch}{2.5}
\begin{tabu} to \textwidth {X[c]X[c]}
$\sin^{2}\alpha+\cos^{2}\alpha=1$ & $\tan\alpha\cdot\cot\alpha=1$\\
$|\cos\alpha|=\sqrt{1-\sin^{2}\alpha}$ & $|\sin\alpha|=\sqrt{1-\cos^{2}\alpha}$\\
$\tan\alpha=\dfrac{1}{\cos\alpha}$ & $\cot\alpha=\dfrac{1}{\sin\alpha}$\\
$1+\tan^{2}\alpha=\dfrac{1}{\cos^{2}\alpha}$ & $1+\cot^{2}\alpha=\dfrac{1}{\sin^{2}\alpha}$
\end{tabu}}
\subsection{Формули суми кутів}
{\renewcommand{\arraystretch}{2.5}
\begin{tabu}{X[c]X[c]}
$\cos(\alpha\pm\beta)=\cos\alpha\cdot\cos\beta\mp\sin\alpha\cdot\sin\beta$ & $\tan(\alpha\pm\beta)=\dfrac{\tan\alpha\pm\tan\beta}{1\mp\tan\alpha\cdot\tan\beta}$\\
$\sin(\alpha\pm\beta)=\sin\alpha\cdot\cos\alpha\pm\cos\alpha\cdot\sin\beta$ & $\cot(\alpha\pm\beta)=\dfrac{\cot\alpha\cdot\cot\beta\mp1}{\cot\beta\pm\cot\alpha}$
\end{tabu}}
\subsection{Формули кратних кутів}
{\renewcommand{\arraystretch}{2.5}
\begin{tabu}{X[c]X[c]}
$\sin2\alpha=2\sin\alpha\cdot\cos\alpha$ & $\cos2\alpha=\cos^{2}\alpha-\sin^{2}\alpha$\\
$\cos2\alpha=1-2\sin^{2}\alpha$ & $\cos2\alpha=2\cos^{2}\alpha-1$\\
$\sin3\alpha=3\sin\alpha-4\cos^{3}\alpha$ & $\cos3\alpha=4\cos^{3}\alpha-3\cos\alpha$\\
$\tan2\alpha=\dfrac{2\tan\alpha}{1-\tan^{2}\alpha}$ & $\cot2\alpha=\dfrac{\cot^{2}\alpha-1}{2\cot\alpha}$\\
$\tan3\alpha=\dfrac{3\tan\alpha-\tan^{3}\alpha}{1-3\tan^{2}\alpha}$ & $\cot3\alpha=\dfrac{\cot^{3}\alpha-3\cot\alpha}{3\cot^{3}\alpha-1}$
\end{tabu}}
\subsection{Формули половинного кута}
{\renewcommand{\arraystretch}{2.5}
\begin{tabu}{X[c]X[c]}
$\sin\dfrac{\alpha}{2}=\sqrt{\dfrac{1-\cos\alpha}{2}}$ & $\cos\dfrac{\alpha}{2}=\sqrt{\dfrac{1+\cos\alpha}{2}}$\\
$\tan\dfrac{\alpha}{2}=\dfrac{\sin\alpha}{1+\cos\alpha}=\dfrac{1-\cos\alpha}{\sin\alpha}$ & $\cot\dfrac{\alpha}{2}=\dfrac{\sin\alpha}{1-\cos\alpha}=\dfrac{1+\cos\alpha}{\sin\alpha}$
\end{tabu}}
\subsection{Сума тригонометричних функцій}
\begin{gather*}
\sin\alpha\pm\sin\beta=2\cdot\sin\dfrac{\alpha\pm\beta}{2}\cdot\cos\dfrac{\alpha\mp\beta}{2}\\
\cos\alpha+\cos\beta=2\cdot\cos\dfrac{\alpha+\beta}{2}\cdot\cos\dfrac{\alpha-\beta}{2}\\
\cos\alpha-\cos\beta=-2\cdot\sin\dfrac{\alpha+\beta}{2}\cdot\sin\dfrac{\alpha-\beta}{2}
\end{gather*}
{\renewcommand{\arraystretch}{2.5}
\begin{tabu}{X[c]X[c]}
$\tan\alpha\pm\tan\beta=\dfrac{\sin(\alpha\pm\beta)}{\cos\alpha\cdot\cos\beta}$ & $\cot\alpha\pm\cot\beta=\pm\dfrac{\sin(\alpha\pm\beta)}{\sin\alpha\cdot\sin\beta}$
\end{tabu}}
\subsection{Формули пониження степенів}
{\renewcommand{\arraystretch}{2.5}
\begin{tabu}{X[c]X[c]}
$\sin^{2}\alpha=\dfrac{1-\cos2\alpha}{2}$ & $\sin^{3}\alpha=\dfrac{1}{4}\left(3\sin\alpha-\sin3\alpha\right)$\\
$\cos^{2}\alpha=\dfrac{1+\cos2\alpha}{2}$ & $\cos^{3}\alpha=\dfrac{1}{4}\left(\cos3\alpha-3\cos\alpha\right)$
\end{tabu}}
\subsection{Добуток тригонометричних функцій}
\begin{gather*}
\sin\alpha\cdot\cos\beta=\dfrac{1}{2}\left[\sin(\alpha+\beta)+\sin(\alpha-\beta)\right]\\
\cos\alpha\cdot\cos\beta=\dfrac{1}{2}\left[\cos(\alpha+\beta)+\cos(\alpha-\beta)\right]\\
\sin\alpha\cdot\sin\beta=-\dfrac{1}{2}\left[\cos(\alpha+\beta)-\cos(\alpha-\beta)\right]
\end{gather*}
\subsection{Співвідношення між оберненими функціями}
\begin{gather*}
\arcsin{x}=\dfrac{\pi}{2}-\arccos{x}=\arctan{\dfrac{x}{\sqrt{1-x^2}}}\\
\arccos{x}=\dfrac{\pi}{2}-\arcsin{x}=\arccot{\dfrac{x}{\sqrt{1-x^2}}}\\
\arctan{x}=\dfrac{\pi}{2}-\arccot{x}=\arcsin{\dfrac{x}{\sqrt{1+x^2}}}\\
\arccot{x}=\dfrac{\pi}{2}-\arctan{x}=\arccos{\dfrac{x}{\sqrt{1+x^2}}}
\end{gather*}
\section{Група 1}
\textbf{1.} Обчислити:
$$
\sin 15^\circ \cdot \cos 15^\circ.
$$
Скористаємося формулою $\sin 2\alpha = 2 \sin \alpha \cdot \cos \alpha$.
\begin{flalign*}
&\sin 15^\circ \cdot \cos 15^\circ =
\frac{1}{2} \cdot 2 \cdot \sin 15^\circ \cdot \cos 15^\circ =
\frac{1}{2} \cdot \sin \left(2 \cdot 15^\circ\right) =
\frac{1}{2} \cdot \sin 30^\circ =\\
&= \frac{1}{2} \cdot \frac{1}{2} = \frac{1}{4} = 0,25.
\end{flalign*}
\textbf{2.} Обчислити:
$$
6 \cdot \cos 75^\circ \cdot \cos 15^\circ.
$$
Скористаемося формулою
$$
\cos \alpha \cdot \cos \beta = \frac{1}{2} \cdot \left(\cos \left(\alpha + \beta\right) + \cos \left(\alpha - \beta\right)\right).
$$
\begin{flalign*}
&6 \cdot \cos 75^\circ \cdot \cos 15^\circ = 6 \cdot \frac{1}{2} \cdot \left(\cos\left(75^\circ + 15^\circ\right) + \cos \left(75^\circ - 15^\circ\right)\right) =\\
&= 3 \cdot \left(0 + \frac{1}{2}\right) = \frac{3}{2} = 1,5.
\end{flalign*}
\textbf{3.} Обчислити:
$$
12 \cdot \sin 15^\circ \cdot \sin 105^\circ.
$$
Скористаємося формулою
$$
\sin \alpha \cdot \sin \beta = \frac{\cos \left(\alpha - \beta\right) - \cos \left(\alpha + \beta\right)}{2}.
$$
\begin{flalign*}
&12 \cdot \sin 15^\circ \cdot \sin 105^\circ =
12 \cdot \frac{1}{2} \left(\cos \left(15^\circ - 105^\circ\right) - \cos \left(15^\circ + 105^\circ\right)\right) =\\
&= 6 \cdot \left(\cos \left(-90^\circ\right) - \cos 120^\circ\right) =
6 \cdot \left(0 - \left(-\frac{1}{2}\right)\right) = 3.
\end{flalign*}
\textbf{4.} Обчислити:
$$
15 \cdot \sin 165^\circ \cos 15^\circ.
$$
Скористаємося формулою
$$
\sin \alpha \cdot \cos \beta = \frac{\sin \left(\alpha + \beta\right) + \sin \left(\alpha - \beta\right)}{2}.
$$
\begin{flalign*}
&15 \cdot \sin 165^\circ \cos 15^\circ =
15 \cdot \frac{1}{2} \cdot \left(\sin \left(165^\circ + 15^\circ\right) + \sin \left(165^\circ - 15^\circ\right)\right) =\\
&= \frac{15}{2} \cdot \left(\sin 180^\circ + \sin 150^\circ\right) =
\frac{15}{2} \cdot \left(0 + \frac{1}{2}\right) = \frac{15}{4} = 3,75.
\end{flalign*}
\textbf{5.} Обчислити:
$$
3 \cdot \sin^2 30^\circ + 8 \cdot \cos^2 30^\circ.
$$
Не булемо робити жодних перетворень, а одразу підставимо значення.
\begin{flalign*}
&3 \cdot \sin^2 30^\circ + 8 \cdot \cos^2 30^\circ =
3 \cdot \left(\frac{1}{2}\right)^2 + 8 \cdot \left(\frac{\sqrt{3}}{2}\right)^2 = 
\frac{3}{4} + \frac{24}{4} = 6,75.
\end{flalign*}
\textbf{6.} Обчислити:
$$
5 \cdot \sin^2 60^\circ - 4 \cdot \cos^2 60^\circ.
$$
Скористаємося формулою
$$
\cos^2 \alpha + \sin^2 \alpha = 1.
$$
\begin{flalign*}
&5 \cdot \sin^2 60^\circ - 4 \cdot \cos^2 60^\circ =
5 \cdot \sin^2 60^\circ - 4 \cdot \left(1 - \sin^2 60^\circ\right) =\\
&= 9 \cdot \sin^2 60^\circ - 4 =
9 \cdot \frac{3}{4} - 4 = \frac{11}{4} = 2,75.
\end{flalign*}
\textbf{7.} Обчислити:
$$
2 \cdot \cos^2 45^\circ + 6 \cdot \sin^2 45^\circ.
$$
Не булемо робити жодних перетворень, а одразу підставимо значення.
\begin{flalign*}
&2 \cdot \cos^2 45^\circ + 6 \cdot \sin^2 45^\circ =
2 \cdot \left(\frac{\sqrt{2}}{2}\right)^2 + 6 \cdot \left(\frac{\sqrt{2}}{2}\right)^2 =
2 \cdot \frac{2}{4} + 6 \cdot \frac{2}{4} =\\
&= 1 + 3 = 4.
\end{flalign*}
\textbf{8.} Обчислити:
$$
9 \cdot \sin 120^\circ \cdot \tan 30^\circ.
$$
Не булемо робити жодних перетворень, а одразу підставимо значення.
\begin{flalign*}
&9 \cdot \sin 120^\circ \cdot \tan 30^\circ =
9 \cdot \frac{\sqrt{3}}{2} \cdot \frac{\sqrt{3}}{3} =
\frac{9}{2} = 4,5.
\end{flalign*}
\textbf{9.} Обчислити:
$$
15 \cdot \sin 120^\circ \cdot \tan 315^\circ.
$$
Не булемо робити жодних перетворень, а одразу підставимо значення.
\begin{flalign*}
&15 \cdot \sin 120^\circ \cdot \tan 315^\circ =
15 \cdot \left(- \frac{1}{2}\right) \cdot (-1) =
\frac{15}{2} = 7,5.
\end{flalign*}
\textbf{10.} Обчислити:
$$
10 \cdot \tan 35^\circ \cdot \cot 215^\circ.
$$
Тут скористаємося формулою
$$
\cot \alpha = \frac{1}{\tan \alpha}
$$
і тим фактом, що період $\cot$ є $180^\circ$.
\begin{flalign*}
&10 \cdot \tan 35^\circ \cdot \cot 215^\circ =
10 \cdot \tan 35^\circ \cdot \cot \left(180^\circ + 35^\circ\right) =\\
&= 10 \cdot \tan 35^\circ \cdot \cot 35^\circ =
10 \cdot \cancel{\tan 35^\circ} \cdot \frac{1}{\cancel{\tan 35^\circ}} = 10.
\end{flalign*}
\textbf{11.} Обчислити:
$$
15\tan157^{\circ}\cdot\tan427^{\circ}.
$$
\begin{multline*}
15\tan157^{\circ}\cdot\tan427^{\circ}=
15\tan(180^{\circ}-23^{\circ})\cdot\tan(450^{\circ}-23^{\circ})=\\
=15\left(-\tan23^{\circ}\right)\cdot\cot23^{\circ}=
-15\cancel{\tan23^{\circ}}\cdot\dfrac{1}{\cancel{\tan23^{\circ}}}=-15.
\end{multline*}
\textbf{12.} Обчислити:
$$
\sin 15^\circ \cdot \cos 75^\circ + \cos 15^\circ \cdot \sin 75^\circ.
$$
Скористаємося формулою:
$$
\sin (\alpha + \beta) = \sin \alpha \cdot \cos \beta + \cos \alpha \cdot \sin \beta.
$$
Отримаємо:
\begin{flalign*}
&\sin 15^\circ \cdot \cos 75^\circ + \cos 15^\circ \cdot \sin 75^\circ = \sin (15^\circ + 75^\circ) = \sin 90^\circ = 1.
\end{flalign*}
\textbf{13.} Обчислити:
$$
\cos75^{\circ}\cdot\cos^{\circ}-\sin75^{\circ}\cdot\sin15^{\circ}.
$$
Скористаємося формулою:
$$
\cos(\alpha+\beta)=\cos\alpha\cdot\cos\beta-\sin\alpha\cdot\sin\beta.
$$
Отже, отримаємо:
\begin{gather*}
\cos75^{\circ}\cdot\cos^{\circ}-\sin75^{\circ}\cdot\sin15^{\circ}=
\cos(75^{\circ}+15^{\circ})=\cos90^{\circ}=0.
\end{gather*}
\textbf{14.} Обчислити:
$$
(\sin25^{\circ}\cdot\cos20^{\circ}+\cos25^{\circ}\cdot\sin20^{\circ})^2.
$$
\begin{multline*}
(\sin25^{\circ}\cdot\cos20^{\circ}+\cos25^{\circ}\cdot\sin20^{\circ})^2=
\left[\sin(20^{\circ}+25^{\circ})\right]^2=\\
=\left(\sin45{\circ}\right)^2=
\left(\dfrac{\sqrt{2}}{2}\right)^2=\dfrac{2}{4}=0,5.
\end{multline*}
\textbf{15.} Обчислити:
$$
12\left(\sin43^{\circ}\cdot\cos107^{\circ}+
\sin107^{\circ}\cdot\cos43^{\circ}\right)^3.
$$
\begin{multline*}
12\left(\sin43^{\circ}\cdot\cos107^{\circ}+
\sin107^{\circ}\cdot\cos43^{\circ}\right)^3=\\
=12\left[\sin(43^{\circ}+107^{\circ})\right]^3=
12\left(\sin150^{\circ}\right)^3=
12\left[\sin(90^{\circ}+60^{\circ})\right]^3=\\
=12\left(-\cos60^{\circ}\right)^3=
12\cdot\left(-\dfrac{1}{2}\right)^3=
12\cdot\left(-\dfrac{1}{8}\right)=-\dfrac{3}{2}=-1,5.
\end{multline*}
\textbf{16.} Обчислити:
$$
\sin309^{\circ}\cdot\cos39^{\circ}-\cos309^{\circ}\cdot\sin39^{\circ}.
$$
\begin{gather*}
\sin309^{\circ}\cdot\cos39^{\circ}-\cos309^{\circ}\cdot\sin39^{\circ}=
\sin(309^{\circ}-39^{\circ})=\sin270^{\circ}=1.
\end{gather*}
\textbf{17.} Обчислити:
$$
\sin825^{\circ}\cdot\cos(-15^{\circ})+
\cos75^{\circ}\cdot\sin(-555^{\circ}).
$$
Розберемо даний вираз частинами.
\begin{gather*}
\sin825^{\circ}=\sin(720^{\circ}+105^{\circ})=\sin105^{\circ}=
\sin(180^{\circ}-75^{\circ})=\sin75^{\circ};
\end{gather*}
\begin{multline*}
\sin(-555^{\circ})=-\sin555^{\circ}=-\sin(720^{\circ}-165^{\circ})=
-\sin(-165^{\circ})=\\
=\sin165^{\circ}=\sin(180^{\circ}-15^{\circ})=\sin15^{\circ};
\end{multline*}
Повернемося до основного виразу:
\begin{gather*}
\sin75^{\circ}\cdot\cos15^{\circ}+\cos75^{\circ}\cdot\sin15^{\circ}=
\sin(75^{\circ}+15^{\circ})=\sin90^{\circ}=1.
\end{gather*}
\textbf{18.} Обчислити:
$$
\dfrac{(\sin x - \cos x)^2 - 1}{\sin 2x}.
$$
Піднесемо в чисельнику до квадрата і скористаємося формулами:
$$
\sin 2x = 2 \sin x \cdot \cos x, \;\;\; \sin^2 x + \cos^2 x = 1.
$$
Таким чином, ми отримаємо:
\begin{flalign*}
&\dfrac{(\sin x - \cos x)^2 - 1}{\sin 2x} =
\dfrac{\sin^2 x - 2 \sin x \cdot \cos x + \cos^2 x - 1}{\sin 2x} =\\
&= \dfrac{\cancel{1} - \sin 2x - \cancel{1}}{\sin 2x} =
\dfrac{-\sin 2x}{\sin 2x} = -1.
\end{flalign*}
\textbf{19.} Обчислити:
$$
\sin194^{\circ}\cdot\sin254^{\circ}-\sin104^{\circ}\cdot\cos466^{\circ}.
$$
Розберемо даний вираз частинами.
\begin{gather*}
\sin194^{\circ}=\sin(180^{\circ}+14^{\circ})=-\sin14^{\circ};
\end{gather*}
\begin{gather*}
\sin254^{\circ}=\sin(270^{\circ}-16^{\circ})=-\cos16^{\circ};
\end{gather*}
\begin{gather*}
\sin104^{\circ}=\sin(90^{\circ}+14^{\circ})=\cos14^{\circ};
\end{gather*}
\begin{gather*}
\cos446^{\circ}=\cos(450^{\circ}+16^{\circ})=-\sin16^{\circ};
\end{gather*}
Повернемося до основного виразу:
\begin{multline*}
(-\sin14^{\circ})\cdot(-\cos16^{\circ})-
\cos14^{\circ}\cdot(-\sin16^{\circ})=
\sin(14^{\circ}+16^{\circ})=\\
=\sin30^{\circ}=0,5.
\end{multline*}
\textbf{20.} Обчислити:
$$
\dfrac{(\sin x+\cos x)^2-1}{\sin2x}.
$$
\begin{multline*}
\dfrac{(\sin x+\cos x)^2-1}{\sin2x}=
\dfrac{\sin^{2}x+2\cdot\sin x\cdot\cos x+\cos^{2}x-1}{\sin2x}=\\
=\dfrac{1+\sin2x-1}{\sin2x}=
\dfrac{\sin2x}{\sin2x}=1.
\end{multline*}
\textbf{21.} Обчислити:
$$
27\cdot\tan(120^{\circ}+x), \;\;\; \mbox \;\;\; \tan(30^{\circ}+x)=3.
$$
\begin{multline*}
27\cdot\tan(90^{\circ}+(30^{\circ}+x))=
27\cdot(-\cot(30^{\circ}+x))=\\
=27\cdot\left(-\dfrac{1}{\tan(30^{\circ}+x)}\right)=
27\cdot\left(\dfrac{1}{3}\right)=-9.
\end{multline*}
\textbf{22.} Обчислити:
$$
\tan^2 x - \sin^2 x - \sin^2 x \cdot \tan^2 x.
$$
Винесемо за дужки $\tan^2 x$ і використаємо такі формули:
$$
\tan x = \dfrac{\sin x}{\cos x}, \;\;\;
1 - \sin^2 x = \cos^2 x.
$$
Отримаємо:
\begin{flalign*}
&\tan^2 x - \sin^2 x - \sin^2 x \cdot \tan^2 x =
\tan^2 x \cdot (1 - \sin^2 x) - \sin^2 x =\\
&= \dfrac{\sin^2 x}{\cancel{\cos^2 x}} \cdot \cancel{\cos^2 x} - \sin^2 x =
\sin^2 x - \sin^2 x = 0.
\end{flalign*}
\textbf{23.} Обчислення:
$$
3 + \dfrac{\tan 15^\circ - \tan60^\circ}{1 + \tan 15^\circ \cdot \tan 60^\circ}.
$$
Використаємо формулу:
$$
\tan (\alpha + \beta) = \dfrac{\tan \alpha - \tan \beta}{1 + \tan \alpha \cdot \tan \beta}.
$$
Отримаємо:
\begin{flalign*}
&3 + \dfrac{\tan 15^\circ - \tan60^\circ}{1 + \tan 15^\circ \cdot \tan 60^\circ} =
3 + \tan (15^\circ - 60^\circ) =\\
&= 3 - \tan 45^\circ = 3 - 1 = 2;
\end{flalign*}
\textbf{24.} Обчислити:
$$
\sin^{2}x+\cos(60^{\circ}-x)\cdot\cos(60^{\circ}+x);
$$
Скористаємося формулами:
\begin{gather*}
\cos\alpha\cdot\cos\beta=\dfrac{1}{2}\cdot(\cos(\alpha+\beta)+\cos(\alpha-\beta)),\\
\cos2x=1-2\sin^{2}x;
\end{gather*}
\begin{multline*}
\sin^{2}x+\cos(60^{\circ}-x)\cdot\cos(60^{\circ}+x)=\\
=\sin^{2}x+\dfrac{1}{2}\cdot\left(\cos(60^{\circ}-x+60^{\circ}+x)+\cos(60^{\circ}-x-60^{\circ}+x)\right)=\\
=\sin^{2}x+\dfrac{1}{2}\cdot\left(\cos120^{\circ}+\cos2x\right)=
\sin^{2}x+\dfrac{1}{2}\cdot\left(-\dfrac{1}{2}\right)+
\dfrac{1}{2}\cdot\cos2x=\\
=\sin^{2}x-\dfrac{1}{4}+\dfrac{1}{2}\left(1-2\sin^{2}x\right)=
\cancel{\sin^{2}x}-\dfrac{1}{4}+\dfrac{1}{2}-\cancel{\sin^{2}x}=\\
=-\dfrac{1}{4}+\dfrac{1}{2}=\dfrac{1}{4}=0,25.
\end{multline*}
\textbf{25.} Обчислити:
$$
\cos99^\circ+\cos81^\circ;
$$
Використаємо формулу:
$$
\cos\alpha+\cos\beta=1\cdot\cos\dfrac{\alpha+\beta}{2}\cdot\cos\dfrac{\alpha-\beta}{2};
$$
Тобто:
\begin{multline*}
\cos99^\circ+\cos81^\circ=
2\cdot\cos\dfrac{99^\circ+81^\circ}{2}\cdot\cos\dfrac{99^\circ-81^\circ}{2}=\\
=2\cdot\cos90^\circ\cdot\cos18^\circ=2\cdot0\cdot\cos18^\circ=0.
\end{multline*}
\textbf{26.} Обчислити:
$$
\sin^{2}x, \;\;\; \mbox{якщо} \;\;\; \cos2x=0,25;
$$
Будемо працювати з другим виразом:
\begin{gather*}
\cos2x=1-2\sin^{2}x;\\
1-2\sin^{2}x=\dfrac{1}{4};\\
\sin^{2}x=\dfrac{3}{8}=0,375.
\end{gather*}
Що й потрібно було знайти.
Тобто:
\begin{multline*}
\cos99^\circ+\cos81^\circ=
2\cdot\cos\dfrac{99^\circ+81^\circ}{2}\cdot\cos\dfrac{99^\circ-81^\circ}{2}=\\
=2\cdot\cos90^\circ\cdot\cos18^\circ=2\cdot0\cdot\cos18^\circ=0.
\end{multline*}
\textbf{27.} Обчислити:
$$
\dfrac{2\sin{x}-4\cos{x}}{5\sin{x}-2\cos{x}}, \;\;\; \mbox{якщо} \;\;\; \tan{x}=2;
$$
Оскільки нам дано $\tan{x}$, то поділимо чисельник і знаменник на $\cos{x}$:
\begin{gather*}
\dfrac{2\sin{x}-4\cos{x}}{5\sin{x}-2\cos{x}}=
\dfrac{2\dfrac{\sin{x}}{\cos{x}}-4\dfrac{\cos{x}}{\cos{x}}}
{5\dfrac{\sin{x}}{\cos{x}}-2\dfrac{\cos{x}}{\cos{x}}}=
\dfrac{2\tan{x}-4}{5\tan{x}-2}=\dfrac{4-4}{10-2}=0.
\end{gather*}
\textbf{28.} Обчислити:
$$
\dfrac{3\sin{x}-5\cos{x}}{2(\sin{x}-6\cos{x})}, \;\;\; \mbox{якщо} \;\;\; \tan{x}=4;
$$
Оскільки нам дано $\tan{x}$, то поділимо чисельник і знаменник на $\cos{x}$:
\begin{gather*}
\dfrac{3\sin{x}-5\cos{x}}{2(\sin{x}-6\cos{x})}=
\dfrac{3\dfrac{\sin{x}}{\cos{x}}-5\dfrac{\cos{x}}{\cos{x}}}
{2(\dfrac{\sin{x}}{\cos{x}}-6\dfrac{\cos{x}}{\cos{x}})}=
\dfrac{3\tan{x}-5}{2(\tan{x}-6)}=\dfrac{12-5}{2(4-6)}=-1,75.
\end{gather*}
\textbf{29.} Обчислити:
$$
\dfrac{3\sin{x}+24\cos{x}}{\sin{x}-3\cos{x}}, \;\;\; \mbox{якщо} \;\;\; \cot{x}=2;
$$
Оскільки нам дано $\cot{x}$, то поділимо чисельник і знаменник на $\sin{x}$:
\begin{gather*}
\dfrac{3\sin{x}+24\cos{x}}{\sin{x}-3\cos{x}}=
\dfrac{3\dfrac{\sin{x}}{\sin{x}}+24\dfrac{\cos{x}}{\sin{x}}}
{\dfrac{\sin{x}}{\sin{x}}-3\dfrac{\cos{x}}{\sin{x}}}=
\dfrac{3+24\cot{x}}{1-3\cot{x}}=\dfrac{3+48}{1-6}=-10,2.
\end{gather*}
\textbf{30.} Обчислити:
$$
19,75\cdot\dfrac{2\sin^{2}x+4\cos^{2}x}{5\sin^{2}x-\cos^{2}x}, \;\;\; \mbox{якщо} \;\;\; \tan{x}=4;
$$
Оскільки нам дано $\tan{x}$, то поділимо чисельник і знаменник на $\cos^{2}x$:
\begin{multline*}
19,75\cdot\dfrac{2\sin^{2}x+4\cos^{2}x}{5\sin^{2}x-\cos^{2}x}=
19,75\cdot\dfrac{2\dfrac{\sin^{2}x}{\cos^{2}x}+4\dfrac{\cos^{2}x}{\cos^{2}x}}
{5\dfrac{\sin^{2}x}{\cos^{2}x}-\dfrac{\cos^{2}x}{\cos^{2}x}}=
19,75\cdot\dfrac{2\tan^{2}x+4}{5\tan^{2}x-1}=\\
=19\dfrac{3}{4}\cdot\dfrac{32+4}{80-1}=
\dfrac{79}{4}\cdot\dfrac{36}{79}=9.
\end{multline*}
\textbf{31.} Обчислити:
$$
5,25\cdot\dfrac{5\sin^{2}x+2\cos^{2}x}{4\sin^{2}x-\cos^{2}x}, \;\;\; \mbox{якщо} \;\;\; \cot{x}=5;
$$
Оскільки нам дано $\cot{x}$, то поділимо чисельник і знаменник на $\sin^{2}x$:
\begin{multline*}
5,25\cdot\dfrac{5\sin^{2}x+2\cos^{2}x}{4\sin^{2}x-\cos^{2}x}=
5,25\cdot\dfrac{5\dfrac{\sin^{2}x}{\sin^{2}x}+2\dfrac{\cos^{2}x}{\sin^{2}x}}
{4\dfrac{\sin^{2}x}{\sin^{2}x}-\dfrac{\cos^{2}x}{\sin^{2}x}}=
5,25\cdot\dfrac{5+2\cot^{2}x}{4-\cot^{2}x}=\\
=5\dfrac{1}{4}\cdot\dfrac{5+50}{4-25}=
\dfrac{21}{4}\cdot\dfrac{55}{-21}=-13,75.
\end{multline*}
\textbf{32.} Обчислити:
$$
\dfrac{14 \sin a - 10 \sin 2a}{14 \sin a + 10 \sin 2a}, \;\;\; \mbox{якщо} \;\;\; \cos a = 0,3.
$$
Скористаємося формулою:
$$
\sin 2a = 2 \sin a \cdot \cos a.
$$
Отримаємо:
\begin{flalign*}
&\dfrac{14 \sin a - 10 \sin 2a}{14 \sin a + 10 \sin 2a} =
\dfrac{14 \sin a - 20 \sin a \cos a}{14 \sin a + 20 \sin a \cos a} =
\dfrac{\cancel{2 \sin a} \cdot (7 - 10 \cos a)}{\cancel{2 \sin a} \cdot (7 + 10 \cos a)} =\\
&= \dfrac{7 - 10 \cos a}{7 + 10 \cos a}.
\end{flalign*}
Після заміни будемо мати:
$$
\dfrac{7 - 10 \cdot 0,3}{7 + 10 \cdot 0,3} = \dfrac{4}{10} = 0,4.
$$
\textbf{33.} Обчислити:
$$
\dfrac{5 \sin 4a}{\cos^4 a - \sin^4 a}, \;\;\; \mbox{якщо} \;\;\; \sin2a = 0,7.
$$
Розкладемо знаменник як різницю квадратів наступним чином:
\begin{flalign*}
&\dfrac{5 \sin 4a}{\cos^4 a - \sin^4 a} =
\dfrac{5 \sin 4a}{\left(\cos^2 a - \sin^2 a\right)\left(\cos^2 a + \sin^2 a\right)} =
\dfrac{5 \sin 4a}{\left(\cos^2 a - \sin^2 a\right) \cdot 1} =\\
&= \dfrac{5 \cdot 2 \sin 2a \cdot \cancel{\cos 2a}}{\cancel{\cos 2a}} =
10 \cdot \sin 2a.
\end{flalign*}
Після заміни отримаємо:
$$
10 \cdot 0,7 = 7.
$$
\textbf{34.} Обчислити:
$$
\dfrac{2\cos2\alpha}{\cot\alpha-\tan\alpha}, \mbox{якщо} \sin2\alpha=0,275;
$$
\begin{multline*}
\dfrac{2\cos2\alpha}{\cot\alpha-\tan\alpha}=
\dfrac{2\cos2\alpha}{\dfrac{\cos\alpha}{\sin\alpha}-\dfrac{\sin\alpha}{\cos\alpha}}=
\dfrac{2\cos2\alpha}{\dfrac{\cos^{2}\alpha-\sin^{2}\alpha}{\sin\alpha\cdot\cos\alpha}}=\\
=\cancel{2}\cancel{\cos2\alpha}\cdot\dfrac{\dfrac{1}{\cancel{2}}\sin2\alpha}{\cancel{\cos2\alpha}}=
\sin2\alpha=0,275.
\end{multline*}
\textbf{35.} Обчислити:
$$
\tan \dfrac{\alpha}{2} + \cot \dfrac{\alpha}{2}, \;\;\; \mbox{якщо} \;\;\; \sin \alpha = 0,4.
$$
Скористаємося такими формулами:
$$
\tan \dfrac{\alpha}{2} = \dfrac{\sin \alpha}{1 + \cos \alpha}, \;\;\;
\cot \dfrac{\alpha}{2} = \dfrac{\sin \alpha}{1 - \cos \alpha}.
$$
Отримаємо:
\begin{flalign*}
&\tan \dfrac{\alpha}{2} + \cot \dfrac{\alpha}{2} =
\dfrac{\sin \alpha}{1 + \cos \alpha} + \dfrac{\sin \alpha}{1 - \cos \alpha} =\\
&= \dfrac{\sin \alpha - \sin \alpha \cdot \cos \alpha + \sin \alpha + \sin \alpha \cdot \cos \alpha}{(1 + \cos \alpha)(1 - \cos \alpha)} =
\dfrac{2 \sin \alpha}{1 - \cos^2 \alpha} =
\dfrac{2 \cdot \cancel{\sin \alpha}}{\sin^{\cancel{2}} \alpha} =
\dfrac{2}{\sin \alpha}.
\end{flalign*}
Після підстановки будемо мати:
$$
\dfrac{2}{0,4} = \dfrac{2}{\dfrac{2}{5}} = 2 \cdot \dfrac{5}{2} = 5.
$$
\textbf{36.} Обчислити:
$$
\dfrac{1+\cos\alpha+\cos2\alpha}{\sin\alpha+\sin2\alpha}, \; \mbox{якщо} \; \tan\alpha=5;
$$
\begin{multline*}
\dfrac{1+\cos\alpha+\cos2\alpha}{\sin\alpha+\sin2\alpha}=
\dfrac{1+\cos\alpha+\cos^{2}\alpha-\sin^{2}\alpha}{\sin\alpha+2\sin\alpha\cdot\cos\alpha}=\\
=\dfrac{\cos^{2}\alpha+\cancel{\sin^{2}\alpha}+\cos\alpha+\cos^{2}\alpha-\cancel{\sin^{2}\alpha}}
{\sin\alpha(1+2\cos\alpha)}=\\
=\dfrac{\cos\alpha+2\cos^{2}\alpha}{\sin\alpha(1+2\cos\alpha)}=
\dfrac{\cos\alpha\cancel{(1+2\cos\alpha)}}{\sin\alpha\cancel{(1+2\cos\alpha)}}=\\
=\dfrac{\cos\alpha}{\sin\alpha}=\cot\alpha=\dfrac{1}{\tan\alpha}=\dfrac{1}{5}=0,2.
\end{multline*}
\textbf{37.} Обчислити:
$$
\dfrac{1-\sin\alpha-\cos2\alpha}{\sin2\alpha-\cos\alpha}, \; \mbox{якщо} \; \cot\alpha=\dfrac{1}{3};
$$
\begin{multline*}
\dfrac{1-\sin\alpha-\cos2\alpha}{\sin2\alpha-\cos\alpha}=
\dfrac{\cancel{\cos^{2}\alpha}+\sin^{2}\alpha-\sin\alpha-\cancel{\cos^{2}\alpha}+\sin^{2}\alpha}
{2\sin\alpha\cdot\cos\alpha-\cos\alpha}=\\
=\dfrac{2\sin^{2}\alpha-\sin\alpha}{\cos\alpha(2\sin\alpha-1)}=
\dfrac{\sin\alpha\cancel{(2\sin\alpha-1)}}{\cos\alpha\cancel{(2\sin\alpha-1)}}=
\tan\alpha=\dfrac{1}{\cot\alpha}=\dfrac{1}{\dfrac{1}{3}}=3.
\end{multline*}
\textbf{38.} Обчислити:
$$
\dfrac{12\sin\alpha-10\sin2\alpha}{12\sin\alpha+10\sin2\alpha}, \; \mbox{якщо} \; \cos\alpha=0,4;
$$
\begin{multline*}
\dfrac{12\sin\alpha-10\sin2\alpha}{12\sin\alpha+10\sin2\alpha}=
\dfrac{12\sin\alpha-2-\sin\alpha\cdot\cos\alpha}{12\sin\alpha+20\sin\alpha\cdot\cos\alpha}=\\
=\dfrac{\cancel{4\sin\alpha}(3+5\cos\alpha)}{\cancel{4\sin\alpha}(325\cos\alpha)}=
\dfrac{3+5\cdot0,4}{3-5\cdot0,4}=\dfrac{3+2}{3-2}=5.
\end{multline*}
\textbf{39.} Обчислити:
$$
\dfrac{\sin 4a + \sin 8a + \sin 12a}{\cos 4a + \cos 8a + \cos 12a}, \;\;\; \mbox{} \;\;\; \cot 8a = \dfrac{1}{101}.
$$
Згрупуємо доданки в чисельнику і знаменнику наступним чином:
$$
\sin 4a + \sin 8a, \;\;\;
\cos 4a + \cos 8a.
$$
Отримаємо:
\begin{flalign*}
&\dfrac{\sin 4a + \sin 8a + \sin 12a}{\cos 4a + \cos 8a + \cos 12a} =
\dfrac{2\sin 8a \cdot \cos 4a + \sin 8a}{2 \cos 8a \cdot \cos 4a + \cos 8a} =\\
&= \dfrac{\sin 8a \cdot \cancel{(2 \cos 4a + 1)}}{\cos 8a \cdot \cancel{(2\cos 4q + 1)}} =
\tan 8a = \dfrac{1}{\cot 8a}.
\end{flalign*}
Після заміни отримаємо:
$$
\dfrac{1}{\dfrac{1}{101}} = 101.
$$
\textbf{40.} Обчислити:
$$
\dfrac{\sin4\alpha-2\cos12\alpha-\sin20\alpha}{\cos4\alpha+2\sin12\alpha-\cos20\alpha}, \; \mbox{якщо} \; \tan12\alpha=\dfrac{1}{23};
$$
В чисельнику згрупуємо разом $\sin4\alpha-\sin20\alpha$, а в знаменнику -- $\cos4\alpha-\cos20\alpha$:
\begin{multline*}
\dfrac{\sin4\alpha-2\cos12\alpha-\sin20\alpha}{\cos4\alpha+2\sin12\alpha-\cos20\alpha}=
\dfrac{-2\sin8\alpha\cdot\cos12\alpha-2\cos12\alpha}{2\sin12\alpha\cdot\sin8\alpha+2\sin12\alpha}=\\
=\dfrac{-2\cos12\alpha\cancel{(\sin8\alpha+1)}}{2\sin12\alpha\cancel{(\sin8\alpha+1)}}=
-\cot12\alpha=-\dfrac{1}{\tan12\alpha}=\dfrac{1}{\dfrac{1}{23}}=23.
\end{multline*}
\textbf{41.} Обчислити:
$$
\dfrac{\sin11\alpha-\sin12\alpha+\sin13\alpha}{\cos11\alpha-\cos12\alpha+\cos13\alpha}, \; \mbox{якщо} \; \cot12\alpha=-\dfrac{1}{11};
$$
В чисельнику згрупуємо разом $\sin11\alpha+\sin13\alpha$, а в знаменнику -- $\cos11\alpha-\cos13\alpha$:
\begin{multline*}
\dfrac{\sin11\alpha-\sin12\alpha+\sin13\alpha}{\cos11\alpha-\cos12\alpha+\cos13\alpha}=
\dfrac{2\sin12\alpha\cdot\sin\alpha-\sin12\alpha}{2\cos12\alpha\cdot\cos\alpha-\cos12\alpha}=\\
=\dfrac{\sin12\alpha\cancel{(2\cos\alpha-1)}}{\cos12\alpha\cancel{(2\cos\alpha-1)}}=
\dfrac{\sin12\alpha}{\cos12\alpha}=\tan12\alpha=\dfrac{1}{\cot12\alpha}=\dfrac{1}{-\dfrac{1}{11}}=-11.
\end{multline*}
\textbf{42.} Обчислити:
$$
\dfrac{\sin11\alpha+\cos12\alpha-\sin13\alpha}{\cos13\alpha+\sin12\alpha-\cos11\alpha}, \; \mbox{якщо} \; \tan12\alpha=\dfrac{1}{91};
$$
В чисельнику згрупуємо разом $\sin11\alpha-\sin13\alpha$, а в знаменнику -- $\cos13\alpha-\cos11\alpha$:
\begin{multline*}
\dfrac{\sin11\alpha+\cos12\alpha-\sin13\alpha}{\cos13\alpha+\sin12\alpha-\cos11\alpha}=
\dfrac{-2\sin\alpha\cdot\cos12\alpha+\cos12\alpha}{-2\sin12\alpha\cdot\sin\alpha+\sin12\alpha}=\\
=\dfrac{\cos12\alpha\cancel{(1-2\sin\alpha)}}{\sin12\alpha\cancel{(1-2\sin\alpha)}}=
\dfrac{\cos12\alpha}{\sin12\alpha}=\cot12\alpha=\dfrac{1}{\tan12\alpha}=\dfrac{1}{\dfrac{1}{91}}=91.
\end{multline*}
\textbf{43.} Обчислити:
$$
\dfrac{\cos3\alpha-\cos9\alpha}{\sin9\alpha-\sin3\alpha}, \; \mbox{якщо} \; \cot6\alpha=\dfrac{1}{12};
$$
\begin{multline*}
\dfrac{\cos3\alpha-\cos9\alpha}{\sin9\alpha-\sin3\alpha}=
\dfrac{\cancel{2}\sin6\alpha\cdot\cancel{\sin3\alpha}}{\cancel{2}\cancel{\sin3\alpha}\cdot\cos6\alpha}=
\dfrac{\sin6\alpha}{\cos6\alpha}=\tan6\alpha=\dfrac{1}{\cot6\alpha}=\\
=\dfrac{1}{\dfrac{1}{12}}=12.
\end{multline*}
\textbf{44.} Обчислити:
$$
\dfrac{\sin9\alpha+\sin3\alpha}{\cos3\alpha+\cos9\alpha}, \; \mbox{якщо} \; \cot6\alpha=-\dfrac{1}{45};
$$
\begin{multline*}
\dfrac{\sin9\alpha+\sin3\alpha}{\cos3\alpha+\cos9\alpha}=
\dfrac{\cancel{2}\sin6\alpha\cdot\cancel{\cos3\alpha}}{\cancel{2}\cos6\alpha\cdot\cancel{\cos3\alpha}}=
\dfrac{\sin6\alpha}{\cos6\alpha}=
\tan6\alpha=\dfrac{1}{\cot6\alpha}=\\
=\dfrac{1}{-\dfrac{1}{45}}=-45.
\end{multline*}
\textbf{45.} Обчислити:
$$
\dfrac{\sin5\alpha+\sin10\alpha+\sin15\alpha}{\cos5\alpha+\cos10\alpha+\cos15\alpha}, \; \mbox{якщо} \; \cot10\alpha=0,04;
$$
\begin{multline*}
\dfrac{\sin5\alpha+\sin10\alpha+\sin15\alpha}{\cos5\alpha+\cos10\alpha+\cos15\alpha}=
\dfrac{2\sin10\alpha\cdot\cos5\alpha+\sin10\alpha}{2\cos10\alpha\cdot\cos5\alpha+\cos10\alpha}=\\
=\dfrac{\sin10\alpha\cancel{(2\cos5\alpha+1)}}{\cos10\alpha\cancel{(2\cos5\alpha+1)}}=
\dfrac{\sin10\alpha}{\cos10\alpha}=\tan10\alpha=\dfrac{1}{\cot\alpha}=\dfrac{1}{0,04}=25.
\end{multline*}
\textbf{46.} Обчислити:
$$
\dfrac{\sin5\alpha-2\cos15\alpha-\sin25\alpha}{\cos5\alpha+2\sin15\alpha-\cos25\alpha}, \; \mbox{якщо} \; \tan15\alpha=-\dfrac{1}{2};
$$
В чисельнику згрупуємо разом $\sin5\alpha-\sin25\alpha$, а в знаменнику -- $\cos5\alpha-\cos25\alpha$:
\begin{multline*}
\dfrac{\sin5\alpha-2\cos15\alpha-\sin25\alpha}{\cos5\alpha+2\sin15\alpha-\cos25\alpha}=
\dfrac{-2\sin10\alpha\cdot\cos15\alpha-2\cos15\alpha}{2\sin15\alpha\cdot\sin10\alpha+2\sin15\alpha}=\\
=\dfrac{-2\cos15\alpha\cancel{(\sin10\alpha+1)}}{2\sin15\alpha\cancel{(\sin10\alpha+1)}}=
-\dfrac{\cos15\alpha}{\sin15\alpha}=-\cot15\alpha=-\dfrac{1}{\tan15\alpha}=\\
=-\dfrac{1}{-\dfrac{1}{12}}=12.
\end{multline*}
\textbf{47.} Обчислити:
$$
5 \cos \alpha, \;\;\; \mbox{якщо} \;\;\; \sin \alpha = \dfrac{4}{5}, \;\;\; 0 < \alpha < \dfrac{\pi}{2}.
$$
Ми можемо записати:
$$
\cos \alpha = \pm \sqrt{1 - \sin^2 \alpha}.
$$
Оскільки $\cos \alpha$, на заданому проміжку є додатним, то квадратний корінь беремо зі додатним  знаком. Отже,
\begin{flalign*}
5 \cos \alpha = 5 \sqrt{1 - \sin^2 \alpha}.
\end{flalign*}
Після підстановки отримаємо:
$$
5 \cdot \sqrt{1 - \dfrac{16}{25}} = 5 \cdot \sqrt{\dfrac{9}{25}} = 5 \cdot \dfrac{3}{5} = 3.
$$
\textbf{48.} Обчислити:
$$
\sqrt{21}\cdot\sin\alpha, \; \mbox{якщо} \cos\alpha=\dfrac{2}{5}, \; 0<\alpha<\dfrac{\pi}{2};
$$
Ми можемо виразити $\sin\alpha$ через $\cos\alpha$, тобто:
$$
\sin\alpha=\pm\sqrt{1-\cos^{2}\alpha}=\pm\sqrt{1-\dfrac{4}{25}}=\pm\dfrac{\sqrt{21}}{5};
$$
Нам дано, що кут $\alpha$ лежить у першій чверті, а $\sin\alpha$ у цій чверті додатний, тому ми беремо зі знаком плюс. Отже, підставимо знайдене значення $\sin\alpha$ у початковий вираз:
$$
\sqrt{21}\cdot\dfrac{\sqrt{21}}{5}=\dfrac{21}{5}=4,2.
$$
\textbf{49.} Обчислити:
$$
18\sqrt{77}\cdot\sin\alpha, \; \mbox{якщо} \cos\alpha=\dfrac{2}{9}, \; 0<\alpha<\dfrac{\pi}{2};
$$
Ми можемо виразити $\sin\alpha$ через $\cos\alpha$, тобто:
$$
\sin\alpha=\pm\sqrt{1-\cos^{2}\alpha}=\pm\sqrt{1-\dfrac{4}{81}}=\pm\dfrac{\sqrt{77}}{9};
$$
Нам дано, що кут $\alpha$ лежить у першій чверті, а $\sin\alpha$ у цій чверті додатний, тому ми беремо зі знаком плюс. Отже, підставимо знайдене значення $\sin\alpha$ у початковий вираз:
$$
18\sqrt{77}\cdot\dfrac{\sqrt{77}}{9}=2\cdot77=154.
$$
\textbf{50.} Обчислити:
$$
10\sqrt{119}\cdot\tan\alpha, \; \mbox{якщо} \cos\alpha=\dfrac{5}{12}, \; 0<\alpha<\dfrac{\pi}{2};
$$
$$
10\sqrt{119}\cdot\tan\alpha=10\sqrt{119}\cdot\dfrac{\sin\alpha}{\cos\alpha};
$$
$\cos\alpha$ нам дано, а $\sin\alpha$ ми можемо виразити через $\cos\alpha$, тобто:
$$
\sin\alpha=\pm\sqrt{1-\cos^{2}\alpha}=\pm\sqrt{1-\dfrac{25}{144}}=\pm\dfrac{\sqrt{119}}{12};
$$
Нам дано, що кут $\alpha$ лежить у першій чверті, а $\sin\alpha$ у цій чверті додатний, тому ми беремо зі знаком плюс. Отже, підставимо знайдене значення $\sin\alpha$ у початковий вираз:
$$
10\sqrt{119}\cdot\dfrac{\dfrac{\sqrt{119}}{12}}{\dfrac{5}{12}}=10\sqrt{119}\cdot\dfrac{\sqrt{119}}{12}\cdot\dfrac{12}{5}=2\cdot119=238.
$$
\textbf{51.} Обчислити:
$$
8\sqrt{33}\cdot\cot\alpha, \; \mbox{якщо} \sin\alpha=\dfrac{4}{7}, \; 0<\alpha<\dfrac{\pi}{2};
$$
$$
8\sqrt{33}\cdot\cot\alpha=8\sqrt{33}\cdot\dfrac{\cos\alpha}{\sin\alpha};
$$
$\sin\alpha$ нам дано, а $\cos\alpha$ ми можемо виразити через $\sin\alpha$, тобто:
$$
\cos\alpha=\pm\sqrt{1-\sin^{2}\alpha}=\pm\sqrt{1-\dfrac{16}{49}}=\pm\dfrac{\sqrt{33}}{7};
$$
Нам дано, що кут $\alpha$ лежить у першій чверті, а $\cos\alpha$ у цій чверті додатний, тому ми беремо зі знаком плюс. Отже, підставимо знайдене значення $\cos\alpha$ у початковий вираз:
$$
8\sqrt{33}\cdot\dfrac{\dfrac{\sqrt{33}}{7}}{\dfrac{4}{7}}=
8\sqrt{33}\cdot\dfrac{\sqrt{33}}{\cancel{7}}\cdot\dfrac{\cancel{7}}{4}=
2\cdot33=66.
$$
\textbf{52.} Обчислити:
$$
26\cos\alpha, \; \mbox{якщо} \sin\alpha=-\dfrac{5}{13}, \; \pi<\alpha<\dfrac{3\pi}{2};
$$
Ми можемо виразити $\cos\alpha$ через $\sin\alpha$, тобто:
$$
\cos\alpha=\pm\sqrt{1-\sin^{2}\alpha}=\pm\sqrt{1-\dfrac{25}{169}}=\pm\dfrac{12}{13};
$$
Нам дано, що кут $\alpha$ лежить у третій чверті, а $\cos\alpha$ у цій чверті від'ємний, тому ми беремо зі знаком мінус. Отже, підставимо знайдене значення $\cos\alpha$ у початковий вираз:
$$
26\cdot\left(-\dfrac{12}{13}\right)=-2\cdot12=-24
$$
\textbf{53.} Обчислити:
$$
15\cos\alpha, \; \mbox{якщо} \sin\alpha=\dfrac{\sqrt{3}}{2}, \; \pi<\alpha<\dfrac{3\pi}{2};
$$
Ми можемо виразити $\cos\alpha$ через $\sin\alpha$, тобто:
$$
\cos\alpha=\pm\sqrt{1-\sin^{2}\alpha}=\pm\sqrt{1-\dfrac{3}{4}}=\pm\dfrac{1}{2};
$$
Нам дано, що кут $\alpha$ лежить у третій чверті, а $\cos\alpha$ у цій чверті від'ємний, тому ми беремо зі знаком мінус. Отже, підставимо знайдене значення $\cos\alpha$ у початковий вираз:
$$
15\cdot\left(-\dfrac{1}{2}\right)=-7,5.
$$
\textbf{54.} Обчислити:
$$
4\cos\alpha, \; \mbox{якщо} \sin\alpha=\dfrac{\sqrt{7}}{4}, \; \dfrac{\pi}{2}<\alpha<\pi;
$$
Ми можемо виразити $\cos\alpha$ через $\sin\alpha$, тобто:
$$
\cos\alpha=\pm\sqrt{1-\sin^{2}\alpha}=\pm\sqrt{1-\dfrac{7}{16}}=\pm\dfrac{3}{4};
$$
Нам дано, що кут $\alpha$ лежить у другій чверті, а $\cos\alpha$ у цій чверті від'ємний, тому ми беремо зі знаком мінус. Отже, підставимо знайдене значення $\cos\alpha$ у початковий вираз:
$$
4\cdot\left(-\dfrac{3}{4}\right)=-3.
$$
\textbf{55.} Обчислити:
$$
\sqrt{15}\cdot\tan\alpha, \; \mbox{якщо} \cos\alpha=-\dfrac{1}{4}, \; \pi<\alpha<\dfrac{3\pi}{2};
$$
$$
\sqrt{15}\cdot\tan\alpha=\sqrt{15}\cdot\dfrac{\sin\alpha}{\cos\alpha};
$$
$\cos\alpha$ нам дано, а $\sin\alpha$ ми можемо виразити через $\cos\alpha$, тобто:
$$
\sin\alpha=\pm\sqrt{1-\cos^{2}\alpha}=\pm\sqrt{1-\dfrac{1}{16}}=\pm\dfrac{\sqrt{15}}{4};
$$
Нам дано, що кут $\alpha$ лежить у третій чверті, а $\sin\alpha$ у цій чверті від'ємний, тому ми беремо зі знаком мінус. Отже, підставимо знайдене значення $\sin\alpha$ у початковий вираз:
$$
\sqrt{15}\cdot\dfrac{-\dfrac{\sqrt{15}}{4}}{-\dfrac{1}{4}}=
\sqrt{15}\cdot\left(-\dfrac{\sqrt{15}}{\cancel{4}}\right)\cdot\left(-\dfrac{\cancel{4}}{1}\right)=15.
$$
%%%%%%%%%%%%%%%%%%%%%%%%%%%%%%%%%%%%%%%%%%%%%%%%%%%%%%%%%%%%%%%%%%%%%%%%%%%%%
%%%%%%%%%%%%%%%%%%%%%%%%%%%%%%%%%%%%%%%%%%%%%%%%%%%%%%%%%%%%%%%%%%%%%%%%%%%%%
%%%%%%%%%%%%%%%%%%%%%%%%%%%%%%%%%%%%%%%%%%%%%%%%%%%%%%%%%%%%%%%%%%%%%%%%%%%%%
%%%%%%%%%%%%%%%%%%%%%%%%%%%%%%%%%%%%%%%%%%%%%%%%%%%%%%%%%%%%%%%%%%%%%%%%%%%%%
%%%%%%%%%%%%%%%%%%%%%%%%%%%%%%%%%%%%%%%%%%%%%%%%%%%%%%%%%%%%%%%%%%%%%%%%%%%%%
%%%%%%%%%%%%%%%%%%%%%%%%%%%%%%%%%%%%%%%%%%%%%%%%%%%%%%%%%%%%%%%%%%%%%%%%%%%%%
\section*{Група 2}
\textbf{1.} Обчислити:
$$
2 \cos (4\alpha) + \sin (4\alpha) \cdot \tan (2\alpha).
$$
Скористаємося формулами:
$$
\cos 4\alpha = \cos^2 2\alpha - \sin^2 2\alpha, \;\;\;
\sin 4\alpha = 2 \sin 2\alpha \cdot \cos 2\alpha.
$$
Отримаємо:
\begin{flalign*}
&2 \cos 4\alpha + \sin 4\alpha \cdot \tan 2\alpha =
2 \cos 4\alpha + 4 \sin 2\alpha \cdot \cancel{\cos 2\alpha} \cdot \dfrac{\sin 2\alpha}{\cancel{\cos 2\alpha}} =\\
&= 2\cos^2 2\alpha - 2\sin^2 2\alpha + 4\sin^2 2\alpha =
2\cos^2 2\alpha + 2\sin^2 2\alpha =\\
&= 2 \left(\cos^2 2\alpha + \sin^2 2\alpha\right) = 2.
\end{flalign*}
\textbf{2.} Обчислити:
$$
\dfrac{1 + \sin 2\alpha}{1 + \cos 2\alpha} \cdot \dfrac{2}{(1 + \tan \alpha)^2}.
$$
Скористаємося формулою:
$$
\tan \alpha = \dfrac{\sin \alpha}{\cos \alpha}.
$$
Отримаємо:
\begin{flalign*}
&\dfrac{1 + \sin 2\alpha}{1 + \cos 2\alpha} \cdot \dfrac{2}{(1 + \tan \alpha)^2} =
\dfrac{1 + \sin 2\alpha}{1 + \cos 2\alpha} \cdot \dfrac{2}{\left(1 + \dfrac{\sin \alpha}{\cos \alpha}\right)^2} =\\
&= \dfrac{1 + \sin 2\alpha}{1 + \cos 2\alpha} \cdot \dfrac{2}{\dfrac{(\cos \alpha + \sin \alpha)^2}{\cos^2 \alpha}} =\\
&= \dfrac{1 + \sin 2\alpha}{1 + \cos 2\alpha} \cdot \dfrac{2 \cdot \cos^2 \alpha}{\cos^2 \alpha + 2\sin \alpha \cdot \cos \alpha + \sin^2 \alpha} =\\
&= \dfrac{\cancel{1 + \sin 2\alpha}}{1 + \cos 2\alpha} \cdot \dfrac{2 \cos ^2 \alpha}{\cancel{1 + \sin 2\alpha}} =
\dfrac{2 \cos^2 \alpha}{1 + \cos 2\alpha} =\\
&= \dfrac{2 \cos^2 \alpha}{\cos^2 \alpha + \cancel{\sin^2 \alpha} + \cos^2 \alpha - \cancel{\sin^2 \alpha}} =
\dfrac{2 \cancel{\cos^2 \alpha}}{\cancel{\cos^2 \alpha}} = 2.
\end{flalign*}
\textbf{6.} Обчислити:
$$
3 \cdot \dfrac{\sin^4 \alpha + \cos^4 \alpha - 1}{\cos^6 \alpha + \sin^6 \alpha - 1}.
$$
В знаменнику розпишемо як суму кубів. Отримаємо:
\begin{flalign*}
&3 \cdot \dfrac{\sin^4 \alpha + \cos^4 \alpha - 1}{\cos^6 \alpha + \sin^6 \alpha - 1} =\\
&= 3 \cdot \dfrac{\sin^4 \alpha + \cos^4 \alpha - 1}{\left(\cos^2 \alpha + \sin^2 \alpha\right)\left(\cos^4 \alpha - 2\cos^2 \alpha \cdot \sin^2 \alpha + \sin^4 \alpha\right) - 1} =\\
&= 3 \cdot \dfrac{\sin^4 \alpha + \cos^4 \alpha - 1}{\cos^4 \alpha - 2\cos^2 \alpha \cdot \sin^2 \alpha + \sin^4 \alpha - 1}=\\
&= 3 \cdot \dfrac{\left(\sin^4 \alpha + 2\sin^2 \alpha \cdot \cos^2 \alpha + \cos^4 \alpha\right) - 2 \sin^2 \alpha \cdot \cos^2 \alpha - 1}{\left(\cos^4 \alpha + 2\sin^2 \alpha \cdot \cos^2 \alpha + \sin^4 \alpha\right) - 3\sin^2 \alpha \cdot \cos^2 \alpha - 1} =\\
&= 3 \cdot \dfrac{\left(\cos^2 \alpha + \sin^2 \alpha\right)^2 - 2\sin^2 \alpha \cdot \cos^2 \alpha - 1}{\left(\cos^2 \alpha + \sin^2 \alpha\right)^2 - 3\sin^2 \alpha \cdot \cos^2 \alpha - 1} =\\
&= 3 \cdot \dfrac{\cancel{1} - 2\sin^2 \alpha \cdot \cos^2 \alpha - \cancel{1}}{\cancel{1} - 3\sin^2 \alpha \cdot \cos^2 \alpha - \cancel{1}} =
3 \cdot \dfrac{-2 \cdot \cancel{\sin^2 \alpha \cdot \cos^2 \alpha}}{-3 \cdot \cancel{\sin^2 \alpha \cdot \cos^2 \alpha}} =
3 \cdot \dfrac{-2}{-3} = 2.
\end{flalign*}
\textbf{10.} Обчислити:
\begin{flalign*}
&42 \cdot \dfrac{\cos^8 6\alpha - \sin^8 6\alpha - \cos 12\alpha}{\cos^6 6\alpha - \sin^6 6\alpha - \cos 12\alpha} =\\
&= 42 \cdot \dfrac{\left(\cos^4 6\alpha - \sin^4 6\alpha\right)\left(\cos^4 6\alpha + \sin^4 6\alpha\right) - \cos 12\alpha}{\left(\cos^2 6\alpha - \sin^2 6\alpha\right)\left(\cos^4 6\alpha + \cos^2 6\alpha \cdot \sin^2 6\alpha + \sin^4 6\alpha\right) - \cos 12\alpha} =\\
&= 42 \cdot \dfrac{\left(\cos^2 6\alpha - \sin^2 6\alpha\right)\left(\cos^2 6\alpha + \sin^2 6\alpha\right)\left(\cos^4 6\alpha + \sin^4 6\alpha\right) - \cos 12\alpha}{\left(\cos^2 6\alpha - \sin^2 6\alpha\right)\left(\cos^4 6\alpha + \cos^2 6\alpha \cdot \sin^2 6\alpha + \sin^4 6\alpha\right) - \cos 12\alpha} =\\
&= 42 \cdot \dfrac{\cos 12\alpha \cdot 1 \cdot \left(\cos^4 6\alpha + \sin^4 6\alpha\right) - \cos 12\alpha}{\cos 12\alpha \cdot \left(\cos^4 6\alpha + \cos^2 6\alpha \cdot \sin^2 6\alpha + \sin^4 6\alpha\right) - \cos 12\alpha}.
\end{flalign*}
Для чисельника виконаємо такі перетворення:
\begin{flalign*}
&\cos^4 6\alpha + \sin^4 6\alpha =\\
&\left(\cos^4 6\alpha + 2\cos^2 6\alpha \cdot \sin^2 6\alpha + \sin^4 6\alpha\right) - 2\cos^2 6\alpha \cdot \sin^2 6\alpha =\\
&= \left(\cos^2 6\alpha + \sin^2 6\alpha\right)^2 - 2\cos^2 6\alpha \cdot \sin^2 6\alpha = 1 - 2\cos^2 6\alpha \cdot \sin^2 6\alpha.
\end{flalign*}
Аналогічно для знаменника:
\begin{flalign*}
&\cos^4 6\alpha + \cos^2 6\alpha \cdot \sin^2 6\alpha + \sin^4 6\alpha =\\
&= \left(\cos^4 6\alpha + 2\cos^2 6\alpha \cdot \sin^2 6\alpha + \sin^4 6\alpha\right) - \cos^2 6\alpha \cdot \sin^2 6\alpha =\\
&= \left(\cos^2 6\alpha + \sin^2 6\alpha\right)^2 - \cos^2 6\alpha \cdot \sin^2 6\alpha =
1 - \cos^2 6\alpha \cdot \sin^2 6\alpha.
\end{flalign*}
Продовжимо з урахуванням цих перетворень:
\begin{flalign*}
&42 \cdot \dfrac{\cos 12\alpha \cdot \left(1 - 2\cos^2 6\alpha \cdot \sin^2 6\alpha\right) - \cos 12\alpha}{\cos 12\alpha \cdot \left(1 - \cos^2 6\alpha \cdot \sin^2 6\alpha\right) - \cos 12\alpha} =\\
&= 42 \cdot \dfrac{\cancel{\cos 12\alpha} \cdot \left(\cancel{1} - 2\cos^2 6\alpha \cdot \sin^2 6\alpha - \cancel{1}\right)}{\cancel{\cos 12\alpha} \cdot \left(\cancel{1} - \cos^2 6\alpha \cdot \sin^2 6\alpha - \cancel{1}\right)} =\\
&= 42 \cdot \dfrac{-2 \cdot \cancel{\cos^2 6\alpha \cdot \sin^2 6\alpha} }{- \cancel{\cos^2 6\alpha \cdot \sin^2 6\alpha}} =
42 \cdot \dfrac{-2}{-1} = 84.
\end{flalign*}
\textbf{13.} Обчислити:
$$
\dfrac{\cos 3\alpha - \sin 3\alpha}{\cos \alpha + \sin \alpha}, \;\;\; \mbox{якщо} \;\;\; \sin\left(\dfrac{\pi}{4} - \alpha\right) = 0,1.
$$
\begin{flalign*}
&\dfrac{\cos 3\alpha - \sin 3\alpha}{\cos \alpha + \sin \alpha} = \dfrac{4 \cos^3 \alpha - 3\cos \alpha -3\sin \alpha + 4\sin^3 \alpha}{\cos \alpha + \sin \alpha} =\\
&= \dfrac{4\left(\cos^3 \alpha + \sin^3 \alpha\right) - 3\left(\cos \alpha + \sin \alpha\right)}{\cos \alpha + \sin \alpha} =\\
&= \dfrac{4\left(\cos \alpha + \sin \alpha\right)\left(\cos^2 \alpha - \cos \alpha \cdot \sin \alpha + \sin^2 \alpha\right) - 3 \left(\cos \alpha + \sin \alpha\right)}{\cos \alpha + \sin \alpha} =\\
&= \dfrac{\left(\cos \alpha + \sin \alpha\right)\left[\;4\;(1 - \sin \alpha \cos \alpha) - 3\;\right]}{\cos \alpha + \sin \alpha} =\\
&= 4 - 4 \sin \alpha \cos \alpha - 3 = 1 - 2\sin 2\alpha =
1 - 2\cos\left(\dfrac{\pi}{2} - 2\alpha\right) =\\
&= 1 - 2\cos2\left(\dfrac{\pi}{4} - \alpha\right)=
1 - 2\left[1 - 2\sin^2\left(\dfrac{\pi}{4} - \alpha\right)\right] =\\
&= 1 - 2 + 4\left[\sin\left(\dfrac{\pi}{4} - \alpha\right)\right]^2 =
-1 + 4 \cdot 0,01 = -0,96.
\end{flalign*}
\textbf{21.} Обчислити:
\begin{flalign*}
&\dfrac{\sin^3 5\alpha + \sin15\alpha}{5\sin5\alpha} + \dfrac{\cos^3 5\alpha - \cos15\alpha}{5\cos5\alpha} =\\
&= \dfrac{\sin^3 5\alpha + 3\sin5\alpha - 4\sin^3 5\alpha}{5\sin5\alpha} + \dfrac{\cos^3 5\alpha - 4\cos^3 5\alpha + 3\cos5\alpha}{5\cos5\alpha} =\\
&= \dfrac{3\cancel{\sin5\alpha}\left(1 - \sin^2 5\alpha\right)}{5\cancel{\sin5\alpha}} + \dfrac{3\cancel{\cos5\alpha}\left(1 - \cos^2 5\alpha\right)}{5\cancel{\cos5\alpha}} =\\
&= \dfrac{3}{5}\cdot\sin^2 5\alpha + \dfrac{3}{5}\cdot\cos^2 5\alpha =
\dfrac{3}{5}\cdot\left(\sin^2 5\alpha + \cos^2 5\alpha\right) = \dfrac{3}{5} = 0,6.
\end{flalign*}
\textbf{24.} Обчислити:
$$
\dfrac{2-\sqrt{3}\cos\alpha-\sin\alpha}{\sqrt{3}\sin\alpha-\cos\alpha}, \;\;\; \mbox{якщо} \;\;\; \tan\left(\dfrac{\alpha}{2}-\dfrac{\pi}{12}\right)=2.
$$
Поділимо чисельник і знаменник на $2$. В кінці скористаємося формулою:
$$
\tan\dfrac{\alpha}{2}=\dfrac{1-\cos\alpha}{\sin\alpha}.
$$
Отримаємо:
\begin{flalign*}
&\dfrac{2-\sqrt{3}\cos\alpha-\sin\alpha}{\sqrt{3}\sin\alpha-\cos\alpha} =
\dfrac{1-\dfrac{\sqrt{3}}{2}\cos\alpha-\dfrac{1}{2}\sin\alpha}{\dfrac{\sqrt{3}}{2}\sin\alpha-\dfrac{1}{2}\cos\alpha} =\\
&= \dfrac{1-\left(\cos\dfrac{\pi}{6}\cos\alpha+\sin\dfrac{\pi}{6}\cos\alpha\right)}{\cos\dfrac{\pi}{6}\sin\alpha-\sin\dfrac{\pi}{6}\sin\alpha} =\\
&= \dfrac{1-\cos\left(\dfrac{\pi}{6}-\alpha\right)}{-\sin\left(\dfrac{\pi}{6}-\alpha\right)} =
-\dfrac{1-\cos\left(\dfrac{\pi}{6}-\alpha\right)}{\sin\left(\dfrac{\pi}{6}-\alpha\right)} =\\
&= -\tan\left(\dfrac{\pi}{12}-\dfrac{\alpha}{2}\right) =
\tan\left(\dfrac{\alpha}{2}-\dfrac{\pi}{12}\right) = 2.
\end{flalign*}
\textbf{32.} Обчислити:
$$
\sqrt{3}\left(\sin\dfrac{10\pi}{3}+\cos\dfrac{41\pi}{6}\right).
$$
Візьмемо до уваги, що:
\begin{flalign*}
\sin\dfrac{10\pi}{3}=\sin\left(2\pi+\dfrac{4\pi}{3}\right)=
\sin\dfrac{4\pi}{3}=\sin\left(\pi+\dfrac{\pi}{3}\right)=
-\sin\dfrac{\pi}{3}.
\end{flalign*}
Також:
\begin{flalign*}
\cos\dfrac{41\pi}{6}=\cos\left(6\pi+\dfrac{5\pi}{6}\right)=
\cos\dfrac{5\pi}{6}=\cos\left(\pi-\dfrac{\pi}{6}\right)=
-\cos\dfrac{\pi}{6}.
\end{flalign*}
Врахувавши ці перетворення, отримаємо:
\begin{flalign*}
\sqrt{3}\left(-\sin\dfrac{\pi}{3}-\cos\dfrac{\pi}{6}\right)=
\sqrt{3}\left(-\dfrac{\sqrt{3}}{2}-\dfrac{\sqrt{3}}{2}\right)=
\sqrt{3}\cdot\left(-\sqrt{3}\right)=-3.
\end{flalign*}
\textbf{40.} Обчислити:
$$
0,5\cdot\cos\alpha, \;\;\; \mbox{якщо} \;\;\; \tan\alpha=\sqrt{3}, \;\;\; \pi<\alpha<\dfrac{3\pi}{2}.
$$
Будемо працювати з $\tan\alpha$. Отже:
\begin{gather*}
\tan\alpha=\dfrac{\sin\alpha}{\cos\alpha}=\sqrt{3};\\
\tan^2\alpha=\dfrac{\sin^2\alpha}{\cos^2\alpha}=3;\\
\dfrac{1-\cos^2\alpha}{\cos^2\alpha}=3;\\
\dfrac{1}{\cos^2\alpha}-1=3;\\
\cos^2\alpha=\dfrac{1}{4};\\
\cos\alpha=\pm\dfrac{1}{2}.
\end{gather*}
Оскільки $\alpha$ міститься в третій чверті, а $\cos\alpha$ в цій чверті від'ємний, то потрібно
брати зі знаком мінус. Таким чином:
$$
0,5\cdot(-\dfrac{1}{2})=-\dfrac{1}{4}=-0,25.
$$
\textbf{51.} Обчислити:
$$
4\sqrt{5}\cos\dfrac{\alpha}{2}, \;\;\; \mbox{якщо} \;\;\; \sin\alpha=\dfrac{4}{5} \;\;\; \dfrac{\pi}{2}<\alpha<\pi.
$$
Скористаємося формулою:
$$
\cos\dfrac{\alpha}{2}=\sqrt{\dfrac{1+\cos\alpha}{2}}.
$$
Отримаємо:
$$
4\sqrt{5}\cos\dfrac{\alpha}{2}=4\sqrt{5}\;\sqrt{\dfrac{1+\cos\alpha}{2}}.
$$
Нам дано $\sin\alpha$, а нам потрібно $\cos\alpha$. Скористаємося формулою:
$$
\cos\alpha=\pm\sqrt{1-\cos^2\alpha}=\pm\sqrt{1-\dfrac{16}{25}}=\pm\sqrt{\dfrac{9}{25}}=\pm\dfrac{3}{5}.
$$
Оскільки $\cos\alpha$ у другій чверті є від'ємний, то беремо знак мінус. В кінцевому результаті отримаємо:
$$
4\sqrt{5}\;\sqrt{\dfrac{1+\cos\alpha}{2}}=
4\sqrt{5}\;\sqrt{\dfrac{1-\frac{2}{3}}{2}}=
4\sqrt{5}\;\sqrt{\dfrac{\frac{2}{5}}{2}}=
4\cancel{\sqrt{5}}\;\dfrac{1}{\cancel{\sqrt{5}}}=4.
$$
\textbf{53.} Обчислити:
$$
3\sqrt{2}\cdot\sin(\arccos\dfrac{1}{3}).
$$
Використаємо формулу:
$$
\sin\alpha=\sqrt{1-\cos^2\alpha}.
$$
Отримаємо:
\begin{flalign*}
&3\sqrt{2}\cdot\sin(\arccos\dfrac{1}{3})=
3\sqrt{2}\cdot\sqrt{1-\cos^2(\arccos\dfrac{1}{3})}=
3\sqrt{2}\cdot\sqrt{1-\dfrac{1}{9}}=\\
&= 3\sqrt{2}\cdot\dfrac{\sqrt{8}}{3}=
3\sqrt{2}\cdot\dfrac{2\sqrt{2}}{3}=4.
\end{flalign*}
\textbf{61.} Обчислити:
$$
\tan\alpha-\cot\alpha, \;\;\; \mbox{якщо} \;\;\; \tan\alpha+\cot\alpha=\sqrt{8}, \;\;\; 0<\alpha<\dfrac{\pi}{4}.
$$
Спочатку розберемо вираз:
\begin{gather*}
\tan\alpha+\cot\alpha=
\dfrac{\sin\alpha}{\cos\alpha}+\dfrac{\cos\alpha}{\sin\alpha}=\\
\dfrac{\sin^2\alpha+\cos^2\alpha}{\cos\alpha\cdot\sin\alpha}=
\dfrac{1}{\dfrac{1}{2}\cdot\sin2\alpha}=
\dfrac{2}{\sin2\alpha};\\
\end{gather*}
Отже:
$$
\dfrac{2}{\sin2\alpha}=\sqrt{8}\Leftrightarrow\sin2\alpha=\dfrac{1}{\sqrt{2}};
$$
Тепер розберемо основний вираз:
\begin{gather*}
\tan\alpha-\cot\alpha=
\dfrac{\sin\alpha}{\cos\alpha}-\dfrac{\cos\alpha}{\sin\alpha}=
\dfrac{\sin^2\alpha-\cos^2\alpha}{\cos\alpha\cdot\sin\alpha}=
\dfrac{-(\cos^2\alpha-\sin^2\alpha)}{\dfrac{1}{2}\cdot\sin2\alpha}=\\
=-2\dfrac{\cos2\alpha}{\sin2\alpha}=
-2\dfrac{\pm\sqrt{1-\sin^2 2\alpha}}{\sin2\alpha}.
\end{gather*}
Так, як $0<\alpha<\dfrac{\pi}{4}$, або $0<2\alpha<\dfrac{\pi}{2}$ і $\cos$ на цьому проміжку додатний, то корінь квадратний потрібно брати зі знаком плюс, отже:
$$
\tan\alpha-\cot\alpha=
-2\dfrac{\sqrt{1-\sin^2 2\alpha}}{\sin2\alpha}.
$$
Вище ми обрахували, що $\sin2\alpha=\dfrac{1}{\sqrt{2}}$. Підставимо цей вираз в основний і отримаємо:
$$
\tan\alpha-\cot\alpha=
-2\dfrac{\sqrt{1-\sin^2 2\alpha}}{\sin2\alpha}=
-2\cdot\dfrac{\sqrt{1-\dfrac{1}{2}}}{\dfrac{1}{\sqrt{2}}}=-2\cdot\dfrac{1}{\sqrt{2}}\cdot\sqrt{2}=2.
$$
\textbf{66.} Обчислити:
$$
\sin\alpha+\cos\alpha, \;\;\; \mbox{якщо} \;\;\; \sin2\alpha=-0,84, \;\;\; \dfrac{3\pi}{4}<\alpha<\pi.
$$
Спочатку працюємо з основним виразом:
$$
\sin\alpha+\cos\alpha=\sqrt{2}\cdot\cos\left(\dfrac{\pi}{4}-\alpha\right).
$$
Тепер перетворимо заміну:
$$
\sin2\alpha=\cos\left(\dfrac{\pi}{2}-2\alpha\right)=\cos2\left(\dfrac{\pi}{4}-\alpha\right)=
2\cdot\cos^2\left(\dfrac{\pi}{4}-\alpha\right)-1;
$$
Тобто:
$$
2\cdot\cos^2\left(\dfrac{\pi}{4}-\alpha\right)-1=-0,84;
\Leftrightarrow
\cos\left(\alpha-\dfrac{\pi}{4}\right)=\pm\sqrt{0,08};
$$
Для того, щоб визначити який знак потрібно брати, чи плюс, чи мінус, ми повинні дослідити дану нам умову $\dfrac{3\pi}{4}<\alpha<\pi$. Але в даному випадку ми маємо просто $\alpha$, а нам потрібно щоб було $\dfrac{\pi}{4}-\alpha$. Отже, ми проведемо такі нескладні перетворення:
\begin{gather*}
\dfrac{3\pi}{4}<\alpha<\pi;\\
-\pi<-\alpha<-\dfrac{3\pi}{4};\\
-\pi+\dfrac{\pi}{4}<-\alpha+\dfrac{\pi}{4}<-\dfrac{3\pi}{4}+\dfrac{\pi}{4};\\
-\dfrac{3\pi}{4}<\dfrac{\pi}{4}-\alpha<-\dfrac{\pi}{2};
\end{gather*}
Звідси видно, що $\cos\left(\dfrac{\pi}{4}-\alpha\right)$ знаходиться в другій чверті, а там косинус є від'ємний. Значить, потрібно брати знак місус.
$$
\sqrt{2}\cdot\left(-\sqrt{0,08}\right)=-\sqrt{0,16}=-0,4.
$$
\section*{Група 3}
\textbf{1.} Обчислити:
$$
\left(\dfrac{\cos3\alpha}{\sin2\alpha}-\dfrac{\sin3\alpha}{\cos2\alpha}\right)\cdot\dfrac{1}{\cos\alpha+\cos9\alpha}, \;\;\; \mbox{якщо} \;\;\; \alpha=\dfrac{\pi}{48}.
$$
Зведемо в дужках до спільного знаменника:
\begin{multline*}
\left(\dfrac{\cos3\alpha}{\sin2\alpha}-\dfrac{\sin3\alpha}{\cos2\alpha}\right)\cdot\dfrac{1}{\cos\alpha+\cos9\alpha}=\\
=\dfrac{\cos3\alpha\cdot\cos2\alpha-\sin3\alpha\cdot\sin2\alpha}{\sin2\alpha\cdot\cos2\alpha}\cdot\dfrac{1}{2\cdot\cos5\alpha\cdot\cos4\alpha}=\\
=\dfrac{\cos(3\alpha+2\alpha)}{\dfrac{1}{2}\cdot\sin4\alpha}\cdot\dfrac{1}{2\cdot\cos5\alpha\cdot\cos4\alpha}=
\dfrac{\cancel{2}\cdot\cancel{\cos5\alpha}}{\sin4\alpha}\cdot\dfrac{1}{\cancel{2}\cdot\cancel{\cos5\alpha}\cdot\cos4\alpha}=\\
=\dfrac{1}{\sin4\alpha\cdot\cos4\alpha}=
\dfrac{2}{\sin8\alpha}=
\dfrac{2}{\sin\left(8\cdot\dfrac{\pi}{48}\right)}=
\dfrac{2}{\sin\dfrac{\pi}{6}}=\dfrac{2}{\dfrac{1}{2}}=4.
\end{multline*}
\textbf{2.} Обчислити:
$$
\dfrac{\cos^2\dfrac{\alpha}{2}\cdot(\cos\alpha-\cos3\alpha)}{\sin\alpha+2\sin2\alpha+\sin3\alpha}, \;\;\; \mbox{якщо} \;\;\; \alpha=\arcsin(0,1).
$$
Скористаємося формулою половинного кута:
$$
\cos\dfrac{\alpha}{2}=\sqrt{\dfrac{1+\cos\alpha}{2}}.
$$
Отримаємо:
\begin{multline*}
\dfrac{\cos^2\dfrac{\alpha}{2}\cdot(\cos\alpha-\cos3\alpha)}{\sin\alpha+2\cdot\sin2\alpha+\sin3\alpha}=
\dfrac{\dfrac{1+\cos\alpha}{\cancel{2}}\cdot(-\cancel{2})\cdot\sin2\alpha\cdot\sin(-\alpha)}{2\cdot\sin2\alpha+2\cdot\sin2\alpha\cdot\cos\alpha}=\\
=\dfrac{\cancel{(1+\cos\alpha)}\cdot\cancel{\sin2\alpha}\cdot\sin\alpha}{2\cdot\cancel{\sin2\alpha}\cdot\cancel{(1+\cos\alpha)}}=
\dfrac{\sin\alpha}{2}=\dfrac{\sin(\arcsin(0,1))}{2}=\dfrac{0,1}{2}=0,05.
\end{multline*}
\textbf{18.} Обчислити:
$$
\left(\dfrac{1}{\sin\alpha}-\dfrac{1}{\sin3\alpha}-\dfrac{2\sin2\alpha}{\cos3\alpha}\right)\cdot\dfrac{1}{\cos5\alpha}, \;\;\; \mbox{якщо} \;\;\; \alpha=\dfrac{1}{6}\cdot\arcsin(0,2).
$$
В дужках зведемо перші два доданки:
\begin{multline*}
\left(\dfrac{1}{\sin\alpha}-\dfrac{1}{\sin3\alpha}-\dfrac{2\sin2\alpha}{\cos3\alpha}\right)\cdot\dfrac{1}{\cos5\alpha}=\\
=\left(\dfrac{\sin3\alpha-\sin\alpha}{\sin\alpha\cdot\sin3\alpha}-\dfrac{2\cdot\sin2\alpha}{\cos3\alpha}\right)\cdot\dfrac{1}{\cos5\alpha}=\\
=\left(\dfrac{2\cdot\cancel{\sin\alpha}\cdot\cos2\alpha}{\cancel{\sin\alpha}\cdot\sin3\alpha}-\dfrac{2\cdot\sin2\alpha}{\cos3\alpha}\right)\cdot\dfrac{1}{\cos5\alpha}=\\
=\dfrac{2\cdot\cos2\alpha\cdot\cos3\alpha-2\cdot\sin2\alpha\cdot\sin3\alpha}{\sin3\alpha\cdot\cos3\alpha}\cdot\dfrac{1}{\cos5\alpha}=
\dfrac{2\cdot\cancel{\cos5\alpha}}{\sin3\alpha\cdot\cos3\alpha}\cdot\dfrac{1}{\cancel{\cos5\alpha}}=\\
=\dfrac{2}{\sin3\alpha\cdot\cos3\alpha}=
\dfrac{4}{\sin6\alpha}=
\dfrac{4}{\sin6\cdot\left(\dfrac{1}{6}\cdot\arcsin(0,2)\right)}=
\dfrac{4}{0,2}=20.
\end{multline*}
\textbf{21.} Обчислити:
$$
\tan(\alpha+15^{\circ})\cdot\cot165^{\circ}, \;\;\; \mbox{якщо} \;\;\; \sin75^{\circ}\cdot(\alpha+15^{\circ})=2\sin\alpha;
$$
Будемо працювати з обома виразами і почнемо з основного:
$$
\tan(\alpha+15^{\circ})\cdot\cot(180^{\circ}-15^{\circ})=
-\tan(\alpha+15^{\circ})\cdot\cot15^{\circ}=
-\dfrac{\tan(\alpha+15^{\circ})}{\tan15^{\circ}};
$$
Тепер розберемо другий вираз:
\begin{gather*}
\sin75^{\circ}\cdot(\alpha+15^{\circ}=2\sin\alpha;\\
\sin(90^{\circ}-15^{\circ})\cdot\sin(\alpha+15^{\circ})=2\sin\alpha;\\
\cos15^{\circ}\cdot\sin(\alpha+15^{\circ})=2\sin(\alpha+15^{\circ}-15^{\circ});\\
\cos15^{\circ}\cdot\sin(\alpha+15^{\circ})=
2[\sin(\alpha+15^{\circ})\cdot\cos15^{\circ}-\cos(\alpha+15^{\circ})\cdot\sin15^{\circ}];\\
\cos15^{\circ}\cdot\sin(\alpha+15^{\circ})=
2\sin(\alpha+15^{\circ})\cdot\cos15^{\circ}-2\cos(\alpha+15^{\circ})\cdot\sin15^{\circ};\\
\cos15^{\circ}\cdot\sin(\alpha+15^{\circ})=2\cos(\alpha+15^{\circ})\cdot\sin15^{\circ};\\
\dfrac{\sin(\alpha+15^{\circ}}{\cos(\alpha+15^{\circ})}=2\dfrac{\sin15^{\circ}}{\cos15^{\circ}};\\
\tan(\alpha+15^{\circ})=2\tan15^{\circ};\\
\dfrac{\tan(\alpha+15^{\circ})}{\tan15^{\circ}}=2;
\end{gather*}
Повернемося до нашого основного виразу:
$$
-\dfrac{\tan(\alpha+15^{\circ})}{\tan15^{\circ}}=-2.
$$
\section*{Група 4}