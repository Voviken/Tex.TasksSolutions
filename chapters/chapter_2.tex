\section*{Група 1}
\textbf{1.} Обчислити:
$$
3 \log_{\sqrt{8}} 2 + 2^{-2\log_{\frac{1}{2}} 2}.
$$
\begin{multline*}
3 \log_{\sqrt{8}} 2 + 2^{-2\log_{\frac{1}{2}} 2} =
3\frac{\log_2 2}{\log_2 2^\frac{3}{2}} + 2^{-2 \log_{\frac{1}{2}} \left(\frac{1}{2}\right)^{-1}} =
3\frac{1}{\frac{3}{2}} + 2^{(-2)(-1)\log_{\frac{1}{2}} \frac{1}{2}} =\\
= \cancel{3}\frac{2}{\cancel{3}} + 2^2 = 2 + 4 = 6;
\end{multline*}
\textbf{2.} Обчислити:
$$
3\log_{\sqrt{64}} 4 + 4^{-2\log_{\frac{1}{4}} 3}.
$$
\begin{multline*}
3\log_{\sqrt{64}} 4 + 4^{-2\log_{\frac{1}{4}} 3} =
3\frac{\log_4 4}{\log_4 4^{\frac{3}{2}}} + 4^{\frac{\log_4 \frac{1}{9}}{\log_4 \frac{1}{4}}} =
3\frac{1}{\frac{3}{2}} + 4^{\frac{-\log_4 9}{-\log_4 4}} =\\
= \cancel{3}\frac{2}{\cancel{3}} + 4^{\log_4 9} = 2 + 9 = 11;
\end{multline*}
\textbf{3.} Обчислити:
$$
4\cdot\log_{\sqrt{4096}}8+8^{-2\cdot\log_{\frac{1}{8}}16}.
$$
\begin{multline*}
4\cdot\log_{\sqrt{4096}}8+8^{-2\cdot\log_{\frac{1}{8}}16}=
2\cdot\log_{64}8^2+\left(\dfrac{1}{8}\right)^{(-1)\cdot(-2)\cdot\log_{\frac{1}{8}}16}=\\
=2\cdot\log_{64}64+\left(\dfrac{1}{8}\right)^{\log_{\frac{1}{8}}16^2}=
2+16^2=2+256=258.
\end{multline*}
\textbf{4.} Обчислити:
$$
4\cdot\log_{\sqrt{6561}}9+9^{-2\cdot\log_{\frac{1}{9}}17}.
$$
\begin{multline*}
4\cdot\log_{\sqrt{6561}}9+9^{-2\cdot\log_{\frac{1}{9}}17}=
2\cdot\log_{81}9^2+\left(\dfrac{1}{9}\right)^{(-1)\cdot(-2)\cdot\log_{\frac{1}{9}}17}=\\
=2\cdot\log_{81}81+\left(\dfrac{1}{9}\right)^{\log_{\frac{1}{9}}17^2}=
2+17^2=2+289=291.
\end{multline*}
\textbf{5.} Обчислити:
$$
5\cdot\log_{\sqrt{16807}}7+7^{-2\cdot\log_{\frac{1}{7}}23}.
$$
\begin{multline*}
5\cdot\log_{\sqrt{16807}}7+7^{-2\cdot\log_{\frac{1}{7}}23}=
\log_{\sqrt{16807}}16807+\left(\dfrac{1}{7}\right)^{(-1)\cdot(-2)\cdot\log_{\frac{1}{7}}23}=\\
=\log_{\sqrt{16807}}\left(\sqrt{16807}\right)^2+\left(\dfrac{1}{7}\right)^{\log_{\frac{1}{7}}23^2}=
2+23^2=2+529=531.
\end{multline*}
\textbf{6.} Обчислити:
$$
5\cdot\log_{\sqrt{32768}}8+8^{-2\cdot\log_{\frac{1}{8}}24}.
$$
\begin{multline*}
5\cdot\log_{\sqrt{32768}}8+8^{-2\cdot\log_{\frac{1}{8}}24}=
\log_{\sqrt{32768}}32768+\left(\dfrac{1}{8}\right)^{(-1)\cdot(-2)\cdot\log_{\frac{1}{8}}24}=\\
=\log_{\sqrt{32768}}\left(\sqrt{32768}\right)^2+\left(\dfrac{1}{8}\right)^{\log_{\frac{1}{8}}24^2}=
2+24^2=2+576=578.
\end{multline*}
\textbf{7.} Обчислити:
$$
3\cdot\log_{\sqrt{27}}3+3^{-3\cdot\log_{\frac{1}{3}}4}.
$$
\begin{multline*}
3\cdot\log_{\sqrt{27}}3+3^{-3\cdot\log_{\frac{1}{3}}4}=
\log_{\sqrt{27}}27+\left(\dfrac{1}{3}\right)^{(-1)\cdot(-3)\cdot\log_{\frac{1}{3}}4}=\\
=\log_{\sqrt{27}}\left(\sqrt{27}\right)^2+\left(\dfrac{1}{3}\right)^{\log_{\frac{1}{3}}4^3}=
2+4^3=2+64=66.
\end{multline*}
\textbf{8.} Обчислити:
$$
3\cdot\log_{\sqrt{343}}7+7^{-3\cdot\log_{\frac{1}{7}}8}.
$$
\begin{multline*}
3\cdot\log_{\sqrt{343}}7+7^{-3\cdot\log_{\frac{1}{7}}8}=
\log_{\sqrt{343}}343+\left(\dfrac{1}{7}\right)^{(-1)\cdot(-3)\cdot\log_{\frac{1}{7}}8}=\\
=\log_{\sqrt{343}}\left(\sqrt{343}\right)^2+\left(\dfrac{1}{7}\right)^{\log_{\frac{1}{7}}8^3}=
2+8^3=2+512=514.
\end{multline*}
\textbf{9.} Обчислити:
$$
\frac{\log_4 11 + \log_4 23}{\log_8 253}.
$$
\begin{multline*}
\frac{\log_4 11 + \log_4 23}{\log_8 253} =
\frac{\log_4 11 + \log_4 23}{\log_8 (11 \cdot 23)} =
\frac{\log_4 11 + \log_4 23}{\log_8 11 + \log_8 23} =\\
= \frac{\log_4 11 + \log_4 23}{\frac{\log_4 11}{\log_4 8} + \frac{\log_4 23}{\log_4 8}} =
\cancel{\left(\log_4 11 + \log_4 23\right)} \cdot \frac{\log_4 8}{\cancel{\log_4 11 + \log_4 23}} =\\
= \frac{\log_2 2^3}{\log_2 2^2} = \frac{3}{2} = 1,5;
\end{multline*}
\textbf{10.} Обчислити:
$$
\dfrac{\log_{4}13+\log_{4}25}{\log_{64}325}.
$$
\begin{multline*}
\dfrac{\log_{4}13+\log_{4}25}{\log_{64}325}=
\dfrac{\log_{4}(13\cdot25)}{\dfrac{\log_{4}325}{\log_{4}64}}=
\cancel{\left(\log_{4}325\right)}\cdot\dfrac{\log_{4}64}{\cancel{\log_{4}325}}=
\log_{4}64=3.
\end{multline*}
\textbf{11.} Обчислити:
$$
\dfrac{\log_{25}53+\log_{25}13}{\log_{78125}{689}}.
$$
\begin{multline*}
\dfrac{\log_{25}53+\log_{25}13}{\log_{78125}{689}}=
\dfrac{\log_{25}(53\cdot13)}{\log_{78125}689}=
\dfrac{\dfrac{\log_{5}689}{\log_{5}25}}{\dfrac{\log_{5}689}{\log_{5}78125}}=\\
=\dfrac{\cancel{\log_{5}689}}{2}\cdot\dfrac{\log_{5}5^7}{\cancel{\log_{5}689}}=
\dfrac{7}{2}=3,5.
\end{multline*}
\textbf{12.}Обчислити:
$$
\log_4 a, \;\;\; \mbox{якщо} \;\;\; a=\sin\dfrac{\pi}{6}.
$$
\begin{gather*}
\log_{4}a=
\log_{4}\left(\sin\dfrac{\pi}{6}\right)=
\log_4 \frac{1}{2} =
\frac{\log_2 2^{-1}}{\log_2 2^2}=
-\frac{1}{2} = -0,5;
\end{gather*}
\textbf{13.} Обчислити:
$$
3\cdot\log_{\frac{1}{3}}a, \;\;\; \mbox{якщо} \;\;\; a=2\cos\dfrac{\pi}{6}.
$$
\begin{gather*}
3\cdot\log_{\frac{1}{3}}\left(2\cdot\cos\dfrac{\pi}{6}\right)=
3\cdot\log_{\frac{1}{3}}\sqrt{3}=
\log_{3^{-1}}3^{\frac{3}{2}}=
\dfrac{\log_{3}3^{\frac{3}{2}}}{\log_{3}3^{-1}}=
\dfrac{\dfrac{3}{2}}{-1}=-1,5.
\end{gather*}
\textbf{14.} Обчислити:
$$
-5\cdot\log_{3}a, \;\;\; \mbox{якщо} \;\;\; a=\tan\dfrac{\pi}{6}.
$$
\begin{gather*}
-5\cdot\log_{3}a=
-5\cdot\log_{3}\dfrac{1}{\sqrt{3}}=
-5\cdot\log_{3}3^{-\frac{1}{2}}=
-\dfrac{5}{2}\cdot\log_{3}3=\dfrac{5}{2}=2,5.
\end{gather*}
\textbf{15.} Обчислити:
$$
5\cdot\log_{\frac{1}{3}}a, \;\;\; \mbox{якщо} \;\;\; a=\cot\dfrac{\pi}{6}.
$$
\begin{gather*}
5\cdot\log_{\frac{1}{3}}a=
5\cdot\log_{\frac{1}{3}}\sqrt{3}=
5\cdot\log_{\frac{1}{3}}\left(\dfrac{1}{3}\right)^{-\frac{1}{2}}=
-\dfrac{5}{2}\cdot\log_{\frac{1}{3}}\dfrac{1}{3}=
-\dfrac{5}{2}=-2,5.
\end{gather*}
\textbf{16.} Обчислити:
$$
3\cdot\log_{4}a, \;\;\; \mbox{якщо} \;\;\; a=\sin\dfrac{\pi}{4}.
$$
\begin{gather*}
3\cdot\log_{4}a=
3\cdot\log_{4}\dfrac{1}{\sqrt{2}}=
3\cdot\log_{4}4^{-\frac{1}{4}}=
-\dfrac{3}{4}\cdot\log_{4}4=
-\dfrac{3}{4}=0,75.
\end{gather*}
\textbf{17.} Обчислити:
$$
4\cdot\log_{\frac{1}{2}}a, \;\;\; \mbox{якщо} \;\;\; a=\cos\dfrac{\pi}{4}.
$$
\begin{gather*}
4\cdot\log_{\frac{1}{2}}a=
4\cdot\log_{\frac{1}{2}}\dfrac{1}{\sqrt{2}}=
4\cdot\log_{\frac{1}{2}}\left(\dfrac{1}{2}\right)^{\frac{1}{2}}=
\dfrac{4}{2}\cdot\log_{\frac{1}{2}}\dfrac{1}{2}=2.
\end{gather*}
\textbf{18.} Обчислити:
$$
\log_{a}\dfrac{1}{3}, \;\;\; \mbox{якщо} \;\;\; a=-2\cos\dfrac{5\pi}{6}.
$$
\begin{gather*}
\log_{a}\dfrac{1}{3}=
\log_{\sqrt{3}}\dfrac{1}{3}=
\log_{\sqrt{3}}\left(\sqrt{3}\right)^{-2}=
-2\cdot\log_{\sqrt{3}}\sqrt{3}=-2.
\end{gather*}
\textbf{19.} Обчислити:
$$
\log_{a}\dfrac{1}{9}, \;\;\; \mbox{якщо} \;\;\; a=-\cot\dfrac{5\pi}{6}.
$$
\begin{gather*}
\log_{a}\dfrac{1}{9}=
\log_{\sqrt{3}}\dfrac{1}{9}=
\log_{\sqrt{3}}\left(\sqrt{3}\right)^{-4}=
-4\cdot\log_{\sqrt{3}}\sqrt{3}=-4.
\end{gather*}
\textbf{20.} Обчислити:
$$
\log_{a}9\sqrt{3}, \;\;\; \mbox{якщо} \;\;\; a=2\sin\dfrac{2\pi}{3}.
$$
\begin{gather*}
\log_{a}9\sqrt{3}=
\log_{\sqrt{3}}9\sqrt{3}=
\log_{\sqrt{3}}\left(\sqrt{3}\right)^5=
5\cdot\log_{\sqrt{3}}\sqrt{3}=5.
\end{gather*}
\textbf{21.} Обчислити:
$$
a^{\log_{4}25}, \;\;\; \mbox{якщо} \;\;\; a=\sin\dfrac{\pi}{6}.
$$
\begin{gather*}
a^{\log_{4}25}=
\left(\dfrac{1}{2}\right)^{\log_{4}25}=
4^{-\frac{1}{2}\cdot\log_{4}25}=
4^{\log_{4}25^{-\frac{1}{2}}}=
25^{-\frac{1}{2}}=\dfrac{1}{5}=0,2.
\end{gather*}
\textbf{22.} Обчислити:
$$
a^{\log_{9}16}, \;\;\; \mbox{якщо} \;\;\; a=2\cdot\cos\dfrac{\pi}{6}.
$$
\begin{gather*}
a^{\log_{9}16}=
\left(\sqrt{3}\right)^{\log_{9}16}=
9^{\frac{1}{4}\cdot\log_{9}16}=
9^{\log_{9}16^{\frac{1}{4}}}=
16^{\frac{1}{4}}=2.
\end{gather*}
\textbf{23.} Обчислити:
$$
a^{\log_{3}0,25}, \;\;\; \mbox{якщо} \;\;\; a=\tan\dfrac{\pi}{6}.
$$
\begin{gather*}
a^{\log_{3}0,25}=
\left(\dfrac{1}{\sqrt{3}}\right)^{\log_{3}0,25}=
3^{-\frac{1}{2}\cdot\log_{3}\frac{1}{4}}=
3^{\log_{3}\left(\frac{1}{4}\right)^{-\frac{1}{2}}}=
\left(\dfrac{1}{4}\right)^{-\frac{1}{2}}=2.
\end{gather*}
\textbf{24.} Обчислити:
$$
\frac{1}{\log_{8}12}+\frac{1}{\log_{18}12}.
$$
\begin{gather*}
\frac{1}{\log_{8}12}+\frac{1}{\log_{18}12}=
\log_{12}8+\log_{12}18=
\log_{12}(8\cdot18)=
\log_{12}144=2.
\end{gather*}
\textbf{25.} Обчислити:
$$
-\log_{2}\left(\log_{\sqrt{2}}\sqrt[4]{2}\right).
$$
\begin{multline*}
-\log_{2}\left(\log_{\sqrt{2}}\sqrt[4]{2}\right)=
-\log_{2}\left(\log_{\sqrt{2}}\left(\sqrt{2}\right)^{\frac{1}{2}}\right)=
-\log_{2}\left(\dfrac{1}{2}\cdot\log_{\sqrt{2}}\sqrt{2}\right)=\\
=-\log_{2}\dfrac{1}{2}=-\log_{2}2^{-1}=1.
\end{multline*}
\textbf{26.} Обчислити:
$$
\log_{\sqrt{2}}4+\log_{\frac{1}{3}}9.
$$
\begin{gather*}
\log_{\sqrt{2}}4+\log_{\frac{1}{3}}9=
\log_{\sqrt{2}}\left(\sqrt{2}\right)^4+\log_{\frac{1}{3}}\left(\dfrac{1}{3}\right)^{-2}=
4-2=2.
\end{gather*}
\textbf{27.} Обчислити:
$$
9^{\log_{27}8}+4^{\log_{2}3}.
$$
\begin{multline*}
9^{\log_{27}8}+4^{\log_{2}3}=
27^{\frac{2}{3}\cdot\log_{27}8}+2^{2\cdot\log_{2}8}=
27^{\log_{27}8^{\frac{2}{3}}}+2^{\log_{2}3^2}=\\
=8^{\frac{2}{3}}+3^2=4+9=13.
\end{multline*}
\textbf{28.} Обчислити:
$$
\log_{0,2}25+\log_{0,5}4.
$$
\begin{gather*}
\log_{0,2}25+\log_{0,5}4=
\log_{\frac{1}{5}}\left(\dfrac{1}{5}\right)^{-2}+\log_{\frac{1}{2}}\left(\dfrac{1}{2}\right)^{-2}=
-2-2=-4.
\end{gather*}
\textbf{29.} Обчислити:
$$
\left(\log_{\sqrt{2}}3\right)\cdot\left(\log_{3}8\right).
$$
Перейдемо в обох множниках до основи $2$:
\begin{gather*}
\left(\log_{\sqrt{2}}3\right)\cdot\left(\log_{3}8\right)=
\dfrac{\cancel{\log_{2}3}}{\log_{2}\sqrt{2}}\cdot\dfrac{\log_{2}8}{\cancel{\log_{2}3}}=
\dfrac{\log_{2}2^3}{\log_{2}2^{\frac{1}{2}}}=
\dfrac{3}{\frac{1}{2}}=6.
\end{gather*}
\textbf{30.} Обчислити:
$$
\log_{4}32.
$$
\begin{gather*}
\log_{4}32=
\log_{4}4^{\frac{5}{2}}=
\dfrac{5}{2}=2,5.
\end{gather*}
\textbf{31.} Обчислити:
$$
\log_{4}128.
$$
\begin{gather*}
\log_{4}128=
\log_{4}4^{\frac{7}{2}}=
\dfrac{7}{2}=3,5.
\end{gather*}
\textbf{32.} Обчислити:
$$
\log_{16}8.
$$
Перейдемо до основи $2$:
\begin{gather*}
\log_{16}8=
\dfrac{\log_{2}8}{\log_{2}16}=
\dfrac{3}{4}=0,75.
\end{gather*}
\textbf{33.} Обчислити:
$$
\log_{16}128.
$$
Перейдемо до основи $2$:
\begin{gather*}
\log_{16}128=
\dfrac{\log_{2}128}{\log_{2}16}=
\dfrac{7}{4}=1,75.
\end{gather*}
\textbf{34.} Обчислити:
$$
\log_{9}243.
$$
Перейдемо до основи $3$:
\begin{gather*}
\log_{9}243=
\dfrac{\log_{3}243}{\log_{3}9}=
\dfrac{5}{2}=2,5.
\end{gather*}
\textbf{35.} Обчислити:
$$
\log_{81}243.
$$
Перейдемо до основи $3$:
\begin{gather*}
\log_{81}243=
\dfrac{\log_{3}243}{\log_{3}81}=
\dfrac{5}{4}=1,25.
\end{gather*}
\textbf{36.} Обчислити:
$$
\log_{81}2187.
$$
Перейдемо до основи $3$:
\begin{gather*}
\log_{81}2187=
\dfrac{\log_{3}2187}{\log_{3}81}=
\dfrac{7}{4}=1,75.
\end{gather*}
\textbf{37.} Обчислити:
$$
3\cdot\log_{64}128.
$$
Перейдемо до основи $2$:
\begin{gather*}
3\cdot\log_{64}128=
3\cdot\dfrac{\log_{2}128}{\log_{2}64}=
3\cdot\dfrac{7}{6}=3,5.
\end{gather*}
\textbf{38.} Обчислити:
$$
9\cdot\log_{125}3125.
$$
Перейдемо до основи $5$:
\begin{gather*}
9\cdot\log_{125}3125=
9\cdot\dfrac{\log_{5}3125}{\log_{5}125}=
9\cdot\dfrac{\log_{5}5^5}{\log_{5}5^3}=
9\cdot\dfrac{5}{3}=15.
\end{gather*}
\textbf{39.} Обчислити:
$$
7\cdot\log_{25}3125.
$$
Перейдемо до основи $5$:
\begin{gather*}
7\cdot\log_{25}3125=
7\cdot\dfrac{\log_{5}3125}{\log_{5}25}=
7\cdot\dfrac{\log_{5}5^5}{\log_{5}5^2}=
7\cdot\dfrac{5}{2}=17,5.
\end{gather*}
\textbf{40.} Обчислити:
$$
81^{\log_{3}5}.
$$
\begin{gather*}
81^{\log_{3}5}=
3^{4\cdot\log_{3}5}=
3^{\log_{3}5^4}=
5^4=625.
\end{gather*}
\textbf{41.} Обчислити:
$$
49^{\log_{7}3}.
$$
\begin{gather*}
49^{\log_{7}3}=
7^{2\cdot\log_{7}3}=
7^{\log_{7}3^2}=
3^2=9.
\end{gather*}
\textbf{42.} Обчислити:
$$
27^{\log_{3}4}.
$$
\begin{gather*}
27^{\log_{3}4}=
3^{3\cdot\log_{3}4}=
3^{\log_{3}4^3}=
4^3=64.
\end{gather*}
\textbf{43.} Обчислити:
$$
121^{\log_{11}12}.
$$
\begin{gather*}
121^{\log_{11}12}=
11^{2\cdot\log_{11}12}=
11^{\log_{11}12^2}=
12^2=144.
\end{gather*}
\textbf{44.} Обчислити:
$$
216^{\log_{36}49}.
$$
В степені перейдемо до основи $6$:
\begin{multline*}
216^{\log_{36}49}=
6^{3\cdot\frac{\log_{6}49}{\log_{6}36}}=
6^{3\cdot\frac{\log_{6}49}{2}}=
6^{\frac{3}{2}\cdot\log_{6}49}=
6^{\log_{6}49^{\frac{3}{2}}}=\\
=49^{\frac{3}{2}}=7^3=343.
\end{multline*}
\textbf{45.} Обчислити:
$$
49^{\log_{343}27}.
$$
В степені перейдемо до основи $7$:
\begin{multline*}
49^{\log_{343}27}=
7^{2\cdot\frac{\log_{7}27}{\log_{7}343}}=
7^{2\cdot\frac{\log_{7}27}{3}}=
7^{\frac{2}{3}\cdot\log_{7}27}=
7^{\log_{7}27^{\frac{2}{3}}}=
27^{\frac{2}{3}}=3^2=9.
\end{multline*}
\textbf{46.} Обчислити:
$$
9^{\log_{3}2}+3^{1-2\cdot\log_{3}2}.
$$
\begin{multline*}
9^{\log_{3}2}+3^{1-2\cdot\log_{3}2}=
3^{2\cdot\log_{3}2}+3^1\cdot3^{-2\cdot\log_{3}2} =
3^{\log_{3}4}+3\cdot3^{\log_{3}\frac{1}{4}} =\\
=4+3\cdot\dfrac{1}{4}=4,75.
\end{multline*}
\textbf{47.} Обчислити:
$$
16^{\log_{4}2}+4^{1-2\cdot\log_{4}2}.
$$
\begin{multline*}
16^{\log_{4}2}+4^{1-2\cdot\log_{4}2}=
4^{2\cdot\log_{4}2}+4^1\cdot4^{-2\cdot\log_{4}2}=
4^{\log_{4}2^2}+4\cdot4^{\log_{4}2^{-2}}=\\
=2^2+4\cdot2^{-2}=4+4\cdot\dfrac{1}{4}=4+1=5.
\end{multline*}
\textbf{48.} Обчислити:
$$
49^{\log_{7}2}+7^{1-2\cdot\log_{7}2}.
$$
\begin{multline*}
49^{\log_{7}2}+7^{1-2\cdot\log_{7}2}=
7^{2\cdot\log_{7}2}+7^1\cdot7^{-2\cdot\log_{7}2}=
7^{\log_{7}2^2}+7\cdot7^{\log_{7}2^{-2}}=\\
=2^2+7\cdot2^{-2}=
4+\dfrac{7}{4}=5,75.
\end{multline*}
\textbf{49.} Обчислити:
$$
512^{\log_{8}2}+8^{1-2\cdot\log_{8}2}.
$$
\begin{multline*}
512^{\log_{8}2}+8^{1-2\cdot\log_{8}2}=
8^{3\cdot\log_{8}2}+8^1\cdot8^{-2\cdot\log_{8}2}=
8^{\log_{8}2^3}+8\cdot8^{\log_{8}2^{-2}}=\\
=2^3+8\cdot2^{-2}=
8+\dfrac{8}{4}=10.
\end{multline*}
\textbf{50.} Обчислити:
$$
216^{\log_{6}2}+6^{2-2\cdot\log_{6}2}.
$$
\begin{multline*}
216^{\log_{6}2}+6^{2-2\cdot\log_{6}2}=
6^3\cdot\log_{6}2+6^{2}\cdot6^{-2\cdot\log_{6}2}=
6^{\log_{6}2^3}+36\cdot6^{\log_{6}2^{-2}}=\\
=2^3+36\cdot2^{-2}=
8+\dfrac{36}{4}=8+9=17.
\end{multline*}
\textbf{51.} Обчислити:
$$
3^{\dfrac{1}{\log_{8}27}}.
$$
\begin{gather*}
3^{\dfrac{1}{\log_{8}27}}=
3^{\log_{27}8}=
27^{\frac{1}{3}\cdot\log_{27}8}=
27^{\log_{27}8^{}\frac{1}{3}}=
8^{\frac{1}{3}}=2.
\end{gather*}
\textbf{52.} Обчислити:
$$
\log^{2}_{4}\log_{3}\sqrt{81}.
$$
\begin{gather*}
\log^{2}_{4}\log_{3}\sqrt{81}=
\log^{2}_{4}\log_{3}9=
\log^{2}_{4}2=
\log^{2}_{4}4^{\frac{1}{2}}=
\left(\dfrac{1}{2}\cdot\log_{4}4\right)^2=
\left(\dfrac{1}{2}\right)^2=0,25.
\end{gather*}
\textbf{53.} Обчислити:
$$
6^{\dfrac{2}{\log_{5}6}}.
$$
\begin{gather*}
6^{\dfrac{2}{\log_{5}6}}=
6^{2\cdot\dfrac{1}{\log_{5}6}}=
6^{2\cdot\log_{6}5}=
6^{\log_{6}5^2}=5^2=25.
\end{gather*}
\textbf{54.} Обчислити:
$$
\log^{2}_{3}\log_{\frac{1}{5}}\dfrac{1}{125}.
$$
\begin{gather*}
\log^{2}_{3}\log_{\frac{1}{5}}\dfrac{1}{125}=
\log^{2}_{3}3=
\left(\log_{3}3\right)^2=1.
\end{gather*}
\textbf{55.} Обчислити:
$$
3^{\dfrac{3}{\log_{7}3}}.
$$
\begin{gather*}
3^{\dfrac{3}{\log_{7}3}}=
3^{3\cdot\dfrac{1}{\log_{7}3}}=
3^{3\cdot\log_{3}7}=
3^{\log_{3}7^3}=7^3=343.
\end{gather*}
\textbf{56.} Обчислити:
$$
\log^{2}_{9}\log_{0,2}\dfrac{1}{125}.
$$
\begin{gather*}
\log^{2}_{9}\log_{0,2}\dfrac{1}{125}=
\log^{2}_{9}\log_{\frac{1}{5}}\dfrac{1}{125}=
\log^{2}_{9}3=
\left(\log_{9}9^{\frac{1}{2}}\right)^2=
\left(\dfrac{1}{2}\right)^2=0,25.
\end{gather*}
\section*{Група 2}
\textbf{1.} Обчислити:
$$
\log_{3}\log_{2}\left(\sqrt[3]{2^k}\right)^{\frac{1}{3}}, \;\;\; \mbox{якщо} \;\;\; \log_{3}k=10.
$$
\begin{multline*}
\log_{3}\log_{2}\left(\sqrt[3]{2^k}\right)^{\frac{1}{3}}=
\log_{3}\log_{2}2^{\frac{k}{3}\cdot\frac{1}{3}}=
\log_{3}\left(\frac{k}{9}\log_{2}2\right)=
\log_{3}\frac{k}{9}=\\
=\log_{3}k-\log_{3}9=
10-2=8.
\end{multline*}
\textbf{2.} Обчислити:
$$
3^{\log_{3}8-2\cdot\log_{3}2+\log_{3}\frac{9}{2}}.
$$
\begin{gather*}
3^{\log_{3}8-2\cdot\log_{3}2+\log_{3}\frac{9}{2}}=
3^{\log_{3}8}\cdot3^{-2\cdot\log_{3}2}\cdot3^{\log_{3}\frac{3}{2}}=
8\cdot2^{-2}\cdot\dfrac{9}{2}=9.
\end{gather*}
\textbf{3.} Обчислити:
$$
\log_{3}\log_{4}\left(\sqrt[3]{4^k}\right)^{\frac{1}{3}}, \;\;\; \mbox{якщо} \;\;\; \log_{3}k=16.
$$
\begin{multline*}
\log_{3}\log_{4}\left(\sqrt[3]{4^k}\right)^{\frac{1}{3}}=
\log_{3}\log_{4}4^{\frac{k}{9}}=
\log_{3}\left(\dfrac{k}{9}\cdot\log_{4}4\right)=
\log_{3}\dfrac{k}{9}=\\
=\log_{3}k-\log_{3}9=
16-2=14.
\end{multline*}
\textbf{4.} Обчислити:
$$
7^{-\log_{7}32+\log_{7}256+2\cdot\log_{7}14}.
$$
\begin{multline*}
7^{-\log_{7}32+\log_{7}256+2\cdot\log_{7}14}=
7^{\log_{7}256+\log_{7}14^2-\log_{7}32}=
7^{\log_{7}(256\cdot14^2)-\log_{7}32}=\\
=7^{\log_{7}\frac{256\cdot196}{32}}=
\dfrac{256\cdot196}{32}=1536.
\end{multline*}
\textbf{5.} Обчислити:
$$
\log_{3}\log_{5}\left(5^k\right)^{3^5}, \;\;\; \mbox{якщо} \;\;\; \log_{3}k=11.
$$
\begin{multline*}
\log_{3}\log_{5}\left(5^k\right)^{3^5}=
\log_{3}\log_{5}5^{k\cdot3^5}=
\log_{3}\left(k\cdot3^5\cdot\log_{5}5\right)=\\
=\log_{3}\left(k\cdot3^5\right)=
\log_{3}k+\log_{3}3^5=11+5=16.
\end{multline*}
\textbf{6.} Обчислити:
$$
3^{\log_{3}63-\log_{3}\frac{14}{27}+\log_{3}18}.
$$
\begin{multline*}
3^{\log_{3}63-\log_{3}\frac{14}{27}+\log_{3}18}=
3^{\log_{3}\frac{63}{\frac{14}{27}}+\log_{3}18}=
3^{\log_{3}\frac{63\cdot18\cdot27}{14}}=\\
=\dfrac{63\cdot18\cdot27}{14}=2187.
\end{multline*}
\textbf{7.} Обчислити:
$$
\log_{3}\log_{7}\left(\sqrt[3]{7^k}\right)^{\frac{1}{3}}, \;\;\; \mbox{якщо} \;\;\; \log_{3}k=21.
$$
\begin{multline*}
\log_{3}\log_{7}\left(\sqrt[3]{7^k}\right)^{\frac{1}{3}}=
\log_{3}\log_{7}7^{\frac{k}{9}}=
\log_{3}\left(\dfrac{k}{9}\cdot\log_{7}7\right)=\\
=\log_{3}\dfrac{k}{9}=
\log_{3}k-\log_{3}9=21-2=19.
\end{multline*}
\textbf{8.} Обчислити:
$$
\left(\log_{3}8\right)\cdot\left(\log_{17}3\right)\cdot\left(\log_{4}17\right).
$$
Перейдемо всюди до основи $2$:
\begin{gather*}
\left(\log_{3}8\right)\cdot\left(\log_{17}3\right)\cdot\left(\log_{4}17\right)=
\dfrac{\log_{2}8}{\cancel{\log_{2}3}}\cdot\dfrac{\cancel{\log_{2}3}}{\cancel{\log_{2}17}}\cdot\dfrac{\cancel{\log_{2}17}}{\log_{2}4}=
\dfrac{3}{2}=1,5.
\end{gather*}
\textbf{9.} Обчислити:
$$
\left(\log_{8}343\right)\cdot\left(\log_{6}8\right)\cdot\left(\log_{49}6\right).
$$
Перейдемо всюди до основи $7$:
\begin{gather*}
\left(\log_{8}343\right)\cdot\left(\log_{6}8\right)\cdot\left(\log_{49}6\right)=
\dfrac{\log_{7}343}{\cancel{\log_{7}8}}\cdot\dfrac{\cancel{\log_{7}8}}{\cancel{\log_{7}6}}\cdot\dfrac{\cancel{\log_{7}6}}{\log_{7}49}=
\dfrac{3}{2}=1,5.
\end{gather*}
\textbf{10.} Обчислити:
$$
\left(\log_{5}512\right)\cdot\left(\log_{7}5\right)\cdot\left(\log_{\frac{1}{2}}7\right).
$$
Перейдемо всюди до основи $2$:
\begin{gather*}
\left(\log_{5}512\right)\cdot\left(\log_{7}5\right)\cdot\left(\log_{\frac{1}{2}}7\right)=
\dfrac{\log_{2}512}{\cancel{\log_{2}5}}\cdot\dfrac{\cancel{\log_{2}5}}{\cancel{\log_{2}7}}\cdot\dfrac{\cancel{\log_{2}7}}{\log_{2}\frac{1}{2}}=
\dfrac{9}{-1}=-9.
\end{gather*}
\textbf{11.} Обчислити:
$$
\left(2^{\log_{4}\left(\sqrt{3}-2\right)^2}+3^{\log_{3}\left(\sqrt{3}+2\right)}\right)^2.
$$
Проаналізуймо перший доданок:
$$
2^{\log_{4}\left(\sqrt{3}-2\right)^2}.
$$
Підлогарифмічний вираз завжди додатний, оскільки він підноситься до квадрату. Припустімо, що ми
зробимо таке перетворення:
$$
2^{2\cdot\log_{4}\left(\sqrt{3}-2\right)}.
$$
В такому випадку, вираз в дужках стане від'ємним, бо $\sqrt{3}<2$. А це суперечить тому, що
він завжди додатний. Щоб поправити ситуацію, ми повинні поміняти місцями доданки у дужках, тобто:
$$
2^{2\cdot\log_{4}\left(2-\sqrt{3}\right)}.
$$
Отже, використаємо ці міркування:
\begin{multline*}
\left(2^{\log_{4}\left(\sqrt{3}-2\right)^2}+3^{\log_{3}\left(\sqrt{3}+2\right)}\right)^2=
\left(2^{2\cdot\log_{4}\left(\textcolor{red}{2-\sqrt{3}}\right)}+3^{\log_{3}\left(\sqrt{3}+2\right)}\right)^2=\\
=\left(4^{\log_{4}\left(2-\sqrt{3}\right)}+3^{\log_{3}\left(\sqrt{3}+2\right)}\right)^2=
\left(2-\cancel{\sqrt{3}}+\cancel{\sqrt{3}}+2\right)^2=\\
=(2+2)^2=4^2=16.
\end{multline*}
\textbf{12.} Обчислити:
$$
\left(5^{\log_{5}\left(\sqrt{3}+8\right)}-3^{\log_{9}\left(\sqrt{3}-8\right)^2}\right)^2.
$$
Проаналізуймо другий доданок:
$$
3^{\log_{9}\left(\sqrt{3}-8\right)^2}.
$$
Підлогарифмічний вираз завжди додатний, оскільки він підноситься до квадрату. Припустімо, що ми
зробимо таке перетворення:
$$
3^{2\cdot\log_{9}\left(\sqrt{3}-8\right)}.
$$
В такому випадку, вираз в дужках стане від'ємним, бо $\sqrt{3}<8$. А це суперечить тому, що
він завжди додатний. Щоб поправити ситуацію, ми повинні поміняти місцями доданки у дужках, тобто:
$$
3^{2\cdot\log_{9}\left(8-\sqrt{3}\right)}.
$$
Отже, використаємо ці міркування:
\begin{multline*}
\left(5^{\log_{5}\left(\sqrt{3}+8\right)}-3^{\log_{9}\left(\sqrt{3}-8\right)^2}\right)^2=
\left(5^{\log_{5}\left(\sqrt{3}+8\right)}-3^{2\cdot\log_{9}\left(8-\sqrt{3}\right)}\right)^2=\\
=\left(5^{\log_{5}\left(\sqrt{3}+8\right)}-9^{\log_{9}\left(8-\sqrt{3}\right)}\right)^2=
\left(\sqrt{3}+\cancel{8}-\cancel{8}+\sqrt{3}\right)^2=\\
=(2\sqrt{3})^2=12.
\end{multline*}
\textbf{13.} Обчислити:
$$
8+\dfrac{1}{2}\cdot\log_{\frac{2}{3}}\left(6-\sqrt{20}\right)+\dfrac{1}{2}\cdot\log_{\frac{2}{3}} \left(6+\sqrt{20}\right)-\log_{\frac{2}{3}}9.
$$
\begin{multline*}
8+\dfrac{1}{2}\cdot\log_{\frac{2}{3}}\left(6-\sqrt{20}\right)+\dfrac{1}{2}\cdot\log_{\frac{2}{3}} \left(6+\sqrt{20}\right)-\log_{\frac{2}{3}}9=\\
8+\log_{\frac{2}{3}}\sqrt{6-\sqrt{20}}+\log_{\frac{2}{3}}\sqrt{6+\sqrt{20}}-\log_{\frac{2}{3}}9=\\
8+\log_{\frac{2}{3}}\frac{\sqrt{\left(6-\sqrt{20}\right)\left(6+\sqrt{20}\right)}}{9}=
8+\log_{\frac{2}{3}}\frac{\sqrt{36-20}}{9}=\\
8+\log_{\frac{2}{3}}\frac{4}{9}=8+2=10.
\end{multline*}
\textbf{14.} Обчислити:
$$
13+\dfrac{1}{2}\cdot\log_{\frac{2}{5}}\left(11-\sqrt{57}\right)+\dfrac{1}{2}\cdot\log_{\frac{2}{5}}\left(11-\sqrt{57}\right)-\log_{\frac{2}{5}}125.
$$
\begin{multline*}
13+\dfrac{1}{2}\cdot\log_{\frac{2}{5}}\left(11-\sqrt{57}\right)+\dfrac{1}{2}\cdot\log_{\frac{2}{5}}\left(11-\sqrt{57}\right)-\log_{\frac{2}{5}}125=\\
=13+\dfrac{1}{2}\cdot\left[\log_{\frac{2}{5}}\left(11-\sqrt{57}\right)+\log_{\frac{2}{5}}\left(11+\sqrt{57}\right)\right]-\log_{\frac{2}{5}}125=\\
=13+\log_{\frac{2}{5}}\left[\left(11-\sqrt{57}\right)\cdot\left(11+\sqrt{57}\right)\right]^{\frac{1}{2}}-\log_{\frac{2}{5}}125=\\
=13+\log_{\frac{2}{5}}\dfrac{\sqrt{121-57}}{125}=
13+\log_{\frac{2}{5}}\dfrac{8}{125}=
13+3=16.
\end{multline*}
\textbf{15.} Обчислити:
$$
8+\dfrac{1}{2}\cdot\log_{\frac{2}{7}}\left(6-\sqrt{20}\right)+\dfrac{1}{2}\cdot\log_{\frac{2}{7}}\left(6+\sqrt{20}\right)-\log_{\frac{2}{7}}49.
$$
\begin{multline*}
8+\dfrac{1}{2}\cdot\log_{\frac{2}{7}}\left(6-\sqrt{20}\right)+\dfrac{1}{2}\cdot\log_{\frac{2}{7}}\left(6+\sqrt{20}\right)-\log_{\frac{2}{7}}49=\\
=8+\dfrac{1}{2}\cdot\left[\log_{\frac{2}{7}}\left(6-\sqrt{20}\right)+\log_{\frac{2}{7}}\left(6+\sqrt{20}\right)\right]-\log_{\frac{2}{7}}49=\\
=8+\log_{\frac{2}{7}}\left[\left(6-\sqrt{20}\right)\cdot\left(6+\sqrt{20}\right)\right]^{\frac{1}{2}}-\log_{\frac{2}{7}}49=
8+\log_{\frac{2}{7}}\dfrac{\sqrt{36-20}}{49}=\\
=8+\log_{\frac{2}{7}}\dfrac{4}{49}=8+2=10.
\end{multline*}
\textbf{16.} Обчислити:
$$
\frac{\log^{2}_{2}14+\left(\log_{2}14\right)\left(\log_{2}7\right)-2\log^{2}_{2}7}{\log_{2}14+2\log_{2}7}.
$$
\begin{multline*}
\frac{\log^{2}_{2}14+\left(\log_{2}14\right)\left(\log_{2}7\right)-2\log^{2}_{2}7}{\log_{2}14+2\log_{2}7}=\\
=\frac{\left(\log_{2}2+\log_{2}7\right)^2+\left(\log_{2}2+\log_{2}7\right)\left(\log_{2}7\right)-
2\log^{2}_{2}7}{\log_{2}2+\log_{2}7+2\cdot\log_{2}7}=\\
\frac{\left(1+\log_{2}7\right)^2+\left(1+\log_{2}7\right)\left(\log_{2}7\right)-2\cdot\log^{2}_{2}7}{1+3\log_{2}7}=\\
\frac{1+2\log_{2}7+\cancel{\log^{2}_{2}7}+\log_{2}7+\cancel{\log^{2}_{2}7}-\cancel{2\log^{2}_{2}7}}{1+3\log_{2}7}=\\
=\frac{1+3\cdot\log_{2}7}{1+3\cdot\log_{2}7}=1;
\end{multline*}
\textbf{17.} Обчислити:
$$
\dfrac{2\cdot\log_{3}12-4\cdot\log^{2}_{3}2+\log^{2}_{3}12+4\cdot\log_{3}2}{3\cdot\log_{3}12+6\cdot\log_{3}2}.
$$
Візьмемо до уваги, що:
$$
\log_{3}12=
\log_{3}(3\cdot4)=
\log_{3}3+\log_{3}4=
1+\log_{3}2^2=
1+2\cdot\log_{3}2.
$$
\begin{multline*}
\dfrac{2\cdot\log_{3}12-4\cdot\log^{2}_{3}2+\log^{2}_{3}12+4\cdot\log_{3}2}{3\cdot\log_{3}12+6\cdot\log_{3}2}=\\
=\dfrac{2\cdot\left(1+2\cdot\log_{3}2\right)-4\cdot\log^{2}_{3}2+\left(1+2\cdot\log_{3}2\right)^2+
4\cdot\log_{3}2}{3\cdot\left(1+2\cdot\log_{3}2\right)+6\cdot\log_{3}2}=\\
=\dfrac{2+4\cdot\log_{3}2-\cancel{4\cdot\log^{2}_{3}2}+1+4\cdot\log_{3}2+
4\cdot\cancel{4\cdot\log^{2}_{3}2}+4\cdot\log_{3}2}{3+6\cdot\log_{3}2+6\cdot\log_{3}2}=\\
=\dfrac{3+12\cdot\log_{3}2}{3+12\cdot\log_{3}2}=1.
\end{multline*}
\textbf{18.} Обчислити:
$$
\dfrac{\log^{2}_{2}28+\left(\log_{2}28\right)\left(\log_{2}7\right)-2\cdot\log^{2}_{2}7}{\log_{2}28+2\cdot\log_{2}7}.
$$
Візьмемо до уваги, що:
$$
\log_{2}28=
\log_{2}(4\cdot7)=
\log_{2}4+\log_{2}7=
2+\log_{2}7.
$$
Скористаємося цим і отримаємо:
\begin{multline*}
\dfrac{\log^{2}_{2}28+\left(\log_{2}28\right)\left(\log_{2}7\right)-2\cdot\log^{2}_{2}7}{\log_{2}28+2\cdot\log_{2}7}=\\
=\dfrac{\left(2+\log_{2}7\right)^2+\left(2+\log_{2}7\right)\cdot\left(\log_{2}7\right)-
2\cdot\log^{2}_{2}7}{2+\log_{2}7+2\cdot\log_{2}7}=\\
=\dfrac{4+4\cdot\log_{2}7+\cancel{\log^{2}_{2}7}+2\cdot\log_{2}7+\cancel{\log^{2}_{2}7}-
\cancel{2\cdot\log^{2}_{2}7}}{2+3\cdot\log_{2}7}=\\
=\dfrac{4+6\cdot\log_{2}7}{2+3\cdot\log_{2}7}=
\dfrac{2\cdot\left(2+3\cdot\log_{2}7\right)}{2+3\cdot\log_{2}7}=2.
\end{multline*}
\textbf{19.} Обчислити:
$$
\dfrac{\log^{2}_{2}12-2\log_{2}12+2\log^{2}_{2}3-3\left(\log_{2}3\right)\left(\log_{2}12\right)+
4\log_{2}3}{\log_{2}12-2\log_{2}3}.
$$
Візьмемо до уваги, що:
$$
\log_{2}12=
\log_{2}(4\cdot3)=
\log_{2}4+\log_{2}3=
2+\log_{2}3.
$$
Скористаємося цим і отримаємо:
\begin{multline*}
\dfrac{\log^{2}_{2}12-2\log_{2}12+2\log^{2}_{2}3-3\left(\log_{2}3\right)\left(\log_{2}12\right)+
4\log_{2}3}{\log_{2}12-2\log_{2}3}=\\
=\dfrac{\left(2+\log_{2}3\right)^2-2\left(2+\log_{2}3\right)+2\log^{2}_{2}3-
3\left(\log_{2}3\right)\left(2+\log_{2}3\right)+4\log_{2}3}{2+\log_{2}3-2\log_{2}3}=\\
=\dfrac{4+4\log_{2}3+\log^{2}_{2}3-4-2\log_{2}3+2\log^{2}_{2}3-6\log_{2}3-3\log^{2}_{2}3+
4\log_{2}3}{2-\log_{2}3}=\\
=\dfrac{0}{2-\log_{2}3}=0.
\end{multline*}
\textbf{20.} Обчислити:
$$
\dfrac{\log^{2}_{35}7-2\left(\log_{35}7\right)\left(\log_{35}5\right)-3\log^{2}_{35}5}{2\left(\log_{35}7-3\log_{35}5\right)}.
$$
Введемо заміну:
$$
\log_{35}7=a, \;\;\; \log_{35}5=b.
$$
Отримаємо:
\begin{multline*}
\dfrac{a^2-2ab-3b^2}{2(a-3b)}=
\dfrac{a^2-2ab+b^2-4b^2}{2(a-3b)}=
\dfrac{(a-b)^2-4b^2}{2(a-3b)}=\\
=\dfrac{(a-b-2b)(a-b+2b)}{2(a-3b)}=
\dfrac{\cancel{(a-3b)}(a+b)}{2\cancel{(a-3b)}}=
\dfrac{a+b}{2}
\end{multline*}
Повернемося до заміни. Отримаємо:
$$
\dfrac{\log_{35}7+\log_{35}5}{2}=
\dfrac{\log_{35}(7\cdot5)}{2}=
\dfrac{1}{2}=0,5.
$$
\section*{Група 3}
\textbf{1.} Обчислити:
$$
\log_{ab}x, \;\;\; \mbox{якщо} \;\;\; \log_{a}x=2, \;\;\; \log_{b}x=3.
$$
\begin{multline*}
\log_{ab}x =\dfrac{1}{\log_{x}ab}=\dfrac{1}{\log_{x}a+\log_{x}b}=
\dfrac{1}{\dfrac{1}{\log_{a}x}+\dfrac{1}{\log_{b}x}}=\\
=\dfrac{1}{\dfrac{1}{2}+\dfrac{1}{3}}=\dfrac{1}{\dfrac{5}{6}}=\dfrac{6}{5}=1,2.
\end{multline*}
\textbf{2.} Обчислити:
$$
\log_{\frac{5}{a}}25, \;\;\; \mbox{якщо} \;\;\; \log_{a}5=2.
$$
\begin{multline*}
\log_{\frac{5}{a}}25=
\dfrac{\log_{5}25}{\log_{5}\dfrac{5}{a}}=
\dfrac{2}{\log_{5}5-\log_{5}a}=
\dfrac{2}{1-\dfrac{1}{\log_{a}5}}=
\dfrac{2}{1-\dfrac{1}{2}}=\dfrac{2}{\dfrac{1}{2}}=4.
\end{multline*}
\textbf{3.} Обчислити:
$$
\log_{\frac{a}{b}}x, \;\;\; \mbox{якщо} \;\;\; \log_{a}x=2, \;\;\; \log_{b}x=3.
$$
\begin{gather*}
\log_{\frac{a}{b}}x =\dfrac{1}{\log_{x}\frac{a}{b}}=\dfrac{1}{\log_{x}a-\log_{x}b}=
\dfrac{1}{\dfrac{1}{\log_{a}x}-\dfrac{1}{\log_{b}x}}=
\dfrac{1}{\dfrac{1}{2}-\dfrac{1}{3}}=\dfrac{1}{\dfrac{1}{6}}=6.
\end{gather*}
\textbf{4.} Обчислити:
$$
\log_{5a}25, \;\;\; \mbox{якщо} \;\;\; \log_{a}5=3.
$$
\begin{gather*}
\log_{5a}25=
\dfrac{\log_{5}25}{\log_{5}5a}=
\dfrac{2}{\log_{5}5+\log_{5}a}=
\dfrac{2}{1+\dfrac{1}{\log_{a}5}}=
\dfrac{2}{1+\dfrac{1}{3}}=
\dfrac{2}{\dfrac{4}{3}}=1,5.
\end{gather*}
\textbf{5.} Обчислити:
$$
\log_{\frac{a}{b}}x, \;\;\; \mbox{якщо} \;\;\; \log_{a}x=2, \;\;\; \log_{b}x=4.
$$
\begin{gather*}
\log_{\frac{a}{b}}x =\dfrac{1}{\log_{x}\frac{a}{b}}=\dfrac{1}{\log_{x}a-\log_{x}b}=
\dfrac{1}{\dfrac{1}{\log_{a}x}-\dfrac{1}{\log_{b}x}}=
\dfrac{1}{\dfrac{1}{2}-\dfrac{1}{4}}=\dfrac{1}{\dfrac{1}{4}}=4.
\end{gather*}
\textbf{6.} Обчислити:
$$
\log_{\frac{b}{a}}x, \;\;\; \mbox{якщо} \;\;\; \log_{a}x=1, \;\;\; \log_{b}x=3.
$$
\begin{gather*}
\log_{\frac{b}{a}}x =\dfrac{1}{\log_{x}\frac{b}{a}}=\dfrac{1}{\log_{x}b-\log_{x}a}=
\dfrac{1}{\dfrac{1}{\log_{b}x}-\dfrac{1}{\log_{a}x}}=
\dfrac{1}{\dfrac{1}{3}-1}=\dfrac{1}{-\dfrac{2}{3}}=-1,5.
\end{gather*}
\textbf{7.} Обчислити:
$$
\log_{ab}\dfrac{\sqrt{b}}{a}+\log_{\sqrt{ab}}b+\log_{a}\sqrt[3]{b}, \;\;\; \mbox{якщо} \;\;\; \log_{a}b=2.
$$
Перейдемо всюди до основи $a$:
\begin{multline*}
\dfrac{\log_{a}\dfrac{\sqrt{b}}{a}}{\log_{a}(ab)}+\dfrac{\log_{a}b}{\log_{a}\sqrt{ab}}+\log_{a}b^{\frac{1}{2}}=\\
=\dfrac{\log_{a}\sqrt{b}-\log_{a}a}{\log_{a}a+\log_{a}b}+\dfrac{\log_{a}b}{\log_{a}(ab)^{\frac{1}{3}}}+\dfrac{1}{3}\log_{a}b=\\
=\dfrac{\dfrac{1}{2}\log_{a}b-1}{1+\log_{a}b}+\dfrac{\log_{a}b}{\dfrac{1}{2}\left(\log_{a}a+\log_{a}b\right)}+\dfrac{1}{3}\log_{a}b=\\
=\dfrac{\dfrac{1}{2}\log_{a}b-1}{1+\log_{a}b}+\dfrac{\log_{a}b}{\dfrac{1}{2}+\dfrac{1}{2}\log_{a}b}+\dfrac{1}{3}\log_{a}b=\\
=\dfrac{\dfrac{1}{2}\cdot2-1}{1+2}+\dfrac{2}{\dfrac{1}{2}+\dfrac{1}{2}\cdot2}+\dfrac{1}{3}\cdot2=\\
=0+\dfrac{2}{\dfrac{2}{3}}+\dfrac{2}{3}=\dfrac{4}{3}+\dfrac{2}{3}=\dfrac{6}{3}=2.
\end{multline*}
\textbf{8.} Обчислити:
$$
\log_{\sqrt{a}}b\sqrt[4]{a}+\log_{\sqrt{b}}a+\log_{a}\sqrt{ab}, \;\;\; \mbox{якщо} \;\;\; \log_{a}b=2.
$$
Перейдемо всюди до основи $a$:
\begin{multline*}
\log_{\sqrt{a}}b\sqrt[4]{a}+\log_{\sqrt{b}}a+\log_{a}\sqrt{ab}=\\
=\dfrac{\log_{a}b\sqrt[4]{a}}{\log_{a}\sqrt{a}}+\dfrac{\log_{a}a}{\log_{a}\sqrt{b}}+\log_{a}(ab)^{\frac{1}{2}}=\\
=\dfrac{\log_{a}b+\log_{a}\sqrt[4]{a}}{\log_{a}\sqrt{a}}+\dfrac{1}{\log_{a}\sqrt{b}}+\dfrac{1}{2}\log_{a}(ab)=\\
=\dfrac{\log_{a}b+\dfrac{1}{4}\log_{a}a}{\dfrac{1}{2}\log_{a}a}+\dfrac{1}{\dfrac{1}{2}\log_{a}b}+\dfrac{1}{2}\left(\log_{a}a+\log_{a}b\right)=\\
=\dfrac{2+\dfrac{1}{4}}{\dfrac{1}{2}}+\dfrac{1}{\dfrac{1}{2}\cdot2}+\dfrac{1}{2}\left(1+2\right)=\\
=\dfrac{\dfrac{9}{2}}{\dfrac{1}{2}}+1+\dfrac{3}{2}=9+1+1,5=11,5.
\end{multline*}
\textbf{9.} Обчислити:
$$
\log_{\sqrt[3]{a}}\dfrac{b}{a}+\log_{\sqrt{b}}a\sqrt[3]{b}, \;\;\; \mbox{якщо} \;\;\; \log_{b}a=9;
$$
Перейдемо всюди до основи $b$:
\begin{multline*}
\log_{\sqrt[3]{a}}\dfrac{b}{a}+\log_{\sqrt{b}}a\sqrt[3]{b}=
\dfrac{\log_{b}\dfrac{b}{a}}{\log_{b}\sqrt[3]{a}}+\dfrac{\log_{b}a\sqrt[3]{b}}{\log_{b}\sqrt{b}}=\\
=\dfrac{\log_{b}b-\log_{b}a}{\dfrac{1}{3}\log_{b}a}+\dfrac{\log_{b}a+\log_{b}\sqrt[3]{b}}{\dfrac{1}{2}\log_{b}b}=
\dfrac{1-\log_{b}a}{\dfrac{1}{3}\log_{b}a}+\dfrac{\log_{b}a+\dfrac{1}{3}\log_{b}b}{\dfrac{1}{2}}=\\
=\dfrac{1-9}{\dfrac{1}{3}\cdot9}+\dfrac{9+\dfrac{1}{3}}{\dfrac{1}{2}}=
-\dfrac{8}{3}+\dfrac{\dfrac{28}{3}}{\dfrac{1}{2}}=-\dfrac{8}{3}+\dfrac{56}{3}=
\dfrac{48}{3}=16.
\end{multline*}
\textbf{10.} Обчислити:
$$
\left(\log_3 2 + \log_2 81 + 4\right)\left(\log_3 2 - 2\log_{18} 2\right)\log_2 3 - \log_3 2;
$$
Перейдемо всюди до основи $3$:
\begin{multline*}
\left(\log_3 2 + \log_2 81 + 4\right)\left(\log_3 2 - 2\log_{18} 2\right)\log_2 3 - \log_3 2 =\\
= \left(\log_3 2 + \frac{\log_3 3^4}{\log_3 2} + 4\right)\left(\log_2 3 - \frac{2\log_3 2}{\log_3 (2 \cdot 3^2)}\right)\frac{1}{\log_3 2} - \log_3 2 =\\
= \frac{\log_3^2 2 + 4 + 4\log_3 2}{\log_3 2} \cdot \left(\log_3 2 - \frac{2\log_3 2}{\log_3 2 + 2}\right) \cdot \frac{1}{\log_3 2} - \log_3 2 =\\
= \frac{\left(\log_3 2 + 2\right)^{\cancel{2}}}{\cancel{\log_3 2}} \cdot \frac{\cancel{\log_3^2 2} + \cancel{2\log_3 2} - \cancel{2\log_3 2}}{\cancel{\log_3 2 + 2}} \cdot \frac{1}{\cancel{\log_3 2}} - \log_3 2 =\\
= \cancel{\log_3 2} + 2 - \cancel{\log_3 2} = 2;
\end{multline*}
\textbf{11.} Обчислити:
$$
\left(\log_{5}2+\log_{2}5+2\right)\left(\log_{5}2-\lg2\right)\log_{2}5-\log_{5}2;
$$
Перейдемо всюди до основ $5$ і візьмемо до уваги, що:
$$
\lg2=\log_{10}2=\dfrac{\log_{5}2}{\log_{5}(2\cdot5)}=
\dfrac{\log_{5}2}{\log_{5}2+\log_{5}5}=
\dfrac{\log_{5}2}{\log_{5}2+1};
$$
\begin{gather*}
\left(\log_{5}2+\dfrac{1}{\log_{5}2}+2\right)\left(\log_{5}2-\dfrac{\log_{5}2}{\log_{5}2+1}\right)\dfrac{1}{\log_{5}2}-\log_{5}2;
\end{gather*}
Введемо заміну: $\log_{5}2=x$. Отримаємо:
\begin{multline*}
\left(x+\dfrac{1}{x}+2\right)\left(x-\dfrac{x}{x+1}\right)\dfrac{1}{x}-x=
\dfrac{x^2+2x+1}{x}\cdot\dfrac{x^2+\cancel{x}-\cancel{x}}{x+1}\cdot\dfrac{1}{x}-x=\\
=\dfrac{(x+1)^{\cancel{2}}}{\cancel{x}}\cdot\dfrac{\cancel{x^2}}{\cancel{x+1}}\cdot\dfrac{1}{\cancel{x}}-x=
x+1-x=1.
\end{multline*}
\textbf{12.} Обчислити:
$$
\left(\log_{2}7+\log_{7}16+4\right)\left(\log_{2}7-2\log_{28}7\right)\log_{7}2-\log_{2}7;
$$
Перейдемо всюди до основ $2$ і візьмемо до уваги, що:
$$
\log_{7}16=\dfrac{\log_{2}16}{\log_{2}7}=\dfrac{4}{\log_{2}7};
$$
і
$$
\log_{28}7=\dfrac{\log_{2}7}{\log_{2}(4\cdot7)}=
\dfrac{\log_{2}7}{\log_{2}4+\log_{2}7}=
\dfrac{\log_{2}7}{2+\log_{2}7};
$$
\begin{gather*}
\left(\log_{2}7+\dfrac{4}{\log_{2}7}+4\right)\left(\log_{2}7-\dfrac{2\log_{2}7}{2+\log_{2}7}\right)\dfrac{1}{\log_{2}7}-\log_{2}7;
\end{gather*}
Введемо заміну: $\log_{2}7=x$. Отримаємо:
\begin{multline*}
\left(x+\dfrac{4}{x}+4\right)\left(x-\dfrac{2x}{2+x}\right)\dfrac{1}{x}-x=
\dfrac{x^2+4x+2}{x}\cdot\dfrac{\cancel{2x}+x^2-\cancel{2x}}{2+x}\cdot\dfrac{1}{x}-x=\\
=\dfrac{(x+2)^{\cancel{2}}}{\cancel{x}}\cdot\dfrac{\cancel{x^2}}{\cancel{2+x}}\cdot\dfrac{1}{\cancel{x}}-x=
x+2-x=2.
\end{multline*}
\textbf{13.} Обчислити:
$$
27^{\log_{\sqrt{3}}\sqrt[6]{3}}+4\cdot5^{\log^{2}_{5}2}-2^{\log_{5}2}\cdot\log_{2}16;
$$
Візьмемо до уваги, що:
$$
\log_{\sqrt{3}}\sqrt[6]{3}=
\dfrac{\log_{3}\sqrt[6]{3}}{\log_{3}\sqrt{3}}=
\dfrac{\dfrac{1}{6}\log_{3}3}{\dfrac{1}{2}\log_{3}3}=
\dfrac{\dfrac{1}{6}}{\dfrac{1}{2}}=\dfrac{1}{3};
$$
Підставимо це в основний вираз і отримаємо:
\begin{multline*}
3^{3\cdot\frac{1}{3}}+4\cdot5^{(\log_{5}2)(\log_{5}2)}-4\cdot2^{\log_{5}2}=
3+\cancel{4\cdot2^{\log_{5}2}}-\cancel{4\cdot2^{\log_{5}2}}=3.
\end{multline*}
\textbf{14.} Обчислити:
$$
\left(\dfrac{1}{4}\right)^{\log_{0,5}3}\cdot7^{\log^{2}_{7}2}-9\cdot2^{\log_{7}2}+3^{\log_{9}4};
$$
Візьмемо до уваги, що:
$$
3^{\log_{9}4}=3^{\frac{\log_{3}4}{\log_{3}9}}=
3^{\frac{\log_{3}4}{2}}=
3^{\frac{1}{2}\log_{3}4}=
3^{\log_{3}4^{\frac{1}{2}}}=2.
$$
Підставимо це в основний вираз і отримаємо:
\begin{multline*}
\left(\dfrac{1}{2}\right)^{2\cdot\log_{\frac{1}{2}}3}\cdot\left(7^{\log_{7}2}\right)^{\log_{7}2}-9\cdot2^{\log_{7}2}+2=
\cancel{9\cdot2^{\log_{7}2}}-\cancel{9\cdot2^{\log_{7}2}}+2=2.
\end{multline*}