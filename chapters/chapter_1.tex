\section*{Група 1}
\textbf{1.} Обчислити:
$$
\frac{3+\sqrt{5}}{3-\sqrt{5}} - \frac{3\sqrt{5}}{2};
$$

Позбудемося радикала в знаменнику домноживши на спряжений:
\begin{multline*}
\frac{3+\sqrt{5}}{3-\sqrt{5}} - \frac{3\sqrt{5}}{2} = 
\frac{3+\sqrt{5}}{3-\sqrt{5}} \;\cdot\; \frac{3+\sqrt{5}}{3+\sqrt{5}} - \frac{3\sqrt{5}}{2} =\\
= \frac{\left(3+\sqrt{5}\right)\cdot\left(3+\sqrt{5}\right)}{\left(3-\sqrt{5}\right)\cdot\left(3+\sqrt{5}\right)} - \frac{3\sqrt{5}}{2} = 
\frac{\left(3+\sqrt{5}\right)^2}{3^2-\left(\sqrt{5}\right)^2} - \frac{3\sqrt{5}}{2} =\\
= \frac{9 + 6\sqrt{5} + 5}{9 - 5} - \frac{3\sqrt{5}}{2} =
\frac{14+6\sqrt{5}}{4} - \frac{3\sqrt{5}}{2} =\\
= \frac{14+\cancel{6\sqrt{5}}-\cancel{6\sqrt{5}}}{4} =
\frac{14}{4} = 3,5.
\end{multline*}
\textbf{2.} Обчислити:
$$
\frac{\sqrt{10} + \sqrt{6}}{\sqrt{10} - \sqrt{6}} - \sqrt{15};
$$

Позбудемося радикала в знаменнику домноживши на спряжений:
\begin{multline*}
\frac{\sqrt{10} + \sqrt{6}}{\sqrt{10} - \sqrt{6}} - \sqrt{15} =
\frac{\sqrt{10} + \sqrt{6}}{\sqrt{10} - \sqrt{6}} \cdot \frac{\sqrt{10} + \sqrt{6}}{\sqrt{10} + \sqrt{6}} - \sqrt{15} =\\
= \frac{\left(\sqrt{10} + \sqrt{6}\right)\cdot\left(\sqrt{10} + \sqrt{6}\right)}{\left(\sqrt{10} - \sqrt{6}\right)\cdot\left(\sqrt{10} + \sqrt{6}\right)} - \sqrt{15} =
\frac{10 + 2\sqrt{60} + 6}{10 - 6} - \sqrt{15} =\\
= \frac{16 + 2\sqrt{4\cdot15}}{4} - \sqrt{15} =
\frac{16 + 2 \cdot 2 \sqrt{15}}{4} - \sqrt{15} =\\
= \frac{16 + 4\sqrt{15}}{4} - \sqrt{15} =
\frac{16 + \cancel{4\sqrt{15}} - \cancel{4\sqrt{15}}}{4} = \frac{16}{4} = 4.
\end{multline*}
\textbf{3.} Обчислити:
$$
\left(\sqrt{\left(0,5 - \sqrt{2}\right)^2} - \sqrt[3]{\left(1 + \sqrt{2}\right)^3}\;\right)^2;
$$

Оскільки \hfill перший \hfill доданок \hfill це \hfill модуль, \hfill то \hfill потрібно \hfill поміняти \break
місцями члени в цьому доданку.
\begin{multline*}
\left(\sqrt{\left(0,5 - \sqrt{2}\right)^2} - \sqrt[3]{\left(1 + \sqrt{2}\right)^3}\;\right)^2 =\\
= \left(\left|0,5 - \sqrt{2}\right| - \left(1 + \sqrt{2}\right)\right)^2 =
\left(\left|\sqrt{2} - 0,5\right| - \left(1 + \sqrt{2}\right)\right)^2 =\\
= \left(\sqrt{2} - 0,5 - 1 - \sqrt{2}\;\right)^2 =
\left(\cancel{\sqrt{2}} - 0,5 - 1 - \cancel{\sqrt{2}}\;\right)^2 =\\
= (-1,5)^2 = 2,25.
\end{multline*}
\textbf{4.} Обчислити:
$$
8 \cdot \left(\sqrt{\left(\sqrt{5} - 2,5\right)^2} - \sqrt[3]{\left(1 - \sqrt{5}\right)^3}\;\right)^2;
$$

Оскільки \hfill перший \hfill доданок \hfill це \hfill модуль, \hfill то \hfill потрібно \hfill поміняти \break
місцями члени в цьому доданку.
\begin{multline*}
8 \cdot \left(\sqrt{\left(\sqrt{5} - 2,5\right)^2} - \sqrt[3]{\left(1 - \sqrt{5}\right)^3}\;\right)^2 =\\
= 8 \cdot \left(\left|\sqrt{5} - 2,5\right| - \left(1 - \sqrt{5}\right)\right)^2 =
8 \cdot \left(2,5 - \cancel{\sqrt{5}} - 1 + \cancel{\sqrt{5}}\right)^2 =\\
= 8 \cdot (1,5)^2 = 18.
\end{multline*}
\textbf{5.} Обчислити:
$$
\left(\sqrt{\left(0,5 - \sqrt{2}\right)} - \sqrt[3]{\left(-1 + \sqrt{2}\right)}\;\right)^3;
$$

Оскільки \hfill перший \hfill доданок \hfill це \hfill модуль, \hfill то \hfill потрібно \hfill поміняти \break
місцями члени в цьому доданку.
\begin{multline*}
\left(\sqrt{\left(0,5 - \sqrt{2}\right)} - \sqrt[3]{\left(-1 + \sqrt{2}\right)}\;\right)^3 =\\
= \left(\left|0,5 - \sqrt{2}\right| - \left(-1 + \sqrt{2}\right)\right)^3 =
\left(\cancel{\sqrt{2}} - 0,5 +1 - \cancel{\sqrt{2}}\right)^3 =\\
= 0,5^3 = 0,125.
\end{multline*}
\textbf{6.} Обчислити:
$$
1 + \frac{1 + 2^\frac{1}{2}}{3 + 2^\frac{1}{2}} : \frac{1}{2^\frac{3}{2} - 1};
$$

Потрібно використати різницю кубів.
\begin{multline*}
1 + \frac{1 + 2^\frac{1}{2}}{3 + 2^\frac{1}{2}} : \frac{1}{2^\frac{3}{2} - 1} = 
1 + \frac{1 + 2^\frac{1}{2}}{3 + 2^\frac{1}{2}} \cdot \left(\left(2^\frac{1}{2}\right)^3 - 1^3\right) =\\
= 1 + \frac{1 + 2^\frac{1}{2}}{3 + 2^\frac{1}{2}} \cdot \left(2^\frac{1}{2} - 1\right) \cdot \left(2 + 2^\frac{1}{2} + 1\right) =\\
= 1 + \frac{1 + 2^\frac{1}{2}}{\cancel{3 + 2^\frac{1}{2}}} \cdot \left(2^\frac{1}{2} - 1\right) \cdot \cancel{\left(3 + 2^\frac{1}{2}\right)} =
1 + \left(\sqrt{2} - 1\right) \cdot \left(\sqrt{2} + 1\right) =\\
= 1 + (2 - 1) = 1 + 1 = 2;
\end{multline*}
\textbf{7.} обчислити:
$$
1 + \frac{1 + 3^{1/2}}{4 + 3^{1/2}} : \frac{1}{3^{3/2} - 1};
$$

Потрібно використати різницю кубів.
\begin{multline*}
1 + \frac{1 + 3^{1/2}}{4 + 3^{1/2}} : \frac{1}{3^{3/2} - 1} =\\
= 1 + \frac{1 + 3^{1/2}}{\cancel{4 + 3^{1/2}}} \cdot \left(3^{1/2} - 1\right) \cancel{\left(3 + 3^{1/2} + 1\right)} =\\
= 1 + \left(1 + 3^{1/2}\right)\left(3^{1/2} - 1\right) = 1 + 3 - 1 = 3.
\end{multline*}
\textbf{8.} Обчислити:
$$
1 + \frac{1 + 30^{1/2}}{31 + 30^{1/2}} : \frac{1}{30^{3/2} - 1};
$$

Потрібно використати різницю кубів.
\begin{multline*}
1 + \frac{1 + 30^{1/2}}{31 + 30^{1/2}} : \frac{1}{30^{3/2} - 1} =\\
= 1 + \frac{1 + 30^{1/2}}{31 + 30^{1/2}} \cdot \left(30^{1/2} - 1\right)\left(30 + 30^{1/2} + 1\right) =\\
= 1 + \left(30^{1/2} + 1\right)\left(30^{1/2} - 1\right) = 1 + 30 - 1 = 30;
\end{multline*}
\textbf{9.} При якому значенні параметра $a$ вираз $25x^2 + 30x + a$ можна записати у вигляді повного квадрата суми двох одночленів?
$$
\left(mx + n\right)^2 = m^2x^2 + 2mnx + n^2;
$$
Звідси ми можемо записати:
$$
\left\{
\begin{aligned}
m^2 &= 25,\\
2mn &= 30,\\
a &= n^2;
\end{aligned}
\right.
\Leftrightarrow
\left\{
\begin{aligned}
m &= \pm 5,\\
mn &= 15,\\
a &= n^2;
\end{aligned}
\right.
\Leftrightarrow
\left\{
\begin{aligned}
m &= \pm 5,\\
n &= \pm 3,\\
a &= 9.
\end{aligned}
\right.
$$
\textbf{10.} При якому значенні параметра $a$ вираз $36x^2 - 12x + a$ можна записати у вигляді
повного квадрата різниці двох одночленів?
$$
\left(mx - n\right)^2 = m^2x^2 - 2mnx + n^2;
$$
Звідси ми можемо записати:
$$
\left\{
\begin{aligned}
m^2 &= 36,\\
2mn &= 12,\\
a &= n^2;
\end{aligned}
\right.
\Leftrightarrow
\left\{
\begin{aligned}
m &= \pm 6,\\
mn &= 6,\\
a &= n^2;
\end{aligned}
\right.
\Leftrightarrow
\left\{
\begin{aligned}
m &= \pm 6,\\
n &= \pm 1,\\
a &= 1.
\end{aligned}
\right.
$$
\textbf{11.} Дано вираз:
$$
\left(\frac{1}{\sqrt{a} + \sqrt{a + 1}} + \frac{1}{\sqrt{a} - \sqrt{a - 1}}\right) \cdot \frac{\sqrt{a - 1}}{\sqrt{a + 1} + \sqrt{a - 1}}.
$$
При якому значенні $a$ цей вираз набуває найменшого значення?

Проаналізуємо всі члени даного виразу. Ми можемо побачити, що перший, за порядком, член є завжди додатний, оскільки знаменник може бути тільки додатним при будь-якому допустимому $a$.
\newline
Розглянемо другий член. Так як
$$
\sqrt{a} - \sqrt{a - 1} > 0
$$
при \hfill будь-якому \hfill допустимому \hfill $a$, \hfill то \hfill другий \hfill член \hfill також \hfill завжди 
\newline
додатний. Сума додатних є додатна. Таким чином, вираз у дужках додатний при будь-якому допустимому $a$.

Очевидно, що останній, за порядком, вираз також додатний при будь-якому допустимому $a$.

Ми довели, що весь наш вираз є невід'ємний при будь-якому допустимому $a$. Тому, при $a = 1$, вираз
набуватиме найменшого значення -- $0$.
\vspace{5mm}
\newline
\textbf{12.} Дано вираз:
$$
\frac{\sqrt[4]{\left(x + 1,1\right)^3} - 1,2^3}{\sqrt[4]{x + 1,1} - 1,2} - 1,2\sqrt[4]{x + 1,1}.
$$
Яке його найменше значення?
\newline
Перш за все спростимо вираз.
\begin{flalign*}
&\frac{\sqrt[4]{\left(x + 1,1\right)^3} - 1,2^3}{\sqrt[4]{x + 1,1} - 1,2} - 1,2\sqrt[4]{x + 1,1} =\\
&= \frac{\cancel{\left(\sqrt[4]{x + 1,1} - 1,2\right)}\left(\sqrt{x + 1,1} + 1,2\sqrt[4]{x + 1,1} + 1,44\right)}{\cancel{\sqrt[4]{x + 1,1} - 1,2}} -\\
&- 1,2\sqrt[4]{x + 1,1} = \sqrt{x + 1,1} + \cancel{1,2\sqrt[4]{x + 1,1}} + 1,44 -\\
&- \cancel{1,2\sqrt[4]{x + 1,1}} = \sqrt{x + 1,1} + 1,44.
\end{flalign*}
Очевидно, що найменше значення вираз набуватиме при $x = -1,1$ і буде рівним $1,44$.
\vspace{5mm}
\newline
\textbf{13.} Дано вираз:
$$
\frac{\sqrt[4]{\left(x - 2\right)^3} - 4^3}{16 - 4\sqrt[4]{x - 2}} + \frac{\sqrt{x - 2}}{4}.
$$
Яке його найбільше значення?
\newline
Перш за все спростимо вираз.
\begin{flalign*}
&\frac{\sqrt[4]{\left(x - 2\right)^3} - 4^3}{16 - 4\sqrt[4]{x - 2}} + \frac{\sqrt{x - 2}}{4} =\\
&= \frac{\cancel{\left(\sqrt[4]{x - 2} - 4\right)}\left(\sqrt{x - 2} - 4\sqrt[4]{x - 2} - 16\right)}{-4\cancel{\left(\sqrt[4]{x - 2} - 4\right)}} + \frac{\sqrt{x - 2}}{4} =\\
&= \frac{\cancel{-\sqrt{x - 2}} -4\sqrt[4]{x - 2} - 16 + \cancel{\sqrt{x - 2}}}{4} =\\
&= -4\frac{\sqrt[4]{x - 2}}{4} = -\left(\sqrt[4]{x - 2} + 4\right).
\end{flalign*}
Вираз набуває найбільшого значення при $x = 2$ і рівне $-4$.
\vspace{5mm}
\newline
\textbf{14.} Обчислити:
$$
\frac{x^5 + 64x^{-1}}{x^3 - 4x + 16x^{-1}} : \frac{x^2 + 4}{2}.
$$
\begin{flalign*}
&\frac{x^5 + 64x^{-1}}{x^3 - 4x + 16x^{-1}} : \frac{x^2 + 4}{2} =
\frac{x^5 + \frac{64}{x}}{x^3 - 4x + \frac{16}{x}} \cdot \frac{2}{x^2 + 4} =\\
&= \frac{\frac{x^6 + 64}{x}}{\frac{x^4 - 4x^2 + 16}{x}} \cdot \frac{2}{x^2 - 4} =
\frac{x^6 + 64}{x^4 - 4x^2 + 16} \cdot \frac{2}{x^2 + 4} =\\
&= \frac{\cancel{\left(x^2\right)^3 + 4^3}}{\cancel{x^4 - 4x^2 + 16}} \cdot \frac{2}{\cancel{x^2 + 4}} = 2.
\end{flalign*}

\section*{Група 2}
\textbf{1.} Обчислити:
$$
\sqrt{27 + 10\sqrt{2}} + \sqrt{27 - 10\sqrt{2}}.
$$
Цю задачу можна розв'язати двома способами. Перший спосіб є менш очевидний і ґрунтується на такій рівності:
$$
27 \pm 10\sqrt{2} = 25 \pm 10\sqrt{2} + 2 = 5^2 \pm 2 \cdot 5 \cdot \sqrt{2} + \left(\sqrt{2}\right)^2 = \left(5 \pm \sqrt{2}\right)^2.
$$
Другий спосіб розглянемо детально.
\\
Введемо таку заміну:
$$
a = \sqrt{27 + 10\sqrt{2}} \;\;\; \mbox{і} \;\;\; b = \sqrt{27 - 10\sqrt{2}};
$$
Нам потрібно знайти:
$$
s = a + b.
$$
$$
s^2 = (a + b)^2 = a^2 + 2ab + b^2;
$$
\begin{flalign*}
a^2 = 27 + 10\sqrt{2};\;\;
b^2 = 27 - 10\sqrt{2};\;\;
ab = \sqrt{729 - 200};
\end{flalign*}
Отже,
$$
s^2 = 27 + \cancel{10\sqrt{2}} + 2 \cdot \sqrt{529} + 27 - \cancel{10\sqrt{2}} =
54 + 2 \cdot 23 = 100;
$$
А звідси $s = \pm 10$.
\\
Оскільки $a \geq 0$  і $b \geq 0$, то їх сума також невід'ємна, а отже $s = -10$ не
є розв'язком рівняння. І залишається $s = 10$.
\vspace{5mm}
\\
\textbf{2.} Обчислити:
$$
\sqrt{29 - 12\sqrt{5}} - \sqrt{20 + 12\sqrt{5}}.
$$




\begin{comment}
\begin{multline*}
{\mathbf 2.}\;
\sqrt{29 - 12\sqrt{5}} - \sqrt{29 + 12\sqrt{5}} =
\sqrt{29 - 2\cdot 3 \cdot \sqrt{5}} - \sqrt{29 + 2\cdot 3 \cdot \sqrt{5}} =\\
= \sqrt{29 - 3\sqrt{4 \cdot 5}} - \sqrt{29 + 3\sqrt{4 \cdot 5}} =
\sqrt{29 - 3\sqrt{20}} - \sqrt{29 + 3\sqrt{20}} =\\
= \sqrt{\left(\sqrt{20}\right)^2 - 3\sqrt{20} + 3^2} - \sqrt{\left(\sqrt{20}\right)^2 + 3\sqrt{20} + 3^2} =\\
= \sqrt{\left(\sqrt{20} - 3\right)^2} - \sqrt{\left(\sqrt{20} + 3\right)^2} =
\left|\sqrt{20} - 3\right| - \left|\sqrt{20} + 3\right| =\\
= \cancel{\sqrt{20}} - 3 - \cancel{\sqrt{20}} - 3 = -6;
\end{multline*}

\begin{multline*}
{\mathbf 4.}\;
\left(\sqrt[6]{3 + 2\sqrt{2}} + \sqrt[3]{1 + \sqrt{2}}\right) \cdot \sqrt[3]{1 - \sqrt{2}} =\\
= \left(\sqrt[6]{\left(\sqrt{2}\right)^2 + 2\sqrt{2} + 1^2} + \sqrt[3]{1 + \sqrt{2}}\right) \cdot \sqrt[3]{1 - \sqrt{2}} =\\
= \left(\sqrt[6]{\left(\sqrt{2} + 1\right)^2} + \sqrt[3]{1 + \sqrt{2}}\right) \cdot \sqrt[3]{1 - \sqrt{2}} =\\
= \left(\sqrt[3]{\sqrt{2} + 1} + \sqrt[3]{1 + \sqrt{2}}\right) \cdot \sqrt[3]{1 - \sqrt{2}} =\\
= 2\sqrt[3]{1 + \sqrt{2}} \cdot \sqrt[3]{1 - \sqrt{2}} =
2\sqrt[3]{\left(1 + \sqrt{2}\right) \cdot \left(1 - \sqrt{2}\right)} =\\
= 2 \cdot \sqrt[3]{1 - 2} = 2 \cdot (-1) = -2;
\end{multline*}

\textbf{5.}
$$
2 \pm \sqrt{5} = \left(\frac{1 \pm \sqrt{5}}{2}\right)^3;
$$
або
$$
s = a + b; a = \sqrt[3]{2 + \sqrt{5}}, b = \sqrt[3]{2 - \sqrt{5}};
$$

\begin{multline*}
{\mathbf 7.}\;
\frac{\sqrt{8 - 2\sqrt{15}}}{\sqrt{3} - \sqrt{5}} =
\frac{\sqrt{\left(\sqrt{3}\right)^2 - 2\sqrt{3}\sqrt{5} + \left(\sqrt{5}\right)^2}}{\sqrt{3} - \sqrt{5}} =
\frac{\sqrt{\left(\sqrt{3} - \sqrt{5}\right)^2}}{\sqrt{3} - \sqrt{5}} =\\
= \frac{\left|\sqrt{3} - \sqrt{5}\;\right|}{-\left(\sqrt{5} - \sqrt{3}\right)} =
\frac{\sqrt{5} - \sqrt{3}}{\sqrt{5} - \sqrt{3}} = -1;
\end{multline*}

\begin{multline*}
{\mathbf 11.}\;
\sqrt{3\left(\sqrt{11} - \sqrt{8}\right)} \cdot \sqrt[4]{19 + 4\sqrt{22}} =\\
= \sqrt{3\left(\sqrt{11} - \sqrt{8}\right)} \cdot \sqrt[4]{19 + 2 \cdot 2 \cdot \sqrt{2 \cdot 11}} =\\
= \sqrt{3\left(\sqrt{11} - \sqrt{8}\right)} \cdot \sqrt[4]{19 + 2 \cdot \sqrt{4 \cdot 2 \cdot 11}} =\\
= \sqrt{3\left(\sqrt{11} - \sqrt{8}\right)} \cdot \sqrt[4]{\left(\sqrt{11}\right)^2 + 2 \cdot \sqrt{8} \cdot \sqrt{11} + \left(\sqrt{8}\right)^2} =\\
= \sqrt{3\left(\sqrt{11} - \sqrt{8}\right)} \cdot \sqrt[4]{\left(\sqrt{11} + \sqrt{8}\right)^2} =\\
= \sqrt{3\left(\sqrt{11} - \sqrt{8}\right) \cdot \left(\sqrt{11} + \sqrt{8}\right)} =
\sqrt{3 \cdot 3} = 3;
\end{multline*}

\begin{multline*}
{\mathbf 15.}\;
\frac{a^2 + 1}{a\sqrt{\left(\frac{a^2 - 1}{2a}\right)^2 + 1}} =
\frac{a^2 + 1}{a\sqrt{\frac{a^4 - 2a^2 + 1}{4a^2} + 1}} =
\frac{a^2 + 1}{a\sqrt{\frac{a^4 - 2a^2 + 1 + 4a^2}{4a^2}}} =\\
\frac{a^2 + 1}{\frac{\cancel{a}}{2\cancel{a}}\sqrt{a^4 + 2a^2 + 1}} =
\frac{a^2 + 1}{\frac{\sqrt{\left(a^2 + 1\right)^2}}{2}} =
2 \cdot \frac{a^2 + 1}{a^2 + 1} = 2;
\end{multline*}

\begin{multline*}
{\mathbf 17.}\;
\frac{\sqrt{x} + 2}{x\sqrt{x} + 2x + 4\sqrt{x}} : \frac{1}{\left(x^2 - 8\sqrt{x}\right)} =
\frac{\left(\sqrt{x} + 2\right) \cdot \left(x^2 - 8\sqrt{x}\right)}{x\sqrt{x} + 2x + 4\sqrt{x}} =\\
= \frac{\left(\sqrt{x} + 2\right) \cdot \sqrt{x} \cdot \left(\sqrt{x^3} - 2^3\right)}{x\sqrt{x} + 2x + 4\sqrt{x}} =
\frac{\left(\sqrt{x} + 2\right) \cdot \sqrt{x} \cdot \left(\sqrt{x} - 2\right)\left(x + 2\sqrt{x} + 4\right)}{x\sqrt{x} + 2x + 4\sqrt{x}} =\\
= \frac{\left(\sqrt{x} + 2\right) \cdot \cancel{\sqrt{x}} \cdot \left(\sqrt{x} - 2\right)\cancel{\left(x + 2\sqrt{x} + 4\right)}}{\cancel{\sqrt{x}}\cancel{\left(x + 2\sqrt{x} + 4\right)}} =
x - 4;
\end{multline*}

\textbf{21.} $\sqrt{4 + \sqrt{x - 2}} + \sqrt{x + 2 + \sqrt{2 - x}}$.

$$
ODZ: 
\begin{cases}
x - 2 \ge 0,\\
2 - x \ge 0;
\end{cases}
\Longleftrightarrow
\begin{cases}
x\ge 2,\\
x \le 2;
\end{cases}
\Longleftrightarrow
x = 2;
$$

$\sqrt{4 + 0} + \sqrt{2 + 2 + 0} = 2 + 2 = 4$;
\end{comment}