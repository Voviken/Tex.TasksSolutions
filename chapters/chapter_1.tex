\section*{Група 1}
\textbf{1.} Обчислити:
$$
\frac{3+\sqrt{5}}{3-\sqrt{5}} - \frac{3\sqrt{5}}{2};
$$

Позбудемося радикала в знаменнику домноживши на спряжений:
\begin{multline*}
\frac{3+\sqrt{5}}{3-\sqrt{5}} - \frac{3\sqrt{5}}{2} = 
\frac{3+\sqrt{5}}{3-\sqrt{5}} \;\cdot\; \frac{3+\sqrt{5}}{3+\sqrt{5}} - \frac{3\sqrt{5}}{2} =\\
= \frac{\left(3+\sqrt{5}\right)\cdot\left(3+\sqrt{5}\right)}{\left(3-\sqrt{5}\right)\cdot\left(3+\sqrt{5}\right)} - \frac{3\sqrt{5}}{2} = 
\frac{\left(3+\sqrt{5}\right)^2}{3^2-\left(\sqrt{5}\right)^2} - \frac{3\sqrt{5}}{2} =\\
= \frac{9 + 6\sqrt{5} + 5}{9 - 5} - \frac{3\sqrt{5}}{2} =
\frac{14+6\sqrt{5}}{4} - \frac{3\sqrt{5}}{2} =\\
= \frac{14+\cancel{6\sqrt{5}}-\cancel{6\sqrt{5}}}{4} =
\frac{14}{4} = 3,5.
\end{multline*}
\textbf{2.} Обчислити:
$$
\frac{\sqrt{10} + \sqrt{6}}{\sqrt{10} - \sqrt{6}} - \sqrt{15};
$$

Позбудемося радикала в знаменнику домноживши на спряжений:
\begin{multline*}
\frac{\sqrt{10} + \sqrt{6}}{\sqrt{10} - \sqrt{6}} - \sqrt{15} =
\frac{\sqrt{10} + \sqrt{6}}{\sqrt{10} - \sqrt{6}} \cdot \frac{\sqrt{10} + \sqrt{6}}{\sqrt{10} + \sqrt{6}} - \sqrt{15} =\\
= \frac{\left(\sqrt{10} + \sqrt{6}\right)\cdot\left(\sqrt{10} + \sqrt{6}\right)}{\left(\sqrt{10} - \sqrt{6}\right)\cdot\left(\sqrt{10} + \sqrt{6}\right)} - \sqrt{15} =
\frac{10 + 2\sqrt{60} + 6}{10 - 6} - \sqrt{15} =\\
= \frac{16 + 2\sqrt{4\cdot15}}{4} - \sqrt{15} =
\frac{16 + 2 \cdot 2 \sqrt{15}}{4} - \sqrt{15} =\\
= \frac{16 + 4\sqrt{15}}{4} - \sqrt{15} =
\frac{16 + \cancel{4\sqrt{15}} - \cancel{4\sqrt{15}}}{4} = \frac{16}{4} = 4.
\end{multline*}
\textbf{3.} Обчислити:
$$
\left(\sqrt{\left(0,5 - \sqrt{2}\right)^2} - \sqrt[3]{\left(1 + \sqrt{2}\right)^3}\;\right)^2;
$$

Оскільки \hfill перший \hfill доданок \hfill це \hfill модуль, \hfill то \hfill потрібно \hfill поміняти \break
місцями члени в цьому доданку.
\begin{multline*}
\left(\sqrt{\left(0,5 - \sqrt{2}\right)^2} - \sqrt[3]{\left(1 + \sqrt{2}\right)^3}\;\right)^2 =\\
= \left(\left|0,5 - \sqrt{2}\right| - \left(1 + \sqrt{2}\right)\right)^2 =
\left(\left|\sqrt{2} - 0,5\right| - \left(1 + \sqrt{2}\right)\right)^2 =\\
= \left(\sqrt{2} - 0,5 - 1 - \sqrt{2}\;\right)^2 =
\left(\cancel{\sqrt{2}} - 0,5 - 1 - \cancel{\sqrt{2}}\;\right)^2 =\\
= (-1,5)^2 = 2,25.
\end{multline*}
\textbf{4.} Обчислити:
$$
8 \cdot \left(\sqrt{\left(\sqrt{5} - 2,5\right)^2} - \sqrt[3]{\left(1 - \sqrt{5}\right)^3}\;\right)^2;
$$

Оскільки \hfill перший \hfill доданок \hfill це \hfill модуль, \hfill то \hfill потрібно \hfill поміняти \break
місцями члени в цьому доданку.
\begin{multline*}
8 \cdot \left(\sqrt{\left(\sqrt{5} - 2,5\right)^2} - \sqrt[3]{\left(1 - \sqrt{5}\right)^3}\;\right)^2 =\\
= 8 \cdot \left(\left|\sqrt{5} - 2,5\right| - \left(1 - \sqrt{5}\right)\right)^2 =
8 \cdot \left(2,5 - \cancel{\sqrt{5}} - 1 + \cancel{\sqrt{5}}\right)^2 =\\
= 8 \cdot (1,5)^2 = 18.
\end{multline*}
\textbf{5.} Обчислити:
$$
\left(\sqrt{\left(0,5 - \sqrt{2}\right)} - \sqrt[3]{\left(-1 + \sqrt{2}\right)}\;\right)^3;
$$

Оскільки \hfill перший \hfill доданок \hfill це \hfill модуль, \hfill то \hfill потрібно \hfill поміняти \break
місцями члени в цьому доданку.
\begin{multline*}
\left(\sqrt{\left(0,5 - \sqrt{2}\right)} - \sqrt[3]{\left(-1 + \sqrt{2}\right)}\;\right)^3 =\\
= \left(\left|0,5 - \sqrt{2}\right| - \left(-1 + \sqrt{2}\right)\right)^3 =
\left(\cancel{\sqrt{2}} - 0,5 +1 - \cancel{\sqrt{2}}\right)^3 =\\
= 0,5^3 = 0,125.
\end{multline*}
\textbf{6.} Обчислити:
$$
1 + \frac{1 + 2^\frac{1}{2}}{3 + 2^\frac{1}{2}} : \frac{1}{2^\frac{3}{2} - 1};
$$

Потрібно використати різницю кубів.
\begin{multline*}
1 + \frac{1 + 2^\frac{1}{2}}{3 + 2^\frac{1}{2}} : \frac{1}{2^\frac{3}{2} - 1} = 
1 + \frac{1 + 2^\frac{1}{2}}{3 + 2^\frac{1}{2}} \cdot \left(\left(2^\frac{1}{2}\right)^3 - 1^3\right) =\\
= 1 + \frac{1 + 2^\frac{1}{2}}{3 + 2^\frac{1}{2}} \cdot \left(2^\frac{1}{2} - 1\right) \cdot \left(2 + 2^\frac{1}{2} + 1\right) =\\
= 1 + \frac{1 + 2^\frac{1}{2}}{\cancel{3 + 2^\frac{1}{2}}} \cdot \left(2^\frac{1}{2} - 1\right) \cdot \cancel{\left(3 + 2^\frac{1}{2}\right)} =
1 + \left(\sqrt{2} - 1\right) \cdot \left(\sqrt{2} + 1\right) =\\
= 1 + (2 - 1) = 1 + 1 = 2;
\end{multline*}
\textbf{7.} обчислити:
$$
1 + \frac{1 + 3^{1/2}}{4 + 3^{1/2}} : \frac{1}{3^{3/2} - 1};
$$

Потрібно використати різницю кубів.
\begin{multline*}
1 + \frac{1 + 3^{1/2}}{4 + 3^{1/2}} : \frac{1}{3^{3/2} - 1} =\\
= 1 + \frac{1 + 3^{1/2}}{\cancel{4 + 3^{1/2}}} \cdot \left(3^{1/2} - 1\right) \cancel{\left(3 + 3^{1/2} + 1\right)} =\\
= 1 + \left(1 + 3^{1/2}\right)\left(3^{1/2} - 1\right) = 1 + 3 - 1 = 3.
\end{multline*}
\textbf{8.} Обчислити:
$$
1 + \frac{1 + 30^{1/2}}{31 + 30^{1/2}} : \frac{1}{30^{3/2} - 1};
$$

Потрібно використати різницю кубів.
\begin{multline*}
1 + \frac{1 + 30^{1/2}}{31 + 30^{1/2}} : \frac{1}{30^{3/2} - 1} =\\
= 1 + \frac{1 + 30^{1/2}}{31 + 30^{1/2}} \cdot \left(30^{1/2} - 1\right)\left(30 + 30^{1/2} + 1\right) =\\
= 1 + \left(30^{1/2} + 1\right)\left(30^{1/2} - 1\right) = 1 + 30 - 1 = 30;
\end{multline*}
\textbf{9.} При якому значенні параметра $a$ вираз $25x^2 + 30x + a$ можна записати у вигляді повного квадрата суми двох одночленів?
$$
\left(mx + n\right)^2 = m^2x^2 + 2mnx + n^2;
$$
Звідси ми можемо записати:
$$
\left\{
\begin{aligned}
m^2 &= 25,\\
2mn &= 30,\\
a &= n^2;
\end{aligned}
\right.
\Leftrightarrow
\left\{
\begin{aligned}
m &= \pm 5,\\
mn &= 15,\\
a &= n^2;
\end{aligned}
\right.
\Leftrightarrow
\left\{
\begin{aligned}
m &= \pm 5,\\
n &= \pm 3,\\
a &= 9.
\end{aligned}
\right.
$$
\textbf{10.} При якому значенні параметра $a$ вираз $36x^2 - 12x + a$ можна записати у вигляді
повного квадрата різниці двох одночленів?
$$
\left(mx - n\right)^2 = m^2x^2 - 2mnx + n^2;
$$
Звідси ми можемо записати:
$$
\left\{
\begin{aligned}
m^2 &= 36,\\
2mn &= 12,\\
a &= n^2;
\end{aligned}
\right.
\Leftrightarrow
\left\{
\begin{aligned}
m &= \pm 6,\\
mn &= 6,\\
a &= n^2;
\end{aligned}
\right.
\Leftrightarrow
\left\{
\begin{aligned}
m &= \pm 6,\\
n &= \pm 1,\\
a &= 1.
\end{aligned}
\right.
$$
\textbf{11.} Дано вираз:
$$
\left(\frac{1}{\sqrt{a} + \sqrt{a + 1}} + \frac{1}{\sqrt{a} - \sqrt{a - 1}}\right) \cdot \frac{\sqrt{a - 1}}{\sqrt{a + 1} + \sqrt{a - 1}}.
$$
При якому значенні $a$ цей вираз набуває найменшого значення?

Проаналізуємо всі члени даного виразу. Ми можемо побачити, що перший, за порядком, член є завжди додатний, оскільки знаменник може бути тільки додатним при будь-якому допустимому $a$.
\newline
Розглянемо другий член. Так як
$$
\sqrt{a} - \sqrt{a - 1} > 0
$$
при \hfill будь-якому \hfill допустимому \hfill $a$, \hfill то \hfill другий \hfill член \hfill також \hfill завжди 
\newline
додатний. Сума додатних є додатна. Таким чином, вираз у дужках додатний при будь-якому допустимому $a$.

Очевидно, що останній, за порядком, вираз також додатний при будь-якому допустимому $a$.

Ми довели, що весь наш вираз є невід'ємний при будь-якому допустимому $a$. Тому, при $a = 1$, вираз
набуватиме найменшого значення -- $0$.
\vspace{5mm}
\newline
\textbf{12.} Дано вираз:
$$
\frac{\sqrt[4]{\left(x + 1,1\right)^3} - 1,2^3}{\sqrt[4]{x + 1,1} - 1,2} - 1,2\sqrt[4]{x + 1,1}.
$$
Яке його найменше значення?
\newline
Перш за все спростимо вираз.
\begin{flalign*}
&\frac{\sqrt[4]{\left(x + 1,1\right)^3} - 1,2^3}{\sqrt[4]{x + 1,1} - 1,2} - 1,2\sqrt[4]{x + 1,1} =\\
&= \frac{\cancel{\left(\sqrt[4]{x + 1,1} - 1,2\right)}\left(\sqrt{x + 1,1} + 1,2\sqrt[4]{x + 1,1} + 1,44\right)}{\cancel{\sqrt[4]{x + 1,1} - 1,2}} -\\
&- 1,2\sqrt[4]{x + 1,1} = \sqrt{x + 1,1} + \cancel{1,2\sqrt[4]{x + 1,1}} + 1,44 -\\
&- \cancel{1,2\sqrt[4]{x + 1,1}} = \sqrt{x + 1,1} + 1,44.
\end{flalign*}
Очевидно, що найменше значення вираз набуватиме при $x = -1,1$ і буде рівним $1,44$.
\vspace{5mm}
\newline
\textbf{13.} Дано вираз:
$$
\frac{\sqrt[4]{\left(x - 2\right)^3} - 4^3}{16 - 4\sqrt[4]{x - 2}} + \frac{\sqrt{x - 2}}{4}.
$$
Яке його найбільше значення?
\newline
Перш за все спростимо вираз.
\begin{flalign*}
&\frac{\sqrt[4]{\left(x - 2\right)^3} - 4^3}{16 - 4\sqrt[4]{x - 2}} + \frac{\sqrt{x - 2}}{4} =\\
&= \frac{\cancel{\left(\sqrt[4]{x - 2} - 4\right)}\left(\sqrt{x - 2} - 4\sqrt[4]{x - 2} - 16\right)}{-4\cancel{\left(\sqrt[4]{x - 2} - 4\right)}} + \frac{\sqrt{x - 2}}{4} =\\
&= \frac{\cancel{-\sqrt{x - 2}} -4\sqrt[4]{x - 2} - 16 + \cancel{\sqrt{x - 2}}}{4} =\\
&= -4\frac{\sqrt[4]{x - 2}}{4} = -\left(\sqrt[4]{x - 2} + 4\right).
\end{flalign*}
Вираз набуває найбільшого значення при $x = 2$ і рівне $-4$.
\vspace{5mm}
\newline
\textbf{14.} Обчислити:
$$
\frac{x^5 + 64x^{-1}}{x^3 - 4x + 16x^{-1}} : \frac{x^2 + 4}{2}.
$$
\begin{flalign*}
&\frac{x^5 + 64x^{-1}}{x^3 - 4x + 16x^{-1}} : \frac{x^2 + 4}{2} =
\frac{x^5 + \frac{64}{x}}{x^3 - 4x + \frac{16}{x}} \cdot \frac{2}{x^2 + 4} =\\
&= \frac{\frac{x^6 + 64}{x}}{\frac{x^4 - 4x^2 + 16}{x}} \cdot \frac{2}{x^2 - 4} =
\frac{x^6 + 64}{x^4 - 4x^2 + 16} \cdot \frac{2}{x^2 + 4} =\\
&= \frac{\cancel{\left(x^2\right)^3 + 4^3}}{\cancel{x^4 - 4x^2 + 16}} \cdot \frac{2}{\cancel{x^2 + 4}} = 2.
\end{flalign*}

\section*{Група 2}
\textbf{1.} Обчислити:
$$
\sqrt{27 + 10\sqrt{2}} + \sqrt{27 - 10\sqrt{2}}.
$$
Цю задачу можна розв'язати двома способами. Перший спосіб є менш очевидний і ґрунтується на такій рівності:
$$
27 \pm 10\sqrt{2} = 25 \pm 10\sqrt{2} + 2 = 5^2 \pm 2 \cdot 5 \cdot \sqrt{2} + \left(\sqrt{2}\right)^2 = \left(5 \pm \sqrt{2}\right)^2.
$$
Другий спосіб розглянемо детально.
\\
Введемо таку заміну:
$$
a = \sqrt{27 + 10\sqrt{2}} \;\;\; \mbox{і} \;\;\; b = \sqrt{27 - 10\sqrt{2}};
$$
Нам потрібно знайти:
$$
s = a + b.
$$
$$
s^2 = (a + b)^2 = a^2 + 2ab + b^2;
$$
\begin{flalign*}
a^2 = 27 + 10\sqrt{2};\;\;
b^2 = 27 - 10\sqrt{2};\;\;
ab = \sqrt{729 - 200};
\end{flalign*}
Отже,
$$
s^2 = 27 + \cancel{10\sqrt{2}} + 2 \cdot \sqrt{529} + 27 - \cancel{10\sqrt{2}} =
54 + 2 \cdot 23 = 100;
$$
А звідси $s = \pm 10$.
\\
Оскільки $a \geq 0$  і $b \geq 0$, то їх сума також невід'ємна, а отже $s = -10$ не
є розв'язком рівняння. І залишається $s = 10$.
\vspace{5mm}
\\
\textbf{2.} Обчислити:
$$
\sqrt{29 - 12\sqrt{5}} - \sqrt{20 + 12\sqrt{5}}.
$$
Менш очевидний спосіб розв'язання базується на рівності:
$$
29 \pm 12\sqrt{5} =
20 \pm 2 \cdot 3 \cdot \sqrt{4 \cdot 5} + 9 =
\left(\sqrt{20}\right)^2 \pm 2 \cdot 3 \cdot \sqrt{20} + 3^2 =
\left(\sqrt{20} \pm 3\right)^2.
$$
Також, цю задачу можна розв'язати ввівши таку заміну:
$$
a = \sqrt{29 - 12\sqrt{5}} \;\;\; \mbox{і} \;\;\; b = \sqrt{29 + 12\sqrt{5}}
$$
Таким чином, нам потрібно знайти:
$$
s = a - b;
$$
$$
s^2 = (a - b)^2 = a^2 - 2ab + b^2;
$$
\begin{flalign*}
a^2 = 29 - 12\sqrt{5};\;\;
b^2 = 29 - 12\sqrt{5};\;\;
ab = \sqrt{841 - 720} = 11;
\end{flalign*}
Отже,
$$
s^2 = 29 - \cancel{12\sqrt{5}} - 2 \cdot 11 + 29 + \cancel{12\sqrt{5}} = 36;
$$
А звідси $s = \pm 6$.
\\
Очевидно, що 
$$\sqrt{29 - 12\sqrt{5}} < \sqrt{29 + 12\sqrt{5}},$$
а отже і 
$$\sqrt{29 - 12\sqrt{5}} - \sqrt{29 + 12\sqrt{5}} < 0.$$
Значить, за розв'язок задачі беремо $s = -6$.
\vspace{5mm}
\\
\textbf{3.} Обчислити:
$$
\sqrt{10}\left(\sqrt{6 - \sqrt{35}} - \sqrt{6 + \sqrt{35}}\right).
$$
Перш за все розкриємо дужки:
$$
\sqrt{10}\left(\sqrt{6 - \sqrt{35}} - \sqrt{6 + \sqrt{35}}\right) =
\sqrt{60 - 10\sqrt{35}} - \sqrt{60 + 10\sqrt{35}}.
$$
Менш очевидний спосіб розв'язання базується на рівності:
$$
60 \pm 10\sqrt{35} =
25 \pm 2 \cdot 5 \cdot \sqrt{35} + 35 =
5^2 \pm 2 \cdot 5 \cdot \sqrt{35} + \left(\sqrt{35}\right)^2 =
\left(5 \pm \sqrt{35}\right)^2.
$$
Також, цю задачу можна розв'язати ввівши таку заміну:
$$
a = \sqrt{60 - 10\sqrt{35}} \;\;\; \mbox{і} \;\;\; b = \sqrt{60 + 10\sqrt{35}}
$$
Таким чином, нам потрібно знайти:
$$
s = a - b;
$$
$$
s^2 = (a - b)^2 = a^2 - 2ab + b^2;
$$
\begin{flalign*}
a^2 = 60 - 10\sqrt{35};\;\;
b^2 = 60 - 10\sqrt{35};\;\;
ab = \sqrt{3600 - 3500} = 10;
\end{flalign*}
Отже,
$$
s^2 = 60 - \cancel{10\sqrt{35}} - 2 \cdot 10 + 60 + \cancel{10\sqrt{35}} = 100;
$$
А звідси $s = \pm 10$.
\\
Очевидно, що 
$$\sqrt{60 - 10\sqrt{35}} < \sqrt{60 + 10\sqrt{35}},$$
а отже і 
$$\sqrt{60 - 10\sqrt{35}} < \sqrt{60 + 10\sqrt{35}} < 0.$$
Значить, за розв'язок задачі беремо $s = -10$.
\vspace{5mm}
\\
\textbf{4.} Обчислити:
$$
\left(\sqrt[6]{3 + 2\sqrt{2}} + \sqrt[3]{1 + \sqrt{2}}\right) \cdot \sqrt[3]{1 - \sqrt{2}}.
$$
Розкриємо дужки і візьмемо до уваги, що $\sqrt[6]{a} \cdot \sqrt[3]{b} = \sqrt[6]{a \cdot b^2}$.
\begin{flalign*}
&\left(\sqrt[6]{3 + 2\sqrt{2}} + \sqrt[3]{1 + \sqrt{2}}\right) \cdot \sqrt[3]{1 - \sqrt{2}} =\\
&= \sqrt[6]{\left(3 + 2\sqrt{2}\right) \cdot \left(1 - \sqrt{2}\right)^2} + \sqrt[3]{\left(1 + \sqrt{2}\right) \cdot \left(1 - \sqrt{2}\right)} =\\
&= \sqrt[6]{\left(3 + 2\sqrt{2}\right) \cdot \left(1 - 2\sqrt{2} + 2\right)} + \sqrt[3]{1 - 2} =\\
&= \sqrt[6]{\left(3 + 2\sqrt{2}\right) \cdot \left(3 - 2\sqrt{2}\right)} - 1 =\\
&= \sqrt[6]{9 - 8} - 1 = 1 - 1 = 0.
\end{flalign*}
\textbf{5.} Обчислити:
$$
\sqrt[3]{2 + \sqrt{5}} + \sqrt[3]{2 - \sqrt{5}}.
$$
Перш за все слід зауважити, що:
$$
2 \pm \sqrt{5} = \left(\frac{1 \pm \sqrt{5}}{2}\right)^3.
$$
Інший спосіб потребує введення заміни:
$$
a = \sqrt[3]{2 + \sqrt{5}} \;\;\; \mbox{і} \;\;\; b = \sqrt[3]{2 - \sqrt{5}}.
$$
Нам потрібно знайти 
\begin{gather*}
s = a + b;\\
s^3 = (a + b)^3 = a^3 + 3ab(a + b) + b^3;\\
a^3 = 2 + \sqrt{5};\;\;
b^2 = 2 - \sqrt{5};\\
ab = \sqrt[3]{\left(2 + \sqrt{5}\right) \cdot \left(2 - \sqrt{5}\right)} =
\sqrt[3]{4 - 5} = -1;
\end{gather*}
Отже,
\begin{flalign*}
&s^3 = 2 + \cancel{\sqrt{5}} + 3 \cdot (-1) \cdot s + 2 - \cancel{\sqrt{5}};\\
&s^3 + 3s - 4 = 0;
\end{flalign*}
Таким чином ми звели наш вираз до рівняння відносно $s$. Розв'яжемо це рівняння:
\begin{flalign*}
&s^3 + 3s - 4 = 0;\\
&s^3 - s + 4s - 4 = 0;\\
&s(s^2 - 1) + 4(s - 1) = 0;\\
&s(s - 1)(s + 1) + 4(s - 1) = 0;\\
&(s - 1)[s(s + 1) + 4] = 0;\\
&(s - 1)(s^2 + s + 4) = 0;\\
\end{flalign*}
Це рівняння має лише один розв'язок $s = 1$. Або, повернувшись до заміни, отримаємо:
$$
s = a + b = \sqrt[3]{2 + \sqrt{5}} + \sqrt[3]{2 - \sqrt{5}} = 1.
$$
Що й потрібно було знайти.
\vspace{5mm}
\\
\textbf{6.} Обчислити:
$$
\sqrt[3]{20 + 14\sqrt{2}} + \sqrt[3]{20 - 14\sqrt{2}}.
$$
Введемо заміну:
$$
a = \sqrt[3]{20 + 14\sqrt{2}} \;\;\; \mbox{і} \;\;\; b = \sqrt[3]{20 - 14\sqrt{2}}.
$$
Нам потрібно знайти 
\begin{gather*}
s = a + b;\\
s^3 = (a + b)^3 = a^3 + 3ab(a + b) + b^3;\\
a^3 = 20 + 14\sqrt{2};\;\;
b^2 = 20 - 14\sqrt{2};\\
ab = \sqrt[3]{\left(20 + 14\sqrt{2}\right) \cdot \left(20 - 14\sqrt{2}\right)} =
\sqrt[3]{400 - 392} = 2;
\end{gather*}
Отже,
\begin{flalign*}
&s^3 = 20 + \cancel{14\sqrt{2}} + 3 \cdot 2 \cdot s + 20 - \cancel{14\sqrt{2}};\\
&s^3 - 6s - 40 = 0;
\end{flalign*}
Таким чином ми звели наш вираз до рівняння відносно $s$. Розв'яжемо це рівняння:
\begin{flalign*}
&s^3 + 6s - 40 = 0;\\
&(s - 4)(s^2 + 4s + 10) = 0;
\end{flalign*}
Це рівняння має лише один розв'язок $s = 4$. Або, повернувшись до заміни, отримаємо:
$$
s = a + b = \sqrt[3]{20 + 14\sqrt{2}} + \sqrt[3]{20 - 14\sqrt{2}} = 4.
$$
Що й потрібно було знайти.
\vspace{5mm}
\\
\textbf{7.} Обчислити:
$$
\frac{\sqrt{8 - 2\sqrt{15}}}{\sqrt{3} - \sqrt{5}}.
$$
Ми можемо скористатися таким виразом:
$$
8 - 2\sqrt{15} =
\left(\sqrt{3}\right)^2 - 2\sqrt{3 \cdot 5} + \left(\sqrt{5}\right)^2 =
\left(\sqrt{3} - \sqrt{5}\right)^2.
$$
Отже,
\begin{flalign*}
\frac{\sqrt{8 - 2\sqrt{15}}}{\sqrt{3} - \sqrt{5}} =
\frac{\sqrt{\left(\sqrt{3} - \sqrt{5}\right)^2}}{\sqrt{3} - \sqrt{5}} =
\frac{\left|\sqrt{3} - \sqrt{5}\right|}{\sqrt{3} - \sqrt{5}} =
\frac{\sqrt{5} - \sqrt{3}}{-\left(\sqrt{5} - \sqrt{3}\right)} = -1.
\end{flalign*}
\textbf{8.} Обчислити:
$$
\frac{\sqrt{48 - 8\sqrt{35}}}{\sqrt{20} - \sqrt{28}}.
$$
Візьмемо до уваги, що:
\begin{gather*}
48 - 8\sqrt{35} = 28 - 2 \cdot 2 \cdot 2 \cdot \sqrt{5 \cdot 7} + 20 =\\
(\sqrt{28})^2 - 2\sqrt{4 \cdot 5 \cdot 4 \cdot 7} + (\sqrt{20})^2 =\\
(\sqrt{28})^2 - 2\sqrt{20 \cdot 28} + (\sqrt{20})^2 =\\
\left(\sqrt{28} - \sqrt{20}\right)^2.
\end{gather*}
Отже,
\begin{flalign*}
&\frac{\sqrt{48 - 8\sqrt{35}}}{\sqrt{20} - \sqrt{28}} =
\frac{\sqrt{\left(\sqrt{28} - \sqrt{20}\right)^2}}{\sqrt{20} - \sqrt{28}} =
\frac{\left|\sqrt{28} - \sqrt{20}\right|}{\sqrt{20} - \sqrt{28}} =\\
&\frac{\sqrt{28} - \sqrt{20}}{-\left(\sqrt{28} - \sqrt{20}\right)} = -1.
\end{flalign*}
\textbf{9.} Обчислити:
$$
\frac{\left(b + 2\right)^2 - 8b}{\sqrt{b} - \frac{2}{\sqrt{b}}}, \;\;\; \mbox{якщо} \;\;\; b = 0,25.
$$
В чисельнику піднесемо до квадрата, а в знаменнику зведемо до спільного знаменника.
\begin{flalign*}
&\frac{\left(b + 2\right)^2 - 8b}{\sqrt{b} - \frac{2}{\sqrt{b}}} =
\frac{\sqrt{b^2 + 4b + 4 - 8b}}{\frac{b - 2}{\sqrt{b}}} =\\
&= \sqrt{(b - 2)^2} \cdot \frac{\sqrt{b}}{b - 2} =
\frac{|b - 2|}{b - 2} \cdot \sqrt{b} =\\
&= \frac{2 - b}{-(2 - b)} \cdot \sqrt{b} = -\sqrt{b}.
\end{flalign*}
Якщо $b = 0,25$, то отримаємо:
$$
-\sqrt{0,25} = -0,5.
$$
\textbf{10.} Обчислити:
\begin{flalign*}
&\frac{\sqrt[3]{25}b^{\frac{1}{3}} - 4}{\sqrt[3]{5}b^{\frac{1}{3}} + 2} - \sqrt[3]{5}b^{\frac{1}{3}} =
\frac{(5b)^{\frac{2}{3}} - 4}{(5b)^{\frac{1}{3}} + 2} - (5b)^{\frac{1}{3}} =
\frac{\left[(5b)^{\frac{1}{3}} - 2\right]\cancel{\left[(5b)^{\frac{1}{3}} + 2\right]}}{\cancel{(5b)^{\frac{1}{3}} + 2}} - (5b)^{\frac{1}{3}} =\\
&= (5b)^{\frac{1}{3}} - 2 - (5b)^{\frac{1}{3}} = -2.
\end{flalign*}
\textbf{11.} Обчислити:
$$
\sqrt{3\left(\sqrt{11} - \sqrt{8}\right)} \cdot \sqrt[4]{19 + 4\sqrt{22}}.
$$
Візьмемо до уваги, що:
\begin{flalign*}
&19 + 4\sqrt{22} = 
11 + 2 \cdot 2 \cdot \sqrt{2 \cdot 11} + 8 = 
11 + 2\sqrt{8 \cdot 11} + 8 =\\
&= \left(\sqrt{11}\right)^2 + 2\sqrt{11} \cdot \sqrt{8} + \left(\sqrt{8}\right)^2 =
\left(\sqrt{11} + \sqrt{8}\right)^2.
\end{flalign*}
Отже,
\begin{flalign*}
&\sqrt{3\left(\sqrt{11} - \sqrt{8}\right)} \cdot \sqrt[4]{19 + 4\sqrt{22}} =
\sqrt{3\left(\sqrt{11} - \sqrt{8}\right)} \cdot \sqrt[4]{\left(\sqrt{11} + \sqrt{8}\right)^2} =\\
&= \sqrt{3\left(\sqrt{11} - \sqrt{8}\right)\left(\sqrt{11} + \sqrt{8}\right)} =
\sqrt{3 \cdot 3} = 3.
\end{flalign*}
\textbf{12.} Обчислити:
$$
\sqrt{18\left(\sqrt{5} - \sqrt{2}\right)} \cdot \sqrt[4]{36\left(7 + 2\sqrt{10}\right)}.
$$
Візьмемо до уваги, що:
\begin{flalign*}
&7 + 2\sqrt{10} = 5 + 2 \cdot \sqrt{5} \cdot \sqrt{2} + 2 =
\left(\sqrt{5}\right)^2 + 2 \cdot \sqrt{5} \cdot \sqrt{2} + \left(\sqrt{2}\right)^2 =\\
&= \left(\sqrt{5} + \sqrt{2}\right)^2.
\end{flalign*}
Отже,
\begin{flalign*}
&\sqrt{18\left(\sqrt{5} - \sqrt{2}\right)} \cdot \sqrt[4]{36\left(7 + 2\sqrt{10}\right)} =
\sqrt{18\left(\sqrt{5} - \sqrt{2}\right)} \cdot \sqrt[4]{36\left(\sqrt{5} + \sqrt{2}\right)^2} =\\
&= \sqrt{18\left(\sqrt{5} - \sqrt{2}\right)} \cdot \sqrt{6\left(\sqrt{5} + \sqrt{2}\right)} =
\sqrt{18 \cdot 6 \cdot 3} = 18.
\end{flalign*}
\textbf{13.} Обчислити:
$$
\left(\sqrt{7} - \sqrt{8}\right) \cdot \sqrt{15 + 4\sqrt{14}}.
$$
Візьмемо до уваги, що:
\begin{flalign*}
&15 + 4\sqrt{14} = 
7 + 2 \cdot 2 \cdot \sqrt{2 \cdot 7} + 8 =
7 + 2 \cdot \sqrt{8} \cdot \sqrt{7} + 8 =\\
&= \left(\sqrt{7}\right)^2 + 2 \cdot \sqrt{7} \cdot \sqrt{7} + \left(\sqrt{8}\right)^2 =
\left(\sqrt{7} + \sqrt{7}\right)^2.
\end{flalign*}
Отже,
\begin{flalign*}
&\left(\sqrt{7} - \sqrt{8}\right) \cdot \sqrt{15 + 4\sqrt{14}} =
\left(\sqrt{7} - \sqrt{8}\right) \cdot \sqrt{\left(\sqrt{7} + \sqrt{8}\right)^2} =\\
&= \left(\sqrt{7} - \sqrt{8}\right)\left(\sqrt{7} + \sqrt{8}\right) = -1.
\end{flalign*}
\textbf{14.} Обчислити:
$$
\left(3\sqrt{2} + \sqrt{10}\right) \sqrt{4\left(7 - 3\sqrt{5}\right)} =
\left(3\sqrt{2} + \sqrt{10}\right) \sqrt{28 - 12\sqrt{5}}.
$$
Візьмемо до уваги, що:
\begin{flalign*}
&28 - 12\sqrt{5} =
18 - 2 \cdot 3\sqrt{2} \cdot \sqrt{5} + 10 =
\left(3\sqrt{2}\right)^2 - 2 \cdot 3\sqrt{2} \cdot \sqrt{5} + \left(\sqrt{10}\right)^2 =\\
&= \left(3\sqrt{2} - \sqrt{10}\right)^2.
\end{flalign*}
Отже,
\begin{flalign*}
&\left(3\sqrt{2} + \sqrt{10}\right) \sqrt{28 - 12\sqrt{5}} =
\left(3\sqrt{2} + \sqrt{10}\right) \sqrt{\left(3\sqrt{2} - \sqrt{10}\right)^2} =\\
&= \left(3\sqrt{2} + \sqrt{10}\right)\left(3\sqrt{2} - \sqrt{10}\right) =
18 - 10 = 8.
\end{flalign*}
\textbf{15.} Обчислити:
$$
\dfrac{a^2 + 1}{a\sqrt{\left(\dfrac{a^2 - 1}{2a}\right)^2  + 1}}, \;\;\; \mbox{якщо} \;\;\; a = -5.
$$
В знаменнику піднесемо до квадрата і потім зведемо до спільного знаменника:
\begin{flalign*}
&\dfrac{a^2 + 1}{a\sqrt{\left(\dfrac{a^2 - 1}{2a}\right)^2  + 1}} =
\dfrac{a^2 + 1}{a\sqrt{\dfrac{a^4 - 2a^2 + 1}{4a^2}  + 1}} =
\dfrac{a^2 + 1}{a\sqrt{\dfrac{a^4 - 2a^2 + 1 + 4a^2}{4a^2}}} =\\
&= \dfrac{a^2 + 1}{\dfrac{a}{2a}\sqrt{a^4 + 2a^2 + 1}} =
\dfrac{a^2 + 1}{\dfrac{a}{2a} \cdot \sqrt{\left(a^2 + 1\right)^2}} =
\dfrac{\cancel{a^2 + 1}}{\dfrac{\cancel{a}}{2\cancel{a}} \cdot \cancel{\left(a^2 + 1\right)}} =\\
&= \dfrac{1}{\dfrac{1}{2}} = 2.
\end{flalign*}
\textbf{16.} Обчислити:
$$
\left(\dfrac{3}{\sqrt{1 + a}} + \sqrt{1 - a}\right) : \left(\dfrac{3}{\sqrt{1 - a^2}} + 1\right),
\;\;\; \mbox{якщо}  \;\;\; a = 0,36.
$$
В обох дужках зведемо до спільного знаменника:
\begin{flalign*}
&\left(\dfrac{3}{\sqrt{1 + a}} + \sqrt{1 - a}\right) : \left(\dfrac{3}{\sqrt{1 - a^2}} + 1\right) =\\
&= \dfrac{3 + \sqrt{(1 - a)(1 + a)}}{\sqrt{1 + a}} : \dfrac{3 + \sqrt{1 - a^2}}{\sqrt{1 - a^2}} =\\
&= \dfrac{\cancel{3 + \sqrt{1 - a^2}}}{\cancel{\sqrt{1 + a}}} \cdot \dfrac{\sqrt{(1 - a)\cancel{(1 + a)}}}{\cancel{3 + \sqrt{1 - a^2}}} = \sqrt{1 - a}.
\end{flalign*}
При $a = 0,36$, отримаємо:
$$
\sqrt{1 - 0,36} = \sqrt{0,64} = 0,8.
$$
\textbf{17.} Обчислити:
$$
\dfrac{\sqrt{x} + 2}{x\sqrt{x} + 2x + 4\sqrt{x}} : \dfrac{1}{x^2 - 8\sqrt{x}}, \;\;\; \mbox{якщо} \;\;\; x = 4,1.
$$
В обох знаменниках винесемо за дужки $\sqrt{x}$:
\begin{flalign*}
&\dfrac{\sqrt{x} + 2}{x\sqrt{x} + 2x + 4\sqrt{x}} : \dfrac{1}{x^2 - 8\sqrt{x}} =
\dfrac{\sqrt{x} + 2}{\cancel{\sqrt{x}}\left(x + 2\sqrt{x} + 4\right)} \cdot \cancel{\sqrt{x}} \cdot \left(x^{\frac{3}{2}} - 2^3\right) =\\
&= \dfrac{\sqrt{x} + 2}{\cancel{x + 2\sqrt{x} + 4}} \cdot \left(x^{\frac{1}{2}} - 2\right)\cancel{\left(x + 2x^{\frac{1}{2}} + 4\right)} =\\
&= \left(\sqrt{x} + 2\right)\left(\sqrt{x} - 2\right) = x - 4.
\end{flalign*}
При $x = 4,1$, отримаємо:
$$
4,1 - 4 = 0,1.
$$
\textbf{18.} Обчислити:
$$
-36 + \dfrac{6 - \sqrt{x}}{36 + x - 6\sqrt{x}} : \dfrac{1}{x\sqrt{x} + 216}, \;\;\; \mbox{} \;\;\; x = 8,1.
$$
Візьмемо до уваги, що:
$$
x\sqrt{x} + 216 = x^{\frac{3}{2}} + 6^3.
$$
Маємо:
\begin{flalign*}
&-36 + \dfrac{6 - \sqrt{x}}{36 + x - 6\sqrt{x}} : \dfrac{1}{x\sqrt{x} + 216} =
-36 + \dfrac{6 - \sqrt{x}}{36 + x - 6\sqrt{x}} \cdot \left(x^{\frac{3}{2}} + 6^3\right) =\\
&= -36 + \dfrac{6 - \sqrt{x}}{36 + x - 6\sqrt{x}} \cdot \left(x^{\frac{1}{2}} + 6\right)\left(x - 6x^{\frac{1}{2}} + 36\right) =\\
&= -36 + \left(6 - \sqrt{x}\right)\left(6 + \sqrt{x}\right) =
-36 + 36 - x = -x.
\end{flalign*}
При $x = 8,1$, отримаємо $-8,1$.
\vspace{5mm}
\\
\textbf{19.} Обчислити:
$$
\left(\dfrac{x^\frac{3}{2} + 8}{x^\frac{1}{2} + 2} - 2x^\frac{1}{2}\right) : \dfrac{x - 4}{12} + \dfrac{48}{x^\frac{1}{2} + 2}, \;\;\; \mbox{якщо} \;\;\; x = 2,1634.
$$
Зведемо до спільного знаменника у дужках:
\begin{flalign*}
&\left(\dfrac{x^\frac{3}{2} + 8}{x^\frac{1}{2} + 2} - 2x^\frac{1}{2}\right) : \dfrac{x - 4}{12} + \dfrac{48}{x^\frac{1}{2} + 2} =\\
&= \dfrac{\left(x^\frac{1}{2} + 2\right)\left(x - 2x^\frac{1}{2} + 2\right) - 2x^\frac{1}{2}\left(x^\frac{1}{2} + 2\right)}{x^\frac{1}{2} + 2} \cdot \dfrac{12}{x - 4} + \dfrac{48}{x^\frac{1}{2} + 2} =\\
&= \dfrac{\cancel{\left(x^\frac{1}{2} + 2\right)}\left(x - 2x^\frac{1}{2} + 4 - 2x^\frac{1}{2}\right)}{\cancel{x^\frac{1}{2} + 2}} \cdot \dfrac{12}{x - 4} + \dfrac{48}{x^\frac{1}{2} + 2} =\\
&= \left(x - 4x^\frac{1}{2} + 4\right) \cdot \dfrac{12}{x - 4} + \dfrac{48}{x^\frac{1}{2} + 2} =\\
&= \left(x^\frac{1}{2} - 2\right)^{\cancel{2}} \cdot \dfrac{12}{\cancel{\left(x^\frac{1}{2} - 2\right)}\left(x^\frac{1}{2} + 2\right)} + \dfrac{48}{x^\frac{1}{2} + 2} =\\
&= \dfrac{12\left(x^\frac{1}{2} - 2\right)}{x^\frac{1}{2} + 2} + \dfrac{48}{x^\frac{1}{2} + 2} =
\dfrac{12x^\frac{1}{2} - 24 + 48}{x^\frac{1}{2} + 2} = \\
&= \dfrac{12\cancel{\left(x^\frac{1}{2} + 2\right)}}{\cancel{x^\frac{1}{2} + 2}} = 12.
\end{flalign*}
\textbf{20.} Обчислити:
$$
\left(\dfrac{a\sqrt{a} + b\sqrt{b}}{\sqrt{a} + \sqrt{b}} - \sqrt{ab}\right) : (a - b) + \dfrac{2\sqrt{b}}{\sqrt{a} + \sqrt{b}}.
$$
Візьмемо до уваги, що:
$$
a\sqrt{a} + b\sqrt{b} = a^\frac{3}{2} + b^\frac{3}{2}.
$$
Отже,
\begin{flalign*}
&\left(\dfrac{a\sqrt{a} + b\sqrt{b}}{\sqrt{a} + \sqrt{b}} - \sqrt{ab}\right) : (a - b) + \dfrac{2\sqrt{b}}{\sqrt{a} + \sqrt{b}} =\\
&=\left(\dfrac{a^\frac{3}{2} + b^\frac{3}{2}}{\sqrt{a} + \sqrt{b} - \sqrt{ab}}\right) \cdot \dfrac{1}{a - b} + \dfrac{2\sqrt{b}}{\sqrt{a} + \sqrt{b}} =\\
&= \left(\dfrac{\cancel{\left(\sqrt{a} + \sqrt{b}\right)}\left(a - \sqrt{ab} + b\right)}{\cancel{\sqrt{a} + \sqrt{b}}} - \sqrt{ab}\right) \cdot \dfrac{1}{a - b} + \dfrac{2\sqrt{b}}{\sqrt{a} + \sqrt{b}} =\\
&= \dfrac{a - 2\sqrt{ab} + b}{a - b} + \dfrac{2\sqrt{b}}{\sqrt{a} + \sqrt{b}} =
\dfrac{\left(\sqrt{a} - \sqrt{b}\right)^2}{\left(\sqrt{a} - \sqrt{b}\right)\left(\sqrt{a} + \sqrt{b}\right)} + \dfrac{2\sqrt{b}}{\sqrt{a} + \sqrt{b}} =\\
&= \dfrac{\sqrt{a} - \sqrt{b}}{\sqrt{a} + \sqrt{b}} + \dfrac{2\sqrt{b}}{\sqrt{a} + \sqrt{b}} =
\dfrac{\sqrt{a} - \sqrt{b} + 2\sqrt{b}}{\sqrt{a} + \sqrt{b}} =
\dfrac{\sqrt{a} + \sqrt{b}}{\sqrt{a} + \sqrt{b}} = 1.
\end{flalign*}
\textbf{21.} Обчислити:
$$
\sqrt{4+\sqrt{x-2}}+\sqrt{x+2+\sqrt{2-x}}.
$$
З ОДЗ можна знайти чому дорівнює $x$:
$$
\left\{
\begin{aligned}
x-2&\geq0,\\
2-x&\geq0;
\end{aligned}
\right.
\Leftrightarrow
\left\{
\begin{aligned}
x&\geq2,\\
x&\leq2;
\end{aligned}
\right.
\Leftrightarrow
x=2.
$$
Підставимо $x=2$ у вираз і отримаємо:
$$
\sqrt{4+0}+\sqrt{2+2+0}=2+2=4.
$$
\textbf{22.} Обчислити:
$$
\sqrt{x-3\cdot\sqrt{x-4}}-\sqrt{9+\sqrt{4-x}}.
$$
З ОДЗ можна знайти чому рівне $x$:
$$
\left\{
\begin{aligned}
x-4&\geq0,\\
4-x&\geq0;
\end{aligned}
\right.
\Leftrightarrow
\left\{
\begin{aligned}
x&\geq4,\\
x&\leq4;
\end{aligned}
\right.
\Leftrightarrow
x=4.
$$
Підставимо $x=4$ у вираз і отримаємо:
$$
\sqrt{4-3\cdot0}-\sqrt{9+0}=2+3=5.
$$
\textbf{23.} Обчислити:
$$
\left(\dfrac{\sqrt{x-1}}{\sqrt{x+1}+\sqrt{x-1}}+\dfrac{x-1}{\sqrt{x^2-1}-x+1}\right):\sqrt{x^2-1}.
$$
Нічого особливого тут видумувати не будемо. Єдине, що кидається одразу в очі, це те, що можна зробити заміну, щоб менше було писанини. Але перед тим запишемо наш вираз в такому вигляді:
$$
\left(\dfrac{\sqrt{x-1}}{\sqrt{x+1}+\sqrt{x-1}}+\dfrac{x-1}{\sqrt{x-1}\cdot\sqrt{x+1}-(x-1)}\right):\sqrt{x-1}\cdot\sqrt{x+1}.
$$
Тепер введемо таку заміну:
$$
a=x-1;\;\;\;b=x+1.
$$
Отримаємо:
\begin{multline*}
\left(\dfrac{\sqrt{a}}{\sqrt{b}+\sqrt{a}}+\dfrac{b}{\sqrt{ab}-a}\right)\cdot\dfrac{1}{\sqrt{ab}}=
\dfrac{a\sqrt{b}-\cancel{a\sqrt{a}}+\cancel{a\sqrt{a}}+a\sqrt{b}}{\cancel{a\sqrt{b}}-a\sqrt{a}+b\sqrt{a}-\cancel{a\sqrt{b}}}\cdot\dfrac{1}{\sqrt{ab}}=\\
=\dfrac{2a\cdot\sqrt{b}}{\sqrt{a}(b-a)}\cdot\dfrac{1}{\sqrt{ab}}=
\dfrac{2a}{a\cdot(b-a)}=\dfrac{2}{b-a}.
\end{multline*}
Повернемося до нашої заміни:
$$
\dfrac{2}{x+1-x+1}=\dfrac{2}{2}=1.
$$